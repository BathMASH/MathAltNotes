\documentclass[a4paper,reqno]{amsart}
\usepackage[english]{babel}
%\usepackage{a4wide}
\usepackage{geometry}
\usepackage{graphicx,color}
\usepackage{pinlabel-test}
%\usepackage{amsthm, amssymb, amsmath, latexsym, upgreek,accents}
%\usepackage{amsfonts, mathrsfs, color}
%\usepackage{a4wide}
%\usepackage{hyperref}
%\usepackage{booktabs}
%\usepackage{textcomp}
%\usepackage{subfigure}

 \geometry{
 a4paper,
% total={170mm,257mm},
right=25mm,
 left=25mm,
 top=25mm,
 bottom = 25mm,
 }


%\usepackage[notcite,notref]{showkeys}
\usepackage{graphicx}
\usepackage{amsmath,amssymb,amsthm, amsgen,mathrsfs}
\usepackage[latin1]{inputenc}
%\usepackage{listings}
\usepackage{color}
\usepackage{dsfont}
\usepackage[usenames,dvipsnames]{xcolor}
\usepackage{rotating}
\usepackage{hhline}
\usepackage{subfigure}
\usepackage{multirow}
\usepackage{enumerate}
\usepackage[bookmarks]{}
\usepackage{amsfonts}   % if you want the fonts
\usepackage{dsfont}
%\usepackage{showkeys}

\numberwithin{equation}{section}
\newtheorem{definition}{Definition}[section]
\newtheorem{lemma}[definition]{Lemma}
\newtheorem{theorem}[definition]{Theorem}
\newtheorem{proposition}[definition]{Proposition}
\newtheorem{corollary}[definition]{Corollary}
\newtheorem{remark}[definition]{Remark}
\newtheorem{example}{Example}
\newtheorem{exercise}[definition]{Exercise}


\DeclareMathOperator{\dist}{dist}
\DeclareMathOperator{\diam}{diam}
\DeclareMathOperator{\id}{id}


\def\eps{\varepsilon}
\def\Omegasp{{\Omega^{\mathrm sp}}}
\def\xx{\mathbf{x}_0}
\def\z0{\mathbf{z}_0}
\def\h{\mathbf{h}}
\def\c{\mathbf{c}}
\def\r{\mathbf{r}}
\def\x{\mathbf{x}}
\def\y{\mathbf{y}}
\def\C{\mathbb{C}}
\def\N{\mathbb{N}}
\def\Q{\mathbb{Q}}
\def\R{\mathbb{R}}
\def\S{\mathbb{S}}
\def\Z{\mathbb{Z}}
\def\calA{\mathcal{A}}
\def\calE{\mathcal{E}}
\def\calF{\mathcal{F}}
\def\calH{\mathcal{H}}
\def\calL{\mathcal{L}}
\def\calM{\mathcal{M}}

\def\curl{\mathop\mathrm{curl}}
\def\dist{\mathop\mathrm{dist}}
\def\Id{\mathrm{Id}}
\def\SBV{\mathrm{SBV}}
\def\SO{\mathrm{SO}}
\def\skw{\mathop{\mathrm{skw}}}
\def\supp{\mathop{\mathrm{supp}}}
\def\sym{\mathop{\mathrm{sym}}}
\def\weak{\rightharpoonup}
\def\weakstar{\overset{\ast}{\rightharpoonup}}

\def\upint{\mathchoice%
    {\mkern13mu\overline{\vphantom{\intop}\mkern7mu}\mkern-20mu}%
    {\mkern7mu\overline{\vphantom{\intop}\mkern7mu}\mkern-14mu}%
    {\mkern7mu\overline{\vphantom{\intop}\mkern7mu}\mkern-14mu}%
    {\mkern7mu\overline{\vphantom{\intop}\mkern7mu}\mkern-14mu}%
  \int}
\def\lowint{\mkern3mu\underline{\vphantom{\intop}\mkern7mu}\mkern-10mu\int}




\renewcommand{\div}{\mathop\mathrm{div}}

%%%%%%%%%%%%%%%%%%Were originally before concise3 - why?
\def\nl{}
\def\bis{}
\def\np{}
\let\h=\emph
\def\sect#1{\newpage\section{#1}}
\def\subsect#1{\subsection{#1}}
\def\slinc#1#2{#1}


\usepackage{hyperref}
\begin{document}
\title[MA20218: Analysis 2A]{MA20218:  Analysis  2A}
%\date{}
\author[]{Lecturer: Karsten Matthies\\University of  Bath, First semester 2019/20 }

\maketitle

\tableofcontents
\thispagestyle{empty}
\newpage
\setcounter{page}{1}
%\vspace*{2cm}

\section*{Chapter 0: Recap from MA10207 and basic results}

\subsection{Sequences and series}

\begin{definition}
Let $(\alpha_k)_{k\in \N}$ be real sequence, namely $\alpha_k\in \R$ for every $k\in \N$. We say that $(\alpha_k)_{k\in \N}$ converges to a limit $L\in \R$, namely $L=\lim_{k\to \infty} \alpha_k$ if, by definition, for every $\varepsilon>0$ there exists $N\in \N$ such that 
$$
|\alpha_k  - L|< \varepsilon \quad \forall\, k\geq N.
$$
Equivalently, this means that 
\begin{equation}\label{e:ls}
L-\varepsilon< \alpha_k  < L+\varepsilon \quad \forall\, k\geq N.
\end{equation}
We say that $(\alpha_k)_{k\in \N}$ diverges to $+\infty$ (or that it diverges positively) if for every $M\in \R$ there exists $N\in \N$ such that 
$$
\alpha_k > M \quad \forall\, k\geq N.
$$
Similarly, we say that $(\alpha_k)_{k\in \N}$ diverges to $-\infty$ (or that it diverges negatively) if for every $M\in \R$ there exists $N\in \N$ such that 
$$
\alpha_k < M \quad \forall\, k\geq N.
$$
\bigskip
\begin{proposition}[Cauchy criterion for convergence]\label{cauchycritseq}
	The sequence $(\alpha_k)_{k\in \N}$ converges if and only if for each $\epsilon >0$, there exists $N$ such that
	$$
	|\alpha_k -\alpha_l| < \epsilon\ \ \forall k,l \geq N.
	$$
\end{proposition}


%Then we can see that while (1) in Proposition \ref{p:limsup} corresponds \underline{exactly} to the upper bound in \eqref{e:ls}, the condition (2) is instead weaker than the lower bound in \eqref{e:ls}.
\end{definition}

\bigskip
In computing limits of real sequences, it is helpful to remember some basic examples:
\bigskip
\begin{proposition} The following limits hold:
	\begin{itemize}
		\item Given $a\in \R$,
		$$
		\lim_{k\to +\infty} a^k = 
		\begin{cases}
		+ \infty \quad & \textrm{if } a>1,\\
		1 \quad & \textrm{if } a=1,\\
		0 \quad & \textrm{if } -1 < a < 1,
		\end{cases}
		$$
		and it does not exist for $a\leq -1$.
		\item For every $a>0$ we have 
		$$
		\lim_{k\to +\infty} a^{\frac1k} =1;
		$$
		\item For $b\in \R$ we have 
		$$
		\lim_{k\to +\infty} k^b = \begin{cases}
		+\infty \quad & \textrm{if } b>0,\\
		0 \quad & \textrm{if } b<0;
		\end{cases}
		$$
		\item For every $b\in \R$:
		$$
		\lim_{k\to +\infty} (k^b)^{\frac1k} =1;
		$$
		\item For every $b>0$ and $a>1$ we have 
		$$
		\lim_{k\to +\infty} \ln k = \lim_{k\to +\infty} k^b = \lim_{k\to +\infty} a^k = \lim_{k\to +\infty} k! = \lim_{k\to +\infty} k^k = +\infty;
		$$
		\item For every $b>0$ and $a>1$ we have 
		$$
		\lim_{k\to +\infty} \frac{\ln k}{k^b} = \lim_{k\to +\infty} \frac{k^b}{a^k} = \lim_{k\to +\infty} \frac{a^k}{k!} = \lim_{k\to +\infty} \frac{k!}{k^k} =0. 
		$$
	\end{itemize}

\bigskip
\begin{remark}
	Recall that, for a non-empty set $A\subset \R $, the infimum of $A$ is denoted by $\inf A$ and is the greatest lower bound for $A$. The infimum (which need not lie in $A$) is an element of $\overline{\R}=\R \cup\{+\infty, -\infty \}$  which is less than  or equal to any element in $A$ and arbitrarily close to elements of $A$.
	Similarly, $\sup A$ (which need not lie in $A$) is the least upper bound for $A$ and is an element of $\overline{\R}$  which is greater than or equal to any element in $A$ and arbitrarily close to elements of $A$.
\end{remark}

Although not every real sequence converges, we can always define the following notions of limit superior (limsup) and limit inferior (liminf) of a sequence:


\begin{definition}[Limit superior and limit inferior]
Let $(\alpha_k)_{k\in \N}$ be a real sequence. We define
\begin{align*}
\liminf_{k\to \infty} \alpha_k
& := \lim_{k\to \infty} \left(\inf\{\alpha_m: m\geq k\}\right) = \sup_{k\in \N} \left\{\inf\{\alpha_m: m\geq k\}\right\}
\end{align*}
 and
 \begin{align*}
\limsup_{k\to \infty} \alpha_k
&:= \lim_{k\to \infty} \left(\sup\{\alpha_m: m\geq k\}\right) = \inf_{k\in \N} \left\{\sup\{\alpha_m: m\geq k\}\right\}
%&=\sup\left(\lim_{m\to \infty} \alpha_{k_m}: (\alpha_{k_m})_m \,\, \text{converging subsequence} \right).
\end{align*}

Note that 
$$
\liminf_{k\to \infty} \alpha_k \leq \limsup_{k\to \infty} \alpha_k
$$
and that  $\liminf_{k\to \infty} \alpha_k = \limsup_{k\to \infty} \alpha_k = L$ if and only if $L=\lim_{k\to \infty} \alpha_k$.
\end{definition}

\bigskip
\begin{proposition}[Properties of $\limsup$ and $\liminf$]\label{p:limsup}
$ $

Let $(\alpha_k)_k$ be a real sequence; let $L_1= \liminf_{k\to \infty} \alpha_k$ and $L_2:= \limsup_{k\to \infty} \alpha_k$. Suppose further that $L_1,L_2\in \R$. Then
\begin{itemize}
\item[(1)] For every $\varepsilon>0$ there exists $N\in \N$ such that 
$$
\alpha_k  > L_1 - \varepsilon \quad \forall\, k\geq N,
$$
namely only a finite number of elements is smaller than $L_1 - \varepsilon$.
\item[(2)] For every $\varepsilon>0$ there exist infinitely many $k\in \N$ such that 
$$
\alpha_k < L_1+\varepsilon.
$$
\end{itemize}
Similarly
\begin{itemize}
\item[(3)] For every $\varepsilon>0$ there exists $N\in \N$ such that 
$$
\alpha_k  < L_2+\varepsilon \quad \forall\, k\geq N,
$$
namely only a finite number of elements is larger than $L_2 + \varepsilon$.
\item[(4)] For every $\varepsilon>0$ there exist infinitely many $k\in \N$ such that 
$$
\alpha_k > L_2-\varepsilon.
$$
\end{itemize}
\end{proposition}
\bigskip
\begin{remark}It can be shown using the last proposition that 
	\begin{align*}
	\limsup_{k\to \infty} \alpha_k&=\mathrm{max}\left\{\lim_{n\to \infty} \alpha_{k_n}: (\alpha_{k_n})_n \,\, \text{is a convergent subsequence} \right\}
	\end{align*}
	so that $\limsup_{k\to \infty} \alpha_k$ is the largest of the subsequential limits of $(\alpha_k)$.
	To see this, let $(\alpha_{k_n})_{n\in\N}$ be a convergent subsequence of $(\alpha_k)_{k\in \N}$ with limit $l$ as $n\rightarrow\infty$.
	Then by Proposition \ref{p:limsup} , for any $\epsilon >0$, there exists $N\in\N$ such that 
	$$
	L_1 -\epsilon  < \alpha_{k_n} < L_2+\epsilon \ \ \forall n\geq N.
	$$
	Passing to the limit $n\rightarrow \infty$ we obtain $L_1-\epsilon \leq l \leq L_2+\epsilon$ and by the arbitrariness of $\epsilon >0$ it follows that $L_1\leq l\leq L_2$.
	 Finally, we can use Proposition \ref{p:limsup} to construct a subsequence converging to $L_2$, it follows then that $L_2$ is itself the limit of a subsequence and hence  $\limsup_{k\rightarrow\infty}(\alpha_k)$ is  the \textbf{largest} subsequential limit of $(\alpha_k)_k$. Similarly,  
	\begin{align*}
	\liminf_{k\to \infty} \alpha_k&= \mathrm{min}\left\{\lim_{n\to \infty} \alpha_{k_n}: (\alpha_{k_n})_n \,\, \text{is a convergent subsequence} \right\}
	\end{align*}
	so that $\liminf_{k\to \infty} \alpha_k$ is the \textbf{smallest} of the subsequential limits of $(\alpha_k)$. 
\end{remark}
\newpage


\end{proposition}


\bigskip


We recall some main results relating to convergence of series in $\R$.
\bigskip
\begin{definition}[Convergence of a series]
	Let $(\alpha_k)$ be a sequence of real numbers, then the series 
	$$
	\sum_{k=1}^\infty \alpha_k
	$$
	is said to \textbf{converge} to the sum $s\in\R$ if and only if  the sequence of partial sums $(s_n)_{n\in \N}$ defined by 
	$$
	s_n=\sum_{k=1}^{n} \alpha _k \ \ \forall n\in\N
	$$ 
	converges to $s$ as $n\rightarrow\infty$.
	
	
	\noindent The series  $\sum_{k=1}^\infty \alpha_k$ is said to be \textbf{absolutely convergent} if $\sum_{k=1}^\infty |\alpha_k|$ is a convergent series. A series which is convergent but not absolutely convergent is said to be \textbf{conditionally convergent}.
\end{definition}

\bigskip
\begin{remark}
Using Proposition \ref{cauchycritseq} on the sequence of partial sums $(s_n)_{n\in\N}$ it follows that	if the series $\sum_{k=1}^{\infty}\alpha _k$ is absolutely convergent, then it is convergent. 
\end{remark}

\bigskip
\begin{theorem}[Vanishing test]
	Let $(\alpha_k)$ be a sequence of real numbers. If the series 
	$$
	\sum_{k=1}^\infty \alpha_k
	$$
	converges (to a finite limit), then 
	$$
	\lim_{k\to \infty} \alpha_k =0.
	$$
\end{theorem}

\bigskip

\begin{theorem}[Comparison test]
	Suppose that $\sum_{k=1}^{\infty}\alpha _k$ and $\sum_{k=1}^{\infty}\beta _k$ are two series satisfying
	$$
	0\leq \alpha_k\leq \beta _k \, \, \forall k\in\N.
	$$
	If $\sum_{k=1}^{\infty}\beta _k$ converges, then $\sum_{k=1}^{\infty}\alpha _k$ converges.
\end{theorem}

\bigskip



\begin{theorem}[Ratio test]
Let $(\alpha_k)$ be a sequence of real numbers such that $\alpha_k$ is nonzero for $k$ sufficiently large and let
$$
L_1 = \liminf_{k\to \infty}\left| \frac{\alpha_{k+1}}{\alpha_k}\right|, \quad \text{and } \quad L_2 = \limsup_{k\to \infty}\left| \frac{\alpha_{k+1}}{\alpha_k}\right|.
$$
Then the series 
$$
\sum_{k=1}^\infty \alpha_k
$$
converges absolutely (to a finite limit) if $0\leq L_2<1$ and diverges or does not converge if $1< L_1\leq +\infty$.
\end{theorem}


\bigskip


\begin{theorem}[Root test]
Let $(\alpha_k)$ be a sequence of real numbers, and let
$$
\gamma = \limsup_{k\to \infty}\left|\alpha_k\right|^{\frac1k}.
$$
Then the series 
$$
\sum_{k=1}^\infty \alpha_k
$$
converges absolutely (to a finite limit) if $0\leq \gamma<1$ and diverges or does not converge if $1<\gamma\leq +\infty$.
\end{theorem}






\subsection{Continuity and integrability}

\begin{definition} Let $A\subset \R$ be a set and $f : A \to \R$ a function. Suppose that $x \in A$. We say that $f$ is continuous at $x$ if
$$
\forall\, \varepsilon \,\, \exists\, \delta>0 \quad \textrm{such that } \, \forall \, y\in A:  |y - x| < \delta \Rightarrow |f (y) - f (x)| < \varepsilon.
$$
Equivalently, in terms of sequences, we have that $f$ is continuous at $x$ if for any sequence $(x_k)_{k\in \N}$ in $A$ with $x=\lim_{k\to \infty} x_k$, we have  
$$
f(x)=\lim_{k\to \infty} f(x_k).
$$
We say that $f$ is continuous if it is continuous at every point of $A$. 

We say that $f$ is uniformly continuous if
$$
\forall\, \varepsilon \,\, \exists\, \delta>0 \quad \textrm{such that } \, \forall \, x, y\in A:  |y - x| < \delta \Rightarrow |f (y) - f (x)| < \varepsilon.
$$
Finally, we say that $f$ is Lipschitz continuous if there exists a number $L > 0$ such that for all $x,y \in A$,
$$
|f(y) - f(x)| \le L |y - x|.
$$
\end{definition}

These definitions may have been given in MA10207 only for functions defined on an interval. The conditions are exactly the same, however, for any set $A\subset \R$. If $A$ is a closed, bounded interval, then continuity has particularly nice consequences.

\begin{theorem}[Theorem of uniform continuity] 
Let $I \subset \R$ be a closed, bounded interval and suppose that $f : I \to \R$ is continuous. Then $f$ is uniformly continuous.
\end{theorem}

\begin{theorem}[Weierstrass extreme value theorem]
Let $I \subset \R$ be a closed, bounded interval and suppose that $f : I \to \R$ is continuous. Then $f$ is bounded and attains its infimum and supremum.
\end{theorem}

\medskip
For Riemann integration, the notation we will use (it might differ slightly from what you have seen last year) is the following.
\bigskip
\begin{definition}
A function $f:[a,b]\ \to \R$ is said to be Riemann integrable on $[a,b]$ if for every $\varepsilon>0$ there exists a subdivision (partition) $\Delta$ of $[a,b]$, 
$\Delta = \{a=x_0<x_1<\dots<x_M=b\}$ such that 
\begin{equation}\label{UL}
U(f,\Delta) - L(f,\Delta) = \sum_{n=1}^M \left(\sup_{I_n} f - \inf_{I_n} f \right) |I_n| <\varepsilon, \quad I_n = x_{n+1}-x_n,
\end{equation}
where $U(f,\Delta), L(f,\Delta)$ denote the  corresponding upper and lower sums respectively. 

We also denote by $\omega(f,I_n):= \sup_{I_n} f - \inf_{I_n} f$ the oscillation of $f$ on $I_n$; in this case \eqref{UL} can be rewritten as 
$$
U(f,\Delta) - L(f,\Delta) = \sum_{n=1}^M \omega(f,I_n)\, |I_n| <\varepsilon.
$$

\end{definition}

\newpage




\section{Chapter 1: Sequences and series of functions}
%{\color{red} Differentiating under the integral. Differentiation and integration of power series.}
In this chapter we combine the major themes in Analysis 1 (sequences\& series and functions): We consider sequences and series of functions, and we investigate convergence concepts and properties of the limits.
\medskip

\noindent
Let $(f_k)_{k\in \mathbb{N}}$ be a sequence of functions $f_k:A \to \R$, with $A\subset \R$, and let $f:A \to \R$. (Usually, $A$ will be an interval.)

\begin{definition}[Pointwise and uniform convergence for sequences]\label{pu-conv}
  Let $(f_k)_{k\in \mathbb{N}}$ be a sequence of functions $f_k:A \to \R$, with $A\subset \R$, and let $f:A \to \R$. The sequence $(f_k)_k$ converges pointwise to $f$ on $A$ as $k\to \infty$ if the sequence $(f_k(x))_{k\in \N}$ converges to $f(x)$ for every $x\in A$; namely if for every $x\in A$ and every $\varepsilon>0$ there exists $N\in \mathbb{N}$ such that for every $k\geq N$ we have $|f_k (x) - f (x)| < \varepsilon$ .

\medskip

The sequence $(f_k)_k$ converges uniformly to $f$ on $A$ if
$$
\forall \,\varepsilon>0 \quad \exists\, N\in \mathbb{N} \quad \textrm{such that } \forall \,k\geq N, \forall\, x\in A: |f_k (x) - f (x)| < \varepsilon.
$$
\end{definition}


\begin{remark}
Note that $f_k$ converges to $f$ uniformly on $A$ if and only if
$$
\lim_{k\to \infty} \left(\sup_{x\in A} |f_k(x)-f(x)|\right)=0.
$$
\end{remark}

Similarly, for series, we have the corresponding convergence concepts (in terms of convergence of their partial sums). %Note in particular that whate

\begin{definition}[Pointwise and uniform convergence for series] Let $(f_k)_{k\in \mathbb{N}}$ be a sequence of functions $f_k:A \to \R$, with $A\subset \R$, and let $s_n:A\to \R$ denote the sequence of partial sums defined as
$$
s_n:= \sum_{k=1}^n f_k, \quad n\in \N.
$$
The series $\sum_{k=1}^\infty f_k$ converges pointwise to a function $f$ on $A$ if the sequence $(s_n)_n$ converges to $f$ pointwise on $A$.

The series $\sum_{k=1}^\infty f_k$ converges uniformly to a function $f$ on $A$ if the sequence $(s_n)_n$ converges to $f$ uniformly on $A$.
\end{definition}

Uniform convergence implies pointwise convergence. The vice versa is not true in general, as shown in the following examples.

\begin{example}\label{example:2}
Let $f_k: [0,1] \to \R$ be defined as
$$
f_k(x)= \frac{kx}{1+k^2x^2}.
$$
Then clearly the sequence $(f_k)_{k\in \N}$ converges pointwise to the function $f:[0,1] \to \R$ defined as $f(x) = 0$ for every $x\in [0,1]$.
The convergence is however not uniform. [Detailed explanation in the lecture.]
\end{example}


\begin{example}
Let $f_k: [0,1] \to \R$ be defined as $f_k(x)=x^k$.

Then clearly the sequence $(f_k)_{k\in \N}$ converges pointwise to the function $f:[0,1] \to \R$ defined as
$$
f(x) = \begin{cases}
0 \quad &\textrm{if } x\neq 1,\\
1 \quad &\textrm{if } x = 1.\\
\end{cases}
$$
The convergence is however not uniform. This is due to the points $x$ close to the endpoint $1$, but $x\neq 1$. Assume for contradiction that $f_k$ converges to $f$ uniformly in $[0,1]$, and let $0<\varepsilon<1$ be
fixed and arbitrary. By definition of uniform convergence, there exists $N\in \N$ such that whenever $k>N$ we have
$$
\sup_{x\in [0,1]} |f_k(x) - f(x)| <\varepsilon.
$$
This means in particular that
$$
x^{N+1} < \varepsilon \quad \forall x\in [0,1);
$$
but this means that $x<\varepsilon^{1/(N+1)}$ for every $x\in [0,1)$, which is clearly false since $0<\varepsilon^{1/(N+1)}<1$ is a fixed number, while $x$ can be chosen arbitrarily close to $1$.

\medskip

The convergence is however uniform in $[0,b]$ for every $0\leq b <1$ (by choosing $N \geq \frac{\log\varepsilon}{\log b}$).
\end{example}

\medskip

A useful criterion for uniform convergence of sequences of functions (due to Cauchy) is as follows.

\begin{theorem}\label{Cauchy:c}
A sequence of functions $(f_k)_{k\in \mathbb{N}}$, $f_k:A\to \R$ converges uniformly on $A$ if and only if for every $\varepsilon>0$ there exists an integer $N$ such that if $j,m\geq N$ and $x\in A$, then
$$
|f_j(x) - f_m(x)| < \varepsilon.
$$
\end{theorem}

\begin{proof}
Assume that $f_k$ converges to $f$ uniformly on $A$. Then, by definition, for every $\varepsilon>0$ there exists $N\in \N$ such that for every $k>N$ we have
$$
|f_k(x) - f(x)|<\frac{\varepsilon}{2} \quad \forall \, x\in A.
$$
By the triangle inequality, for any $j,m>N$ we then have
$$
|f_j(x) - f_m(x)|\leq |f_j(x) - f(x)| + |f_m(x) - f(x)| <\varepsilon.
$$

\medskip
Now, let us assume the Cauchy condition. Let then $\varepsilon>0$ be fixed: by assumption there exists $N\in \N$ such that for every $j,m>N$ we have $|f_j(x) - f_m(x)| <\varepsilon/2$ for every $x\in A$.

\noindent
First, note that this implies that for all $x\in A$ the real sequence $(f_k(x))_k$ is a Cauchy sequence in $\R$, so it converges by the completeness of $\R$. We set
$$
f(x):= \lim_{k\to \infty} f_k(x).
$$
Clearly $(f_k)_k$ converges pointwise to $f$ (by definition of $f$); hence, for every $x\in A$ we can choose $m>N(\varepsilon,x)>N$ such that
$$
|f_m(x) - f(x)|<\frac{\varepsilon}{2}.
$$
By the choice of $m$ and by the uniform Cauchy assumption we have that, for every $j>N$ and every $x\in A$,
$$
|f_j(x) - f(x)| \leq |f_j(x) - f_m(x)| + |f_m(x) - f(x)| <\varepsilon,
$$
implying uniform convergence.

(Note: the fact that $m$ might depend on $x$ is not relevant here: $m$ is chosen as intermediate step of the proof, but the conclusion does not depend on $m$.)

\end{proof}

This result can be translated immediately into an analogous result for series:

\begin{theorem}\label{Cauchy:s}
Let $(f_k)_{k\in \mathbb{N}}$, $f_k:A\to \R$ be a sequence of functions on $A$. The series $\sum_{k=1}^\infty f_k$ converges uniformly on $A$ if and only if for every $\varepsilon>0$ there exists an integer $N$ such that if $j>m\geq N$ and $x\in A$, we have
$$
\left|\sum_{k=m+1}^jf_k(x)\right| < \varepsilon.
$$
\end{theorem}

\begin{example}\label{Cauchy:gs}
We use the Cauchy criterion to show that the geometric series $\sum_{k=0}^\infty x^k$ does not converge uniformly in $(-1,1)$. Assume that it does converge uniformly in $(-1,1)$; then for every
$\varepsilon>0$ there exists $N\in \N$ such that $\forall\, j>m> N$:
$$
\left| \sup_{x\in (-1,1)} \sum_{k=m+1}^j x^k \right| <\varepsilon.
$$
This implies in particular that
\begin{align*}
\varepsilon > \sup_{x\in (0,1)} \sum_{k=m+1}^j x^k > \sup_{x\in (0,1)} \sum_{k=m+1}^j x^j = \sup_{x\in (0,1)} (j-m) x^j \geq \sup_{x\in (0,1)}x^j,
\end{align*}
since for $x\in (0,1)$ we have that $x^k \geq x^j$ for every $k\leq j$. Hence we have that
$$
x^{N+1} < \varepsilon \quad \forall\, x\in (0,1)
$$
which is false, since $\varepsilon^{1/(N+1)}\in (0,1)$ is a fixed number, while $x$ can get arbitrarily close to $1$.

The convergence of the series is however uniform on $[-\rho,\rho]$ for every $0\leq \rho<1$.
\end{example}


For series there is a useful test for uniform convergence, due to Weierstrass. (The name comes from the letter traditionally used to denote the constants, or `majorants', that bound the terms in the series.)

\begin{theorem}[Weierstrass $M$-test]
Let $(f_k)_{k\in \mathbb{N}}$ be a sequence of functions, $f_k:A\to \R$, and suppose that for every $k\in \N$ there exists a constant $M_k>0$ such that
$$
|f_k(x)| \leq M_k \quad \textrm{for every } x\in A, \quad \sum_{k=1}^\infty M_k <\infty.
$$
Then the series $\sum_k f_k$ converges uniformly on $A$.
\end{theorem}

\begin{proof}
Since $\sum_k M_k$ converges, the Cauchy criterion for real series (analogue of Theorem \ref{Cauchy:s} for real series) ensures that for any $\varepsilon>0$ there exists $N\in \N$ such that for $j>m>N$ one has
$$
\sum_{k=m+1}^j M_k < \varepsilon.
$$
Then, by the bound on $\sup_A|f_k|$ we immediately deduce that for $j>m>N$
$$
\Bigg| \sum_{k=m+1}^j f_k(x)\Bigg| \leq \sum_{k=m+1}^j M_k < \varepsilon \quad \forall\, x\in A.
$$
Then the claim follows directly from Theorem \ref{Cauchy:s}.
\end{proof}

This theorem shows the value of the Cauchy condition: We can prove uniform convergence of the series without having to know what its sum is.

\begin{remark}[Absolute convergence]
If uniform convergence of a series is proved by means of the Weierstrass $M$-test, then the series also converge absolutely, namely
$$
\sum_{k=1}^\infty |f_k(x)| <\infty \quad \forall \, x\in A.
$$
(Thus the Weierstrass $M$-test is not applicable to series that converge uniformly but not absolutely.)

Note that uniform convergence does not imply absolute convergence: Consider the trivial example
$$
f_k:\R\to \R, \quad f_k(x) = \frac{(-1)^{k+1}}{k}.
$$
Then $\sum_k f_k$ converges on $\R$ to the constant function $f(x)=c=\log 2$, uniformly since the series does not depend on $x$, but the convergence is not absolute.
\end{remark}


\subsection{Uniform convergence and continuity, integration and differentiation}

One of the main questions in relation to convergence of sequences and series is whether important properties of the functions $f_k$, like continuity, integrability and differentiability, are preserved in the limit process. Also, what is the relation between $f'_k$ and $f'$, or between the integral of $f_k$ and the integral of $f$?
\medskip

First of all, the uniform limit of continuous functions $f_k:[a,b]\to \R$ is continuous. On the other hand, pointwise convergence does not preserve continuity. (See Example \ref{example:2})


\begin{theorem}\label{unif:cont}
Let $(f_k)_{k\in \mathbb{N}}$ be a sequence of continuous functions $f_k:[a,b]\to \R$, and assume that $f_k\to f$ uniformly as $k \to \infty$, where $f: [a,b]\to \R$. Then $f$ is  continuous on $[a,b]$.
\end{theorem}

\begin{proof}
Let $x_0\in [a,b]$ be an arbitrary point; we claim that $f$ is continuous at $x_0$, namely that for every $\varepsilon>0$ there exists $\delta>0$ such that
$$
\text{for every } x\in [a,b], \, |x-x_0| < \delta \quad \Rightarrow \quad |f(x)-f(x_0)|<\varepsilon.
$$
Since, by assumption, $f_k\to f$ uniformly as $k \to \infty$, for the $\varepsilon$ fixed above there exists $N\in \N$ such that for every $k\geq N$,
\begin{equation}\label{infty:norm}
\sup_{x\in [a,b]} |f_k(x)-f(x)| < \frac{\varepsilon}3.
\end{equation}
On the other hand, since by assumption $f_N$ is continuous, for the $\varepsilon$ fixed above there exists $\delta>0$ such that
\begin{equation}\label{cont:N}
\text{for every } x\in [a,b], \, |x-x_0| < \delta \quad \Rightarrow \quad |f_N(x)-f_N(x_0)|<\frac{\varepsilon}3.
\end{equation}
By combining \eqref{infty:norm} and \eqref{cont:N} we then conclude that, whenever $x\in [a,b]$ is such that $|x-x_0| < \delta$ we have
\begin{align*}
 |f(x)-f(x_0)| \leq  |f(x)-f_N(x)| +  |f_N(x)-f_N(x_0)| +  |f_N(x_0)-f(x_0)| \leq \varepsilon,
\end{align*}
which proves the claim.
\end{proof}

\begin{remark}
The continuity of the limit can be phrased in a more suggestive way, in terms of `exchange of limits': for every $x\in [a,b]$:
$$
\lim_{t\to x} \lim_{k\to \infty} f_k(t) = \lim_{k\to \infty}\lim_{t\to x} f_k(t).
$$
We will see that this `commutativity' is common to many results related to uniform convergence.
\end{remark}

%We also have another continuity result, under the assumption of pointwise convergence and monotonicity:
%
%
%\begin{theorem}\label{pointwisemonotone}
%Let $(f_k)_{k\in \mathbb{N}}$ be a sequence of continuous functions $f_k:[a,b]\to \R$. Suppose that for every $x\in [a,b]$ the sequence $(f_k(x))_{k\in \N}$ is decreasing (namely $f_k(x)\geq f_{k+1}(x)$ for every $x$ and every $k$) and that $(f_k)_{k\in \N}$ converges pointwise to a continuous function $f:[a,b]\to \R$ as $k\to \infty$. Then $(f_k)_{k\in \N}$ converges to $f$ uniformly in $[a,b]$.
%\end{theorem}
%
%\begin{remark}
%In the theorem above, the fact that the functions are defined on a closed and bounded set (a compact of $\R$) is crucial. Indeed, the sequence
%$$
%f_k(x):= \frac{1}{kx+1}, \quad x\in (0,1), k\in \N,
%$$
%converges pointwise to $0$ in $(0,1)$ and is decreasing, but the convergence is not uniform.
%\end{remark}


We now address the question of integrability.


\begin{theorem}\label{limit:integral}
Let $(f_k)_{k\in \mathbb{N}}$ be a sequence of Riemann integrable functions on $[a,b]$ converging uniformly to a function $f : [a,b] \to \R$. Then $f$ is Riemann integrable and
$$
\int_a^b f_k(x)dx \to \int_a^b f(x)dx
$$
as $k \to \infty$.
\end{theorem}


\begin{proof}
Let $\varepsilon > 0$ and fix a number $N \in \mathbb{N}$ such that for all $k \geq N$ and all $x \in [a, b]$, we have $|f_k (x) - f (x)| < \varepsilon$.
By the definition of Riemann integrability, we find a subdivision $\Delta$ of $[a,b]$ such that $U(f_N,\Delta) - L(f_N,\Delta) < \varepsilon$. Let $I_1,...,I_M$ be the intervals of
$\Delta$. Then for every $n = 1,...,M$, we have
\begin{align*}
\omega(f, I_n) &= \sup_{I_n} f - \inf_{I_n} f\\
&\leq \sup_{I_n} f_N + \varepsilon - \inf_{I_n} f_N  + \varepsilon\\
&= \omega(f_N , I_n) + 2\varepsilon.
\end{align*}
It follows that
\begin{align*}
U(f,\Delta) - L(f,\Delta) &= \sum_{n=1}^M \omega(f, I_n) |I_n| \leq \sum_{n=1}^M \omega(f_N, I_n) |I_n| + 2\varepsilon\sum_{n=1}^M |I_n| \\
& = U(f_N,\Delta) - L(f_N,\Delta) + 2\varepsilon (b-a) < \varepsilon (1+ 2b - 2a).
\end{align*}
The right-hand side can be made arbitrarily small. Thus $f$ is Riemann integrable.
Moreover, by the linearity and by the monotonicity of the integral we have
\begin{align*}
\left|\int_a^b f_k(x)dx - \int_a^b f(x)dx\right| &= \left|\int_a^b \Big(f_k(x) - f(x)\Big)dx \right|\\
&\leq \int_a^b \Big|f_k(x) - f(x)\Big|dx\\
&\leq(b-a) \sup_{x\in [a,b]} \Big|f_k(x) - f(x)\Big| \to 0\\
\end{align*}
as $k\to \infty$. Therefore, we have the desired convergence.
\end{proof}


\begin{remark}
The conclusion of the theorem can be written in the form
$$
\lim_{k\to \infty}\int_a^b f_k(x)dx = \int_a^b \lim_{k\to \infty} f_k(x)dx.
$$
So we can summarise the theorem as follows: if we have uniform convergence,
then we can `exchange' the integral with the limit.
\end{remark}

\begin{corollary}\label{series:i}
If $\sum_{k=1}^\infty f_k$ is a uniformly convergent series of Riemann integrable functions $f_k : [a, b] \to \R$, then the sum of the series is Riemann integrable and
$$
\int_a^b\sum_{k=1}^\infty f_k(x) dx = \sum_{k=1}^\infty \int_a^b f_k(x) dx.
$$
\end{corollary}

In other words, the series may be integrated term by term (or, equivalently, we can exchange the operations of integration and sum).



\medskip

Finally, we address the issue of differentiability. It is easy to see that uniform convergence of a sequence of functions $f_k:[a,b] \to \R$ implies nothing about the sequence
$(f'_k)_{k\in \N}$.

\begin{example}
Let $f_k:[-1,1]\to \R$, $f_k(x) = \sqrt{\frac{1}{k^2}+x^2}$ for every $k\in \N$.

The functions $f_k$ are differentiable for every $k$. Note that $f_k\to f$ uniformly as $k\to +\infty$, where $f(x)=|x|$; indeed
$$
|x|\leq f_k(x) \leq |x|+\frac1k \quad \forall \,k, \, \forall\, x\in [-1,1],
$$
hence
$$
\sup_{x\in [-1,1]} |f_k(x)-f(x)| \leq \frac1k,
$$
and the convergence is therefore uniform. The limit function $f$ is however not differentiable in $(-1,1)$.

%(For another example see also Problem Sheet 1, Homework exercise 2.)
\end{example}



\begin{theorem}\label{uniform:differentiability}
Let $(f_k)_{k\in \N}$ be a sequence of functions $f_k:[a,b]\to \R$ of class $C^1$ in $(a,b)$ (namely differentiable in $(a,b)$ and with continuous derivative in $(a,b)$). Suppose that
\begin{itemize}
\item[(i)] there exists a number $x_0 \in (a, b)$ such that the sequence $(f_k(x_0))_{k\in \mathbb{N}}$ is convergent, and
\item[(ii)] $(f'_k)_{k\in \mathbb{N}}$ converges uniformly on $(a,b)$.
\end{itemize}
Then there exists a continuously differentiable function $f : (a, b) \to \R$ such that $f_k \to f$ uniformly and $f'_k \to f'$ uniformly as $k \to \infty$.
\end{theorem}

\begin{proof}
Let $y_0:=\lim_{k\to\infty} f_k(x_0)$ and let $g : (a, b) \to \R$ be the uniform limit of $f'_k$. Then $g$ is continuous, being the uniform limit of continuous functions.
Define
$$
f(x):= y_0 + \int_{x_0}^x g(t) dt, \quad x\in (a,b).
$$
Then, by the Fundamental Theorem of Calculus, $f'=g$. Moreover, for any $x\in (a,b)$,
\begin{align*}
|f_k(x) - f(x)|&= \left|f_k(x_0)+\int_{x_0}^x f'_k(t) dt - y_0 - \int_{x_0}^x g(t) dt\right|\\
&\leq |f_k(x_0)-y_0|+ \left| \int_{x_0}^x \Big(f'_k(t) - g(t)\Big) dt \right|\\
&\leq |f_k(x_0)-y_0|+  \int_{x_0}^x \Big|f'_k(t) - g(t)\Big| dt \\
&\leq |f_k(x_0)-y_0|+  (b-a)\sup_{t\in (a,b)}\Big|f'_k(t) - g(t)\Big| \to 0.
\end{align*}
Hence $f_k \to f$ uniformly. We already know that $f'_k \to g = f'$ uniformly.
\end{proof}

\begin{remark}
If, in Theorem \ref{uniform:differentiability}, one replaces (i) with the stronger condition that $f_k$ converges pointwise to a function $f$ on $(a,b)$, then the proof simplifies as follows.

Fix some $x_0\in (a,b)$; the uniform convergence of $f'_k$ to $g$ implies that
$$
\int_{x_0}^x f'_k(t) dt \to \int_{x_0}^x g(t) dt \quad \textrm{as } k\to \infty,
$$
by Theorem \ref{limit:integral}. It then follows from the Fundamental Theorem of Calculus that
$$
f_k(x) - f_k(x_0) \to  \int_{x_0}^x g(t) dt \quad \textrm{as } k\to \infty.
$$
The pointwise convergence of $f_k$ to $f$ then gives
$$
f(x) - f(x_0) =  \int_{x_0}^x g(t) dt,
$$
which, again by the Fundamental Theorem of Calculus implies that $f$ is differentiable and $f'=g$.

\end{remark}

\begin{remark}
There is another version of Theorem \ref{uniform:differentiability} under weaker assumptions (only differentiability of $f_k$, no continuity for $f'_k$, but still (i) and (ii)) that guarantees the convergence of the derivatives (but clearly $f'$ will not be continuous in general). The proof however requires more care, as the use of the fundamental theorem is not allowed.
\end{remark}


Similarly, for series, one can exchange the operations of derivative and sum, under the following assumptions.

\begin{theorem}[Differentiation for series]\label{series:d}
Let $(f_k)_{k\in \N}$ be a sequence of functions $f_k:[a,b]\to \R$ of class $C^1$ in $(a,b)$ and assume that the two series of functions
$$
\sum_{k=1}^\infty f_k \quad \textrm{and} \quad \sum_{k=1}^\infty f_k'
$$
converge uniformly to functions $f$ and $g$, respectively. Then $f$ is differentiable in $(a,b)$, its derivative is continuous in $(a,b)$ and
$$
f'=g.
$$

More in general, if the functions $f_k$ are $m$-times differentiable with continuity ($f_k\in C^m(a,b)$) and the series
$$
\sum_{k=1}^\infty f_k, \quad \sum_{k=1}^\infty f_k',  \dots , \textrm{and }  \, \sum_{k=1}^\infty f_k^{(m)}
$$
converge uniformly to functions $f$, $g_1$,..., and $g_m$, respectively, then $f$ is $m$-times differentiable with continuity in $(a,b)$ and
$$
f^{(i)}=g_i.
$$
\end{theorem}

\begin{remark}
Again, the conclusion of Theorem \ref{series:d} is valid under weaker assumptions. Indeed it is enough to have convergence of the series $\sum_{k=1}^\infty f_k(x_0)$ for some $x_0\in (a,b)$ and
uniform convergence of the series of derivatives, in the spirit of Theorem \ref{uniform:differentiability}.
\end{remark}

%\begin{remark}
%In {\color{red}MA20223} you will encounter a very special series of functions, the Fourier series, constructed in terms of trigonometric functions.
%\end{remark}

As a last result, we analyse the `exchange' of integration and differentiation. We first need some preparation.

Let $f:[a,b]\times \R \to \R$, $(x,t)\mapsto f(x,t)$. The function $f$ depends on two variables. We can extend to such $f$ the concepts of continuity, uniform continuity and differentiability that we introduced for
functions of one variable.

\begin{definition}[Continuity]
A function $f:[a,b]\times\R \to \R$ is said to be continuous at $(x_0,t_0) \in [a,b]\times\R$ if for very $\varepsilon>0$ there exists $\delta =\delta(\varepsilon,x_0,t_0)$ such that for every $(x,t)\in [a,b]\times \R$
$$
\|(x,t)-(x_0,t_0)\|<\delta \quad \Rightarrow \quad |f(x,t) - f(x_0,t_0)| < \varepsilon,
$$
where $\|(x,t)-(x_0,t_0)\| = \sqrt{(x-x_0)^2+(t-t_0)^2}$.

\bigskip
A function $f:[a,b]\times \R \to \R$ is said to be uniformly continuous if for every $\varepsilon>0$ there exists $\delta =\delta(\varepsilon)$ such that for every $(x,t), (y,s)\in [a,b]\times \R$:
$$
\|(x,t)-(y,s)\|<\delta \quad \Rightarrow \quad |f(x,t) - f(y,s)| < \varepsilon,
$$
where $\|(x,t)-(y,s)\| = \sqrt{(x-y)^2+(t-s)^2}$.

\medskip
In particular, a continuous function on a `compact' set $[a,b]\times [c,d]$ is uniformly continuous.
\end{definition}



\begin{definition}
Let $f:[a,b]\times \R \to \R$ be a given function. We say that $f$ is partially differentiable in $t$ at the point $(x_0,t_0)\in (a,b)\times \R$ if the following limit exists:
$$
\lim_{h\to 0} \frac{f(x_0,t_0+h) - f(x_0,t_0)}{h};
$$
In this case we set
$$
\frac{\partial f}{\partial t} (x_0,t_0) = \lim_{h\to 0} \frac{f(x_0,t_0+h) - f(x_0,t_0)}{h},
$$
and we say that $\frac{\partial f}{\partial t} (x_0,t_0)$ is the partial derivative of $f$ with respect to $t$ at $(x_0,t_0)$.

\end{definition}

\begin{lemma}
Let $f: [a,b]\times \R \to \R$, $(x,t)\mapsto f(x,t)\in \R$, be a continuous function. Then
$$
F:\R\to \R, \quad F(t):= \int_a^b f(x,t) dx
$$
is continuous.
\end{lemma}

\begin{proof}
Let $t_0\in \R$ and $r>0$, and let $\varepsilon>0$. Since the function $f$ is uniformly continuous in $[a,b]\times [t_0-r, t_0+r]$, we have that there exists $\delta=\delta(\varepsilon,t_0)$ such that
$$
\|(x,t)-(y,s)\|<\delta \, \Rightarrow \, |f(x,t)-f(y,s)|<\varepsilon
$$
for every $(x,t),(y,s) \in [a,b]\times [t_0-r, t_0+r]$. Now let $h\in \R$, with $|h|<\min\{r,\delta\}$, and consider
$$
F(t_0+h) - F(t_0) = \int_a^b \big(f(x,t_0+h) - f(x,t_0)\big) dx.
$$
Since, clearly, $\|(x,t_0+h) - (x,t_0)\| \leq |h|$, by uniform continuity we have
$$
|f(x,t_0+h) - f(x,t_0)|<\varepsilon.
$$
This implies, in terms of $F$, that
$$
|F(t_0+h)-F(t_0)| \leq (b-a) \varepsilon,
$$
which gives the continuity of $F$ at $t_0$.
\end{proof}

\begin{theorem}
Let $f: [a,b]\times \R \to \R$, $(x,t)\mapsto f(x,t)\in \R$, be a continuous function, and assume that the partial derivative
$\frac{\partial f}{\partial t}$ is also continuous. Then the function
$$
F:\R\to \R, \quad F(t):= \int_a^b f(x,t) dx
$$
is differentiable, and
$$
F'(t) = \frac{d}{dt}  \int_a^b f(x,t) dx =  \int_a^b \frac{\partial f}{\partial t}(x,t) dx.
$$
\end{theorem}

\begin{proof}
Let $t_0\in \R$ and $r>0$, and let $\varepsilon>0$. Since the function $\frac{\partial f}{\partial t}$ is uniformly continuous in $[a,b]\times [t_0-r, t_0+r]$, we have that there exists $\delta=\delta(\varepsilon,t_0)$ such that
$$
\|(x,t)-(y,s)\|<\delta \, \Rightarrow \, \left|\frac{\partial f}{\partial t}(x,t)-\frac{\partial f}{\partial t}(y,s)\right|<\varepsilon
$$
for every $(x,t),(y,s) \in [a,b]\times [t_0-r, t_0+r]$. We claim that
\begin{align*}
\left|\frac{F(t_0+h) - F(t_0)}{h} - \int_a^b \frac{\partial f}{\partial t}(x,t_0) dx \right|<\varepsilon \quad \forall \, |h|<\min\{r,\delta\}.
\end{align*}
(Clearly this estimate implies the claim, by the definition of differential of $F$ at $t_0$.)

Expanding the expression above we have that, whenever $|h|<\min\{r,\delta\}$,
\begin{align*}
\left|\frac{F(t_0+h) - F(t_0)}{h} - \int_a^b \frac{\partial f}{\partial t}(x,t_0) dx\right| \leq
\int_a^b \left| \frac{f(x,t_0+h) - f(x,t_0)}{h} - \frac{\partial f}{\partial t}(x,t_0) \right| dx.
\end{align*}
By the Mean Value Theorem applied to the function $t\mapsto f(x,t)$ at fixed $x$ we have that there exists $\theta = \theta(h,x)$ such that
$$
\frac{f(x,t_0+h) - f(x,t_0)}{h} =  \frac{\partial f}{\partial t}(x,t_0+\theta), \quad \textrm{with }\, |\theta|<|h|.
$$
So we have
\begin{align*}
\left|\frac{F(t_0+h) - F(t_0)}{h} - \int_a^b \frac{\partial f}{\partial t}(x,t_0) dx\right| \leq
\int_a^b \left| \frac{\partial f}{\partial t}(x,t_0+\theta)- \frac{\partial f}{\partial t}(x,t_0) \right| dx.
\end{align*}
Note that, since $\|(x,t_0+\theta) - (x,t_0)\|\leq |\theta|<|h|<\delta$, by uniform continuity
$$
\left| \frac{\partial f}{\partial t}(x,t_0+\theta)- \frac{\partial f}{\partial t}(x,t_0) \right| <\varepsilon,
$$
and hence
\begin{align*}
\left|\frac{F(t_0+h) - F(t_0)}{h} - \int_a^b \frac{\partial f}{\partial t}(x,t_0) dx\right| \leq (b-a)\varepsilon
\end{align*}
as claimed.
\end{proof}

%%%%%%%%%%%%%%%%

\subsection{Power series}
A special type of series of functions is given, for $y\in \R$, by the power series:
\begin{equation}\label{powerx0}
\sum_{k=0}^\infty a_k(y-y_0)^k, \quad a_k \in \R, y_0\in \R,
\end{equation}
or, with no loss of generality (e.g. by setting $x=y-y_0$),
\begin{equation}\label{power0}
\sum_{k=0}^\infty a_k x^k.
\end{equation}
Obviously the series \eqref{power0} converges at $x=0$. This could be the only point where the series converges, as for example in the case of the series
$$
\sum_{k=0}^\infty k^k x^k.
$$

\begin{theorem}\label{unifpt_ps}
If the power series
$$
\sum_{k=0}^\infty a_k x^k
$$
converges at some point $y\in \R$, then it converges (pointwise) at every $x\in \R$ with $|x|< |y|$, and uniformly in $[-R,R]$, for every $R<|y|$.
\end{theorem}

\begin{proof}
Assume that $y\neq 0$; since by assumption the (real) series
$$
\sum_{k=0}^\infty a_k y^k
$$
converges, we have that the (real) sequence $(a_k y^k)_k$ converges to zero as $k\to \infty$. In particular, the sequence $(a_k y^k)_k$ is bounded, namely there exists
$M>0$ such that $|a_k y^k| \leq M$ for every $k\in \N$.

Now, let $x\in \R$; since $y\neq 0$, we have
\begin{align}\label{stima:ps}
|a_k x^k| = \left| a_k x^k \, \frac{y^k}{y^k} \right| = |a_k y^k| \, \left| \frac{x}{y}\right|^k \leq M \, \left| \frac{x}{y}\right|^k;
\end{align}
hence, if $|x|<|y|$, by \eqref{stima:ps} we deduce the bound
$$
\sum_{k=0}^\infty a_k x^k \leq \sum_{k=0}^\infty |a_k x^k| \leq M \sum_{k=0}^\infty \left| \frac{x}{y}\right|^k,
$$
and the series at the right-hand side converges since $\left| \frac{x}{y}\right| <1$. By the comparison principle we have that
$$
\sum_{k=0}^\infty a_k x^k
$$
converges for every $x\in \R$ with $|x|<|y|$.

To show uniform convergence, we will use the Weierstrass $M$-test. Let $R\geq 0$ be such that $R<|y|$; then by \eqref{stima:ps} we obtain the uniform bound
$$
\sup_{|x|\leq R}  |a_k x^k| \leq M \frac{R^k}{|y|^k} =:M_k
$$
where $M_k$ is independent of $x$, and $\sum_k M_k<\infty$; the convergence of the series is then uniform on $[-R,R]$.

\end{proof}



\begin{remark}
Note that if the series \eqref{power0} does not converge at $y$, it will not converge for any $x$ with $|x|>|y|$. The set of points $x\in \R$ where the series converges is then an
interval centred at $0$ (or centred at $y_0$ for the series \eqref{powerx0}).
\end{remark}

\begin{definition}
We say that
$$
r = \sup \{|x|\geq 0: \eqref{power0} \, \textrm{converges}\}
$$
is the radius of convergence of the series. The interval of convergence is the open interval $(-r,r)$.
\end{definition}

\begin{example}
The series
$$
\sum_{k=0}^\infty k^k x^k
$$
has radius of convergence $r=0$; the series
$$
\sum_{k=0}^\infty \frac{x^k}{k!}
$$
has radius of convergence $r=\infty$.
The series
$$
\sum_{k=0}^\infty  x^k, \quad \sum_{k=0}^\infty  \frac{x^k}{(k+1)^2}, \quad \sum_{k=0}^\infty  \frac{x^k}{k+1}
$$
all have $r=1$. However, the first series does not converge at $x=\pm1$; the second one converges at both $x=-1$ and $x=1$; the third series converges at $x=-1$ and diverges at $x=1$.

So the endpoints of the interval of convergence of the series have to be checked separately.
\end{example}

The following theorem gives a way of computing the convergence radius of a series:

\begin{theorem}\label{thm:pointwisePS}
Let
$$
\sum_{k=0}^\infty a_k x^k
$$
be a power series, and let
$$
L:= \limsup_{k\to \infty} (|a_k|)^{\frac1k}.
$$
Then the radius of convergence of the series is
$$
r =
\begin{cases}
0 \quad &\textrm{if } L=+\infty\\
+\infty \quad &\textrm{if } L=0\\
\frac1L \quad &\textrm{if } 0<L<+\infty.
\end{cases}
$$
\end{theorem}

\begin{proof}
Assume that $0<L<+\infty$, and let $x\in \R$ with $|x|<\frac1L$. By the definition of $L$ we have that, for every $\varepsilon>0$ there exists $N\in \N$ with the property that
$$
\forall\, k\in \N, \quad  k\geq N \quad \Rightarrow \quad (|a_k|)^{\frac1k} < L+\varepsilon.
$$
So, if we choose $\varepsilon>0$ such that $c:=|x|(L+\varepsilon)<1$, we have that there exists $N\in \N$ such that if $k\geq N$ then
$$
|a_k x^k| = |a_k| |x|^k < (L+\varepsilon)^k |x|^k = (|x|(L+\varepsilon))^k = c^k.
$$
Since $\sum_{k=0}^\infty c^k <\infty$, this proves that the series $\sum_{k=0}^\infty a_k x^k$ converges whenever $|x|< \frac1L$, and hence $r\geq \frac1L$. (Note that we proved that the tails of the power series are controlled by the tails of the
geometric converging series. To conclude we just observe that each of the first $N$ terms is bounded by the constant $\max\{a_k r^k: k=1, \dots, N\} <\infty$.)

To conclude the proof we need to show that the series does not converge for $|x|>\frac1L$. Let then $x\in \R$ be such that $|x|>\frac1L$; by the definition of $L$ we have that, for every $\varepsilon>0$
there exist infinitely many $k\in \N$ such that
$$
(|a_k|)^{\frac1k} > L-\varepsilon.
$$
If we choose $\varepsilon>0$ such that $|x|(L-\varepsilon)> 1$ we then have that for infinitely many $k\in \N$
$$
|a_k x^k| = |a_k| |x|^k > (L-\varepsilon)^k |x|^k = (|x|(L -\varepsilon))^k >1.
$$
This proves that the series does not converge, since a necessary condition for convergence is that $a_k x^k \to 0$ as $k\to \infty$, while for infinitely many $k\in \N$ we have that  $|a_k x^k|>1$, which implies that at least a subsequence of $a_k x^k$ will not converge to zero.
\smallskip

(The cases $L=0$ and $L=+\infty$ are left as exercise.)
\end{proof}

\bigskip

If the power series \eqref{power0} has radius of convergence $r>0$, then in the interval $(-r,r)$ the function
$$
f(x) = \sum_{k=0}^\infty a_k x^k
$$
enjoys lots of properties, as the following theorem shows.


%More in general we have the following theorem.

\begin{theorem}\label{power:deriv}
Suppose that the power series
\begin{equation}\label{l:ps}
\sum_{k=0}^\infty a_k x^k
\end{equation}
has radius of convergence $0<r<\infty$. Then the function $f$ defined as
$$
f(x):=\sum_{k=0}^\infty a_k x^k, \quad |x|<r,
$$
is continuous and differentiable in $(-r,r)$, with
$$
f'(x) = \sum_{k=1}^\infty k a_k x^{k-1}, \quad |x|<r.
$$
\end{theorem}


\begin{proof}
Let $\varepsilon>0$ be arbitrary. By Theorem \ref{unifpt_ps} we know that the series converges uniformly on $[-r+\varepsilon,r-\varepsilon]$ and $f$ is therefore continuous in that interval, as the
uniform limit of a continuous sequence (the partial sums of the power series). Moreover, by the arbitrariness of $\varepsilon$ the function $f$ is well-defined and continuous for $|x|<r$.

Now, to prove differentiability, we first show that the series
\begin{equation}\label{power:d}
\sum_{k=1}^\infty k a_k x^{k-1}
\end{equation}
converges in the same interval, namely for $|x|<r$.

To show this, let $0<|x|<r$, and let $x_0$ be such that $0<|x|<|x_0|<r$, so that $\big|\frac{x}{x_0}\big|=\rho<1$; since $\sum_{k=0}^\infty a_k x_0^k$ converges, we have that
$a_k x_0^k\to 0$ as $k\to \infty$. This implies that there exists a constant $M$ such that $|a_kx_0^k|<M$ for every $k$. As a consequence, for every $k\in \N$
\begin{align*}
|k a_k x^{k-1}| & = \left| k a_k x^{k-1} \frac{x_0^k}{x_0^k}\right| = \left|\frac{k}{x_0} a_k x_0^k \frac{x^{k-1}}{x_0^{k-1}}\right|\leq \frac{k}{|x_0|}\, |a_kx_0^k| \rho^{k-1} \leq \frac{k}{|x_0|} M \rho^{k-1}.
\end{align*}
Note that the series
$$
\sum_{k=1}^\infty \frac{k}{|x_0|} M \rho^{k-1}
$$
converges, by the ratio test, since
$$
\frac{k+1}{|x_0|} M \rho^{k} \frac{|x_0| }{k M \rho^{k-1}} = \frac{k+1}{k} \rho \to \rho <1;
$$
hence, by comparison, the series \eqref{power:d} converges for $|x|<r$. This implies that, denoting with $r'$ the radius of convergence of \eqref{power:d}, $r'\geq r$.

Suppose that $r'>r$ and choose $r<|x|<r'$. For this $x$ the series \eqref{power:d} converges, while the power series \eqref{l:ps} does not converge. But on the other hand
$$
|a_k x^k| = |k a_k x^{k-1}| \frac{|x|}k \leq |k a_k x^{k-1}| \quad \textrm{for } |x|<k.
$$
Note that $r'<k$ for $k$ large enough, which means that the original series can be bounded in terms of the series  \eqref{power:d} for $|x|<r'$.
This means that the original series converges by comparison for this $x$, which is false. Hence $r'=r$.
\medskip

We then define
$$
g(x):= \sum_{k=1}^\infty k a_k x^{k-1}, \quad |x|<r.
$$

The fact that $g = f'$ for $|x|\leq r-\varepsilon$ is a direct consequence of Theorem \ref{series:d}. Since this is true for every $\varepsilon>0$ it follows that $f$ is differentiable in the whole interval $|x|<r$ and $f'=g$ (indeed, if $|x|<r$, then we
can find $\varepsilon>0$ such that $|x|\leq r-\varepsilon$). %Continuity of $f$ follows from the existence of $f'$.
\end{proof}


\begin{remark}
Assuming that the radius of convergence of the power series is finite we have deduced uniform convergence in any closed interval contained in the interior of the interval of convergence. If $r=\infty$ then the convergence is not
necessarily uniform on the entire line, but it is uniform on any closed interval, using the same arguments as above.
\end{remark}

\begin{remark}
Note that if a power series converges for $|x|<r$, then uniform convergence may fail to hold for $|x|<r$. Indeed, the geometric series
$$
\sum_{k=0}^\infty x^k
$$
has radius of convergence $r=1$, so it converges for $|x|<1$ to the function $1/(1-x)$ and does not converges for $|x|>1$. The series does not converge at the endpoints.

The series converges uniformly on $[-\rho, \rho]$ for every $0\leq \rho<1$ but does not converge uniformly on $(-1,1)$, as shown in Example \ref{Cauchy:gs}.

\medskip


Also, the series $\sum_{k=0}^\infty \frac{x^k}{k}$ has radius of convergence $r=1$. At $x=1$ it diverges, while at $x=-1$ it converges, but not absolutely. The series does not converge uniformly on $(-1,1)$.



\end{remark}


Under the same assumptions as in the Theorem \ref{power:deriv}, we have the following result (which can be proved by induction).

\begin{theorem}\label{inf:diff}
Assume that the power series \eqref{power0} has radius of convergence $0<r<\infty$ and let
\begin{equation}\label{sum:pseries}
f(x):= \sum_{k=0}^\infty a_k x^k, \quad x\in (-r,r).
\end{equation}
Then $f$ is infinitely many times differentiable in $(-r,r)$ and for every $m\in \N$ and every $x\in (-r,r)$ we have
\begin{equation}\label{Taylor}
f^{(m)}(x) = \sum_{k=m}^\infty k(k-1)\dots(k-m+1)a_kx^{k-m}.
\end{equation}
(Here, $f^{(m)}$ is the $m$-th derivative of $f$.)
\end{theorem}

\begin{proof}
We only need to show that the series
$$
\sum_{k=m}^\infty k(k-1)\dots(k-m+1)a_kx^{k-m}
$$
has radius of convergence $r$ for every $m$. Once this is proved, we simply apply Theorem  \ref{power:deriv} to deduce that the series equals the
$m$-th derivative of $f$.

To compute the radius of convergence, note that
\begin{align*}
\limsup_{k\to \infty} \left( k(k-1)\dots(k-m+1)|a_k|\right)^{\frac{1}k}  &= \lim_{k\to \infty}  k^{\frac{1}k} \, \lim_{k\to \infty}  (k-1)^{\frac{1}k} \dots \lim_{k\to \infty}  (k-m+1)^{\frac{1}k} \limsup_{k\to \infty} (|a_k|)^{\frac1k} \\
&= \limsup_{k\to \infty} (|a_k|)^{\frac1k}  = \frac1r,
\end{align*}
which completes the proof.
\end{proof}

From Theorem \ref{inf:diff} we deduce that the sum of the series $f$ in \eqref{sum:pseries} is a well-defined and infinitely differentiable function in the interval of convergence of the series. Furthermore, we can get an expression for the coefficients $a_k$ in terms of $f$: The coefficients are simply the Taylor coefficients of $f$.


\begin{theorem}\label{thm:Taylor}
Assume that the power series \eqref{power0} has radius of convergence $0<r<\infty$ and let
\begin{equation*}
f(x):= \sum_{k=0}^\infty a_k x^k, \quad x\in (-r,r).
\end{equation*}
Then
$$
a_k = \frac{f^{(k)}(0)}{k!}.
$$
\end{theorem}
\begin{proof}
By setting $x=0$ we have that
$$
f^{(m)}(0) = m! a_m,
$$
namely if a power series converges to a function $f$, then $f$ is infinitely many times differentiable in its interval of convergence and the power series coincides with the Taylor series of $f$.
\end{proof}

\begin{remark}[Taylor series]
It is however important to stress that a function $f$ may have derivatives of all orders, and at the same time the series $\sum_{k=0}^\infty a_k x^k$, with $f^{(k)}(0) = k! a_k$, may not converge to $f$ for
$x\neq 0$.

The typical example is the function $f:\R\to \R$ defined as
$$
f(x) = \begin{cases}
e^{-1/x^2} \quad &\textrm{if } \, x\neq 0,\\
0 \quad &\textrm{if } \, x= 0.
\end{cases}
$$
The function $f$ has derivatives of all orders at $x=0$, and $f^{(m)}(0)=0$ for every $m\in \N$.
\end{remark}

\medskip

An immediate consequence of Theorem \ref{thm:Taylor} is the following Proposition:

\begin{proposition}
Assume that the power series
$$
\sum_{k=0}^\infty a_k (x - x_0)^k \quad \textrm{ and } \sum_{k=0}^\infty b_k (x - x_0)^k
$$
have nonzero (possibly different) radii of convergence and assume that the series are equal on some neighbourhood of $x_0$. Then $a_k=b_k$ for every $k\in \N$.
\end{proposition}

\begin{proof}
Let $(x_0-\delta, x_0 +\delta)$ be the neighbourhood of $x_0$ (open interval containing $x_0$) where the two series are equal. Note that $(x_0-\delta, x_0+\delta)$
is contained in the (possibly different) intervals of convergence of the two series. We can then define $g: (x_0-\delta, x_0+\delta) \to \R$ as
$$
g(x):= \sum_{k=0}^\infty a_k (x - x_0)^k = \sum_{k=0}^\infty b_k (x - x_0)^k, \quad x\in (x_0-\delta, x_0+\delta).
$$
By Theorem \ref{thm:Taylor} applied to $g$ we have that $g$ is infinitely many times differentiable in $(x_0-\delta, x_0+\delta)$ and
$$
a_k = \frac{g^{(k)}(x_0)}{k!} = b_k,
$$
hence $a_k = b_k$ for every $k\in \N$.
\end{proof}

\bigskip

Power series can be also integrated term by term, as a direct consequence of Corollary \ref{series:i}:

\begin{corollary}
Suppose that the power series
$$
\sum_{k=0}^\infty a_k x^k
$$
converges for $|x|<r$. Then, for every $-r<c<d<r$, we have
$$
\int_c^d \sum_{k=0}^\infty a_k x^k dx = \sum_{k=0}^\infty a_k \int_c^d x^k dx = \sum_{k=0}^\infty a_k \frac{d^{k+1}-c^{k+1}}{k+1}.
$$
\end{corollary}


\subsection{Complex sequences and series}
We can extend to complex numbers all the definitions and theorems in $\R$ that only rely on the `metric structure' of $\R$ (namely on the fact that we can compute
distances among elements in $\R$, with the absolute value), since we have defined a modulus on complex numbers that mimics the absolute value in $\R$.
\smallskip

We will write $(\C, | \cdot|)$ to denote the set of complex numbers
$$\C:= \{z = x+i y: x, y\in \R\},$$
with the modulus defined as
$$|z|:= \sqrt{z\bar z} =\sqrt{x^2+y^2}.$$
Note however that we cannot extend to $\C$ the properties studied in $\R$ that rely on the `ordering' of $\R$, since there is no ordering in $\C$.

\subsubsection{Convergence in $\C$.}
We can immediately extend to $\C$ the concepts of sequences and convergence of sequences and series.

\noindent
If $(z_k)_{k\in \N}$ is a sequence in $\C$ and $z_0\in \C$, we say that $z_k\to z_0$ as $k\to \infty$ if
$$
\forall\, \varepsilon>0 \quad \exists\, N\in \N: \, k\geq N \Rightarrow |z_k-z_0|<\varepsilon.
$$
(Note that we only replaced $\R$ with $\C$ and the absolute value in $\R$ with the modulus in $\C$.)

\smallskip

We can also express convergence in $\C$ in terms of convergence in $\R$: Indeed, since a sequence $(z_k)_{k\in \N}$ in $\C$ corresponds to two real sequences $(x_k)_{k\in \N}$ and $(y_k)_{k\in \N}$ of its real and imaginary parts, we have that $z_k$ converges if and only if the sequences $(x_k)_k$ and $(y_k)_k$ converge.

\medskip

\noindent
Concerning complex series, it is clear that $\sum_k z_k$ converges if and only if the real series $\sum_k x_k$ and $\sum_k y_k$ converge. Similarly, if $\sum_k z_k$ converges, then $z_k \to 0$, as in the real case. %{\color{red} Uniform!!}

Moreover, if the real series $\sum_k |z_k|$ converges (namely, if $\sum_k z_k$ converges absolutely), then the series $\sum_k z_k$ converges as well. In addition, if a complex series converges absolutely, then every rearrangement of the series converges, and they all converge to the same sum.


\begin{example}[Complex geometric series]
The complex geometric series
$$
\sum_{k=0}^\infty z^k, \quad z\in \C
$$
converges if $|z|<1$ and does not converge if $|z|\geq 1$.

This is clear, since for $|z|<1$ we have the bound (in terms of a real series)
$$
\sum_{k=0}^\infty |z^k| \leq \sum_{k=0}^\infty |z|^k,
$$
which ensures convergence by the comparison principle. On the other hand, if $|z|\geq 1$, then $|z^k|\geq 1$, and hence does not converge to zero, which is a necessary condition for the convergence of the series.
\end{example}

Notice that to prove convergence we generally try to bound a series which we
know very little about by a geometric series, which we fully understand. This is a
frequent technique in analysis!


\bigskip

More in general, the series
$$
\sum_{k=0}^\infty \alpha_k z^k, \quad \alpha_k, z\in \C
$$
converges at $z=0$, and if it converges at some $w\neq 0$, then it converges for every $z\in \C$, $|z|<|w|$.
We can therefore define a radius of convergence as for the case of real power series. Note that in $\C$ the set of points
where the series converges is a disc, and not an interval.

\begin{theorem}
Given the power series $\sum_{k=0}^\infty \alpha_k z^k$ with $\alpha_k, \,z\in \C$, set
$$
L:= \limsup_{k\to \infty} (|\alpha_k|)^{\frac1k}, \quad \textrm{and} \quad r=\frac1L.
$$
(If $L=0$, then $r=+\infty$; if $L=+\infty$, then $r=0$.) Then the series converges for $|z|<r$ and does not converge for $|z|>r$.
\end{theorem}


\medskip

Concerning the differentiation of complex power series, similar results as in the real case apply. Namely, within a power series' radius of convergence, a power series is differentiable, and its derivative can be obtained by differentiating the individual terms of the power series term by term and has the same radius of convergence.

\smallskip

First of all we need to define the derivative of a complex function as preview of Analysis 2B.

\begin{definition}
Let $A\subset \C$ and let $z_0\in A$ and $R>0$ be such that $B_R(z_0)=\{z\in \C: |z-z_0|<R\}\subset A$. A function $f:A\to \C$ is said to be (complex-)differentiable at $z_0$ if the limit
$$
\lim_{z\to z_0}\frac{f(z)-f(z_0)}{z-z_0}
$$
exists; in that case we set
$$
f'(z_0):= \lim_{z\to z_0}\frac{f(z)-f(z_0)}{z-z_0},
$$
and call it the derivative of $f$ at $z_0$.
\end{definition}

Now we can state the differentiability result for complex power series.

\begin{theorem}
Let $f(z)=\sum_{k=0}^\infty \alpha_k z^k$ be a power series with radius of convergence $r>0$. Then $f$ is complex-differentiable for $|z|<r$ (also called holomorphic) and its derivative is equal to the power series
$f'(z) = \sum_{k=1}^\infty k \alpha_k z^{k-1}$ obtained by differentiating the series term by term. Also, $f'$ has the same radius of convergence as $f$.
\end{theorem}

\begin{corollary}
A complex power series is infinitely (complex-)differentiable in its disc of convergence, and each of its $m$-th derivatives can be obtained by differentiating the series term-by-term $m$ times. The resulting power series has radius of convergence equal to the original power series.
\end{corollary}

\medskip

\subsubsection{Important functions in $\C$.}

As an illustration of some of the ideas in this subsection, we prove some basic properties of very important functions.

\medskip


\begin{example}[Complex exponential]
The complex power series
\begin{equation}\label{Tay:exp}
\sum_{k=0}^\infty \frac{z^k}{k!}
\end{equation}
is convergent for every $z\in \C$. Indeed we have the bound in terms of
$$
\left|\frac{z^k}{k!}\right| \leq \frac{|z|^k}{k!},
$$
which defines a real and convergent series. (Alternatively: Since $\alpha_k = 1/k!$, and $\lim_k|\alpha_{k+1}|/|\alpha_k|=0$, this power series has radius of convergence $r=\infty$, so it converges for all $z$.)

\smallskip

\noindent
Since on the real line the series \eqref{Tay:exp} is equal to the exponential function (by the convergence of the Taylor series of the exponential), we set
\begin{equation}\label{complex:exp}
e^z:= \sum_{k=0}^\infty \frac{z^k}{k!}
\end{equation}
to define the \textit{complex exponential function}.

\medskip
\noindent

It is obvious that $e^0 =1$, by plugging in $z= 0$ into the power series expression for $e^z$.
\smallskip

It is easy to see that $(e^z)'=e^z$. Indeed, by differentiating the power series expression for $e^z$ term by term (allowed since the derivative series has the same radius of convergence of $e^z$, hence it converges for every $z\in \C$), we find
$$
(e^z)' = 1+ \frac{2z}{2!} + \frac{3z^2}{3!}+\dots = 1+ z+\frac{z^2}{2!} + \frac{z^3}{3!}+\dots = e^z.
$$
We moreover have the following property:
\begin{equation}
e^z e^w = e^{z+w}.
\end{equation}
To prove it, we need to recall the definition of multiplication of complex series. Given $\sum_k \alpha_k$ and $\sum_k \beta_k$, the product of the series is
$$
\sum_k c_k, \quad c_k = \sum_{m=0}^k \alpha_m \beta_{k-m}, \quad k=0,1,\dots.
$$
We also recall that the product of two convergent sequences converges to the product of their sums if at least one of the two converges absolutely.

\medskip

Hence, we can see that
\begin{align*}
e^z e^w &= \left(\sum_{k=0}^\infty \frac{z^k}{k!}\right)\left(\sum_{k=0}^\infty \frac{w^k}{k!} \right) = \sum_{k=0}^\infty \sum_{m=0}^k  \frac{z^m w^{k-m}}{m!(k-m)!}\\
& = \sum_{k=0}^\infty \frac{1}{k!} \sum_{m=0}^k \binom{k}{m}z^m w^{k-m} = \sum_{k=0}^\infty \frac{1}{k!} (z+w)^k = e^{z+w}.
\end{align*}


Now note that if we write \eqref{complex:exp} for $z=iy$, with $y\in \R$, then we have
\begin{align*}
e^{iy} = \sum_{k=0}^\infty \frac{(iy)^k}{k!} = \sum_{m=0}^\infty (-1)^m\frac{y^{2m}}{(2m)!} + i \sum_{m=0}^\infty (-1)^m\frac{y^{2m+1}}{(2m+1)!}  = \cos y+ i\sin y.
\end{align*}
This is the famous Euler formula, a surprising relation between the exponential function and trigonometric functions:
$$
e^{iy} = \cos y + i \sin y.
$$
\end{example}


\begin{example}[Complex trigonometric functions]
We can define $\cos z$ and $\sin z$ by either using their respective power series, or by declaring
$$
\cos z= \frac{e^{iz} + e^{-iz}}{2}, \quad \sin z= \frac{e^{iz} - e^{-iz}}{2i},
$$
and using the power series expression of the complex exponential.
\end{example}


\begin{example}[Complex logarithm]
We have that the complex logarithm is defined as

$$
\log(1+z) = \sum_{k=1}^{\infty} \frac{(-1)^{k+1}z^{k}}{k}, \quad |z|<1
$$

Heuristically (for now, it will be clear in MA20219) we can obtain the formula from the formula of the complex geometric series.
We know that
$$
\sum_{k=0}^\infty z^k = \frac{1}{1-z}, \quad |z|\le1, z\neq -1.
$$
In particular we also have
$$
\frac{1}{1-z} = \sum_{k=0}^n z^k + \frac{z^{n+1}}{1-z}.
$$
Integrating the previous formula on the straight segment $L$ between $0$ and $z$ we have
$$
-\log(1-z) = \sum_{k=0}^n \frac{z^{k+1}}{k+1} + \int_L\frac{w^{n+1}}{1-w} dw.
$$
It is easy to show that the second term in the right-hand side is infinitesimal for $n\to \infty$, since
$$
\left|\int_L\frac{w^{n+1}}{1-w} dw\right|\leq \frac{|z|^{n+2}}{1-|z|}.
$$
In conclusion
$$
-\log(1-z) = \sum_{k=1}^{\infty} \frac{z^{k}}{k}, \quad |z|<1.
$$
\end{example}


%%%%%%%%%%%%%%%%%%%%%%%%%%%%%%%%%%%%%%%%%%
%%%%%%%%%%%%%%%%%%%%%%%%%%%%%%%%%%%%%%%%%%


\newpage


\section{Chapter 2: Metric spaces}

%{\color{red} Bisogna accorciare aperti e chiusi.}

%{\color{red} Equivalent norms?}

Many mathematical problems involve a notion of distance. In this chapter we
are going to develop a theory of `distance' from an abstract point of view. This relies
on the notion of a metric, which is an abstraction of distance between two
points of a given set.

\begin{definition}
Let $X$ be a set. A function $d : X \times X \to \R$ is called
a \emph{metric} if it satisfies the following properties:
\begin{enumerate}
\item \label{def:non-negative} $d(x,y) \ge 0$ for all $x,y \in X$,
\item \label{def:positive} $d(x,y) = 0$ if, and only if, $x = y$,
\item \label{def:symmetry} $d(x,y) = d(y,x)$ for all $x,y \in X$ (symmetry),
\item \label{def:triangle} $d(x,z) \le d(x,y) + d(y,z)$ for all $x,y,z \in X$ (triangle inequality).
\end{enumerate}
The pair $(X,d)$ is then called a \emph{metric space}.
\end{definition}

Sometimes $d$ is called `distance function' rather than `metric'. Often you can
find statements such as `$X$ is a metric space'; but this makes only sense if
it is clear what the corresponding metric is.
%Some authors require that $X$ is non-empty in the definition of a metric space.

\begin{example}[Euclidean space] \label{Eucl:d2}
Let $n \in \N$; we define $d_2 : \R^n \times \R^n \to \R$ as
\[
d_2(x,y):= \|x - y\|_2 = \left(\sum_{i = 1}^n (x_i - y_i)^2\right)^{1/2}, \quad x,y \in \R^n.
\]
If $n=1$ we use the notation $d_2(x,y) = |x-y|$. We often omit the subscript $2$ from the Euclidean distance $\|x-y\|_2$.

We prove that $d_2$ is a metric and thus $(\R^n,d_2)$ is a metric space. Axioms \eqref{def:non-negative}--\eqref{def:symmetry}
are easy to verify. For the triangle inequality, we first need the \textbf{Cauchy-Schwarz inequality} (see 2A Algebra MA20216):
\begin{equation}\label{CS}
 \left|\sum_{i=1}^n u_i v_i \right| \leq \left(\sum_{i=1}^n u_i^2\right)^{\frac12} \left(\sum_{i=1}^n v_i^2\right)^{\frac12} \quad (=\|u\|\, \|v\|).
\end{equation}
To prove \eqref{CS} it is sufficient to prove the inequality for $u_i, v_i\geq 0$ and for $u\neq 0$ and $v\neq 0$ (otherwise it is trivial).
For every $\alpha,\beta>0$, we have
$$
0\leq \sum_{i=1}^n (\alpha u_i - \beta v_i)^2 = \alpha^2 \sum_{i=1}^n u_i^2 + \beta^2 \sum_{i=1}^n v_i^2 - 2\alpha\beta \sum_{i=1}^n u_i v_i
$$
or equivalently, rearranging the terms,
$$
2\alpha\beta \sum_{i=1}^n u_i v_i  \leq \alpha^2 \sum_{i=1}^n u_i^2 + \beta^2 \sum_{i=1}^n v_i^2.
$$
The inequality \eqref{CS} follows by choosing $\alpha = \left(\sum_{i=1}^n v_i^2\right)^{\frac12}$ and $\beta = \left(\sum_{i=1}^n u_i^2\right)^{\frac12}$,
and by dividing the resulting estimate by $2\alpha\beta$ (note that $2\alpha\beta> 0$).

\medskip

Now we are ready to prove the triangle inequality. By the definition of the Euclidean norm, proving \eqref{def:triangle} means to show that
$$
\|x-z\| \leq \|x-y\| + \|y-z\| \quad \textrm{for all } \, x,y,z\in \R^n.
$$
Let $u:=x-y$ and $v:=y-z$; then $u+v=x-z$. We are going to prove (equivalently) that $\|u+v\| \leq \|u\| + \|v\|$.

To this end, note that by expanding the square and by using the Cauchy-Schwarz inequality, we have
\begin{align*}
\|u+v\|^2 = \sum_{i=1}^n  (u_i + v_i)^2 &= \sum_{i=1}^n  u_i^2 + \sum_{i=1}^n  v_i^2 + 2\sum_{i=1}^n u_i v_i \leq
\|u\|^2 + \|v\|^2 + 2 \|u\|\, \|v\| = (\|u\|+\|v\|)^2,
\end{align*}
and the claim follows by taking the square root.
\end{example}

\begin{example}
The following are also metrics on $\R^n$:
\begin{align*}
d_1(x,y):= \|x-y\|_1 = \sum_{i = 1}^n |x_i - y_i|,
\end{align*}
also called the $\ell^1$ metric (also referred to informally as the ``taxi-cab" metric, since it's the distance one would travel by taxi on a rectangular grid of streets);
\begin{align*}
d_\infty(x,y):= \|x-y\|_\infty = \max_{i \in \{1,\ldots,n\}} |x_i - y_i|,
\end{align*}
also called the $\ell^\infty$ metric, or the maximum metric.
More generally, for $1 \le p < \infty$, we have a metric
\[
d_p(x,y):= \|x-y\|_p = \left(\sum_{i = 1}^n |x_i - y_i|^p\right)^{1/p}
\]
on $\R^n$, but this is more difficult to verify (the triangle inequality does not follow from the Cauchy-Schwarz inequality, but from the Minkowski inequality).
For $p < 1$, $d_p$ does not define a metric since the triangle inequality fails! E.g., in $\R^2$, for $p=\frac12$, and $x=(1,0)$, $z=(0,-1)$, $y=(0,0)$, we have
$$
\|x-z\|_{\frac12} = \|(1,1)\|_{\frac12} = (1+1)^2 = 4; \quad \|x\|_{\frac12} + \|z\|_{\frac12} = 2.
$$
\end{example}

\begin{example}[Discrete metric] \label{discrete-metric}
Let $X \not= \emptyset$ be a set. Define
$$
d(x,y) = \begin{cases}
0 & \text{if } x=y, \\
1 & \text{if } x\neq y.
\end{cases}
$$
Then \eqref{def:non-negative}--\eqref{def:symmetry} are clearly satisfied. To
verify \eqref{def:triangle}, we just need to check four different cases, and it turns
out that $d$ is a metric.
\end{example}

\begin{example}[Railway metric]\label{railway}
Define $d : \R^n \times \R^n \to \R$ as follows: for $x,y \in \R^n$,
\[
d(x,y) = \begin{cases} \|x - y\| & \text{if $x,y,0$ are collinear}, \\ \|x\| + \|y\| & \text{otherwise}. \end{cases}
\]
E.g., for $n = 2$,
\begin{align*}
d((1,1),(2,2)) & = \sqrt{2}, \\
d((1,1),(-2,2)) & = 3\sqrt{2}.
\end{align*}
(\textit{Curiosity:} It is called the Railway metric in Britain (and in France), because all the train lines radiate from London (and Paris), located at the origin. To take a train from town $x$ to town $y$, one has to take a train from $x$ to $0$ and then take a train from $0$ to $y$, unless $x$ and $y$ are on the same line, when one can take a direct train.)

\medskip


%We want to verify that $d$ is a metric. Axioms \eqref{def:non-negative} and \eqref{def:symmetry} are obvious.
%
%\eqref{def:positive} If $x = y$, then $x,y,0$ are collinear and $d(x,y) = \|x - y\| = 0$. If $d(x,y) = 0$, then
%either
%\begin{itemize}
%\item $x$, $y$, and $0$ are collinear and $\|x - y\| = 0$, or
%\item $\|x\| = 0$ and $\|y\| = 0$.
%\end{itemize}
%In the first case, we have $x = y$. In the second case, we have $x = 0 = y$.
%
%\eqref{def:triangle} Let $x,y,z \in \R^n$. We have to distinguish a number of cases in order to prove the triangle
%inequality. If $x,y,z,0$ are all collinear, then
%\[
%d(x,z) = \|x - z\| \le \|x - y\| + \|y - z\| = d(x,y) + d(y,z).
%\]
%If $x,y,0$ are collinear and $y,z,0$ are not, then $x,z,0$ are not collinear. Hence
%\[
%d(x,z) = \|x\| + \|z\| \le \|x - y\| + \|y\| + \|z\| = d(x,y) + d(y,z).
%\]
%When the roles of $x$ and $z$ are reversed, the proof remains the same. Finally, if neither $x,y,0$ nor $y,z,0$ nor $x,z,0$ are
%collinear, then
%\[
%d(x,z) \le \|x\| + \|z\| \le \|x\| + 2\|y\| + \|z\| = d(x,y) + d(y,z).
%\]
\end{example}



\begin{example}\label{continuous_functions}
Consider the space $C^0([0,1])$ of all continuous functions from $[0,1]$ to $\R$ (sometimes it is denoted with $C([0,1])$).
Define
\[
d_\infty(f,g) = \max_{t \in [0,1]} |f(t) - g(t)|, \quad f,g \in C^0([0,1]).
\]
Then $(C^0([0,1]),d_\infty)$ is a metric space.

\quad

Another possible metric on $C^0([0,1])$ can be defined as
$$
d^\ast(f,g) = \int_0^1|f(t)-g(t)|dt.
$$

\end{example}


In many of the examples above the distance can be defined in terms of a norm. This is the case for the Euclidean space, but not only.

\subsection{Normed vector spaces}

We start by recalling the definition of vector spaces.

\begin{definition}
A real vector space is a nonempty set $X$ on which two operations are defined, addition in $X$
$$
+ : X\times X \to X
$$
and multiplication by an element in $\R$
$$
\cdot : \R\times X \to X
$$
satisfying the following axioms:
\begin{itemize}
\item The addition is associative and commutative, namely
$$
x+y = y+x, \quad (x+y)+z = x+y+z \quad \forall \, x,y,z \in X;
$$
\item There exists an element $0\in X$ such that $x+0=x$ for every $x\in X$;
\item $0\cdot x = 0$ and $1\cdot x= x$ for every $x\in X$;
\item Addition and multiplication are distributive, namely
$$
(\lambda+\mu)x = \lambda x+ \mu x, \quad \lambda(x+y) = \lambda x+ \lambda y \quad \forall \, x,y \in X, \lambda,\mu\in \R;
$$
\item For every $\lambda, \mu \in \R$ and every $x\in X$:
$$
\lambda(\mu x) = (\lambda \mu) x.
$$
\end{itemize}

The vector space is called complex if the multiplication is defined with elements of $\C$.
\end{definition}


On vector spaces one can define a \textit{norm}, which is essentially a way to measure to magnitude of the elements in the vector space.


\begin{definition}[Normed space]
Let $X$ be a (real) vector space. An application $\|\cdot\|: X\to \R$ is called a norm on $X$ if it satisfies the following axioms:
\begin{itemize}
\item[(N1)] $\|x\|\geq0$ and $\|x\|=0$ if and only $x=0$;
\item[(N2)] $\|\lambda x\| = |\lambda| \|x\|$ for every $x\in X$ and every $\lambda\in \R$;
\item[(N3)] $\|x+y\|\leq \|x\| + \|y\|$.
\end{itemize}
The pair $(X,\|\cdot\|)$ is called a normed space.
\end{definition}

\begin{remark}
The properties in the definition of the norm are natural ones to require of a length: The length of $x$ is $0$ if and only if $x$ is the $0$-vector; multiplying a vector by $\lambda$ multiplies its length by $|\lambda|$; and the length of the side $x + y$ (in a triangle) is less than or equal to the sum of the lengths of the sides $x$, $y$. Because of this last interpretation, property (N3) is referred to as the triangle inequality (as for the distance).
\end{remark}

\begin{example}[Norms in $\R^n$]
$(\R,|\cdot|)$ is a normed space, where $|\cdot|$ denotes the absolute value. Also $(\R^n,\|\cdot\|)$ is a normed space, where $\|\cdot\|$ denotes
$$
\|x\|:= \sqrt{\langle x,x\rangle} = \sqrt{\sum_{i=1}^n x_i^2}.
$$
(We will often write $\|\cdot\|_2$ for the Euclidean norm.)

\medskip

The set $\R^n$ with any of the norms defined for $x = (x_1,\dots, x_n)$ by
$$
\|x\|_1 := \sum_{i=1}^n|x_i|, \quad \|x\|_{\infty} := \max_{i=1,\dots,n}\{ |x_i|\}
$$
is an $n$-dimensional normed vector space. The corresponding metrics are the ``taxi-cab'' metric and the maximum metric, respectively.

\end{example}

\begin{example}[Norms on spaces of functions]\label{continuous_functions-N}
The distance in Example \ref{continuous_functions} can be defined via the norm given, for $f \in C^0([0,1])$, by
$$
\|f\|:= \max_{t \in [0,1]} |f(t)|.
$$
This norm is usually denoted with $\|f\|_{\infty}$ and referred to as the infinity-norm or sup-norm. (We will see a connection between this norm and uniform convergence.)
\medskip

\noindent
In $C^0([0,1])$ one can also define $L^p$-norms, for $p\geq 1$, as
$$
\|f\|_p:= \left(\int_0^1|f(t)|^p dt\right)^{\frac1p},
$$
for $f\in C^0([0,1])$. Then $(C^0([0,1]), \|\cdot\|_p)$ is a normed space.

\medskip

Another example of a normed space is the space $C^1([0,1])$ of continuously differentiable functions in $[0,1]$\footnote{Usually differentiability is defined in an open interval. Here differentiable in the closed interval $[0,1]$ means that it is differentiable in $(0,1)$ and the derivative can be extended up to $[0,1]$ continuously.}, with norm defined as
$$
\|f\|_{C^1}:= \max_{t \in [0,1]} |f(t)| + \max_{t \in [0,1]} |f'(t)| = \|f\|_{\infty}+\|f'\|_{\infty}.
$$

\end{example}


\begin{remark}
A normed space $(X,\|\cdot\|)$ is a metric space, with the distance
$$
d(x,y):= \|x-y\|, \quad x,y\in X.
$$
(The triangle inequality comes from the axiom (N3) of the norm.)

\medskip

The vice versa is not true! A metric associated with a norm is special: It has the additional properties that for all $x, y, z \in X$ and $\lambda\in \R$
$$d(x + z,y + z) = d(x,y), \quad d(\lambda x,\lambda y) = |\lambda| \, d(x,y),$$
which are called translation invariance and homogeneity, respectively. These properties do not even make sense in a general metric space since we cannot add points or multiply them by scalars. If $X$ is a normed vector space, we always use the metric associated with its norm, unless stated specifically otherwise.
\end{remark}



\begin{example}[Metrics not associated to a norm]
Consider the discrete metric in Example \ref{discrete-metric} and take $X=\R$ (or any vector space on $\R$ containing at least two distinct elements). Then, for any $\lambda\in \R$, with $\lambda\neq 0$ and $|\lambda|\neq 1$, and any $x,y \in \R$ with $x\neq y$, we have that
$$
d(\lambda x, \lambda y) =1,
$$
while
$$
|\lambda| d(x,y) = |\lambda| \neq 1.
$$
\medskip

Also the railway metric in Example \ref{railway} is not derived from a norm (prove it!).

\end{example}

\subsection{Open and Closed Sets}

We now extend to metric spaces the definitions of open and closed interval that we learnt in $\R$.



\begin{definition} \label{def:ball}
Let $(X,d)$ be a metric space. Let $x_0 \in X$ and $r > 0$. Then
\[
B_r(x_0) = \{x \in X: d(x_0,x) < r\}
\]
is called the \emph{open ball} of centre $x_0$ and radius $r$, whereas
\[
\overline{B}_r(x_0) = \{x \in X: d(x_0,x) \leq r\}
\]
is called the \emph{closed ball} of centre $x_0$ and radius $r$.
\end{definition}

\begin{example}
In $(\R^n, d_2)$, namely in the Euclidean space, balls are the usual shapes. (Note that, for $n=1$, an open (resp. closed) ball is an open (resp. closed) interval centred at $x_0$ and with length $2r$.)
\end{example}

\begin{example}
Consider $\R^2$ with the metric
\[
d_1(x,y) = |x_1 - y_1| + |x_2 - y_2|.
\]
Then $B_1(0)$ has the shape shown in Figure \ref{fig:ball1}.
\begin{figure}[ht]
\begin{center}
\includegraphics[width=0.3\textwidth]{ball1.pdf}
\caption{The unit ball $B_1(0)$ in the metric $d_1$ (or in the norm $\|\cdot\|_{1}$)}
\label{fig:ball1}
\end{center}
\end{figure}
\end{example}

\begin{example}
Consider $\R^2$ with the metric
\[
d_\infty(x,y) = \max\{|x_1 - y_1|,|x_2 - y_2|\}.
\]
Then $B_1(0)$ has the shape shown in Figure \ref{fig:ballinfty}.
\begin{figure}[ht]
\begin{center}
\includegraphics[width=0.3\textwidth]{ballinfty.pdf}
\caption{The unit ball $B_1(0)$ in the metric $d_\infty$ (or in the norm $\|\cdot\|_{\infty}$)}
\label{fig:ballinfty}
\end{center}
\end{figure}
\end{example}

\begin{example}
Consider $C^0([0,1])$ with the usual metric
\[
d_\infty(f,g) = \max_{t \in [0,1]} |f(t) - g(t)| \quad (=\|f-g\|_\infty)
\]
defined in Example \ref{continuous_functions}.

Let $f_0 \in C^0([0,1])$. A graphical representation of the ball $B_r(f_0)$ is given in Figure \ref{fig:ballC0}.
\begin{figure}[ht]
\begin{center}
\includegraphics[width=0.4\textwidth]{ballC0.pdf}
\caption{The thick solid curve represents the graph of a function $f_0 \in C^0([0,1])$. The open ball $B_r(f_0)$ is given by all functions with graphs contained in the grey region (not including the upper and lower boundary).}
\label{fig:ballC0}
\end{center}
\end{figure}
\end{example}


We can then define what it means for a set to be bounded.



\begin{definition}[Bounded sets]
Let $(X,d)$ be a metric space. A set $A\subseteq X$ is said to be \emph{bounded} if there exist $x_0\in X$ and $r>0$ such that
$$
A\subseteq B_r(x_0).
$$

We say that $(X,d)$ is a \emph{bounded metric space} if $X$ is bounded.
\end{definition}


\begin{remark}
Alternatively, one can say that $A\subseteq X$ is bounded if
\[
\textrm{diam} (A) := \sup_{x ,y \in A} d(x,y) <\infty,
\]
and $\textrm{diam} (A)$ is called the diameter of $A$.

\end{remark}

\begin{example}
The Euclidean space $\R^n$ is unbounded, but the open unit ball
$\{x \in \R^n: \|x\| < 1\}$
in $\R^n$ is bounded (by definition) with diameter $2$.
\end{example}

\begin{example}
On $\R$, define
\[
d(x,y) = |\arctan x - \arctan y|.
\]
Then $(\R,d)$ is a metric space. As $|\arctan x| \le \frac{\pi}{2}$ for every
$x \in \R$, it is bounded. In fact, we have $\textrm{diam} (\R) = \pi$ in this metric.
\end{example}



%\begin{definition}
%Let $(X,d)$ be a metric space and $x_0 \in X$. A set $N \subseteq X$ is a
%\emph{neighbourhood} of $x_0$ if there exists a radius $r > 0$ such that $B_r(x_0) \subseteq N$.
%\end{definition}

\begin{definition}[Open sets]
Let $(X,d)$ be a metric space. A set $U \subseteq X$ is \emph{open} if
\[
\forall \, x \in U \quad \exists \, r > 0 : B_r(x) \subseteq U.
\]
\end{definition}

\begin{lemma} \label{lemma:balls}
Let $(X,d)$ be a metric space. Let $x_0 \in X$ and $r > 0$. Then $B_r(x_0)$ is open.
\end{lemma}

\begin{proof}
Let $x \in B_r(x_0)$. Set $s = r - d(x,x_0)$. Then $s > 0$. Moreover, we have $B_s(x) \subseteq B_r(x_0)$. It follows that $B_r(x_0)$ is open.
\end{proof}

We now prove some properties for open sets.

\begin{theorem} \label{thm:open}
Let $(X,d)$ be a metric space. Then
\begin{enumerate}
\item \label{item1.2.1.i} $\emptyset$ and $X$ are open;
\item \label{item1.2.1.ii} if $\{U_\lambda: \lambda \in \Lambda\}$ is any collection of open sets, then the union
\[
\bigcup_{\lambda \in \Lambda} U_\lambda
\]
is open;
\item \label{item1.2.1.iii} if $\{U_1, \ldots, U_N\}$ is a finite collection of open sets, then the intersection
\[
\bigcap_{k = 1}^N U_k
\]
is open.
\end{enumerate}
\end{theorem}

\begin{proof}
\eqref{item1.2.1.i} Clearly $\emptyset$ is open. Moreover, for every $x \in X$, we have
$B_1(x)=\{y \in X: d(x,y) < 1\} \subseteq X$, so $X$ is open.
\medskip

\eqref{item1.2.1.ii} Let $x \in \bigcup_{\lambda \in \Lambda} U_\lambda$. Then there exists an $\lambda^\ast \in \Lambda$ such
that $x \in U_{\lambda^\ast}$. As $U_{\lambda^\ast}$ is open, there exists an $r > 0$ such that $B_r(x) \subseteq U_{\lambda^\ast}$, and
then
\[
B_r(x) \subseteq \bigcup_{\lambda \in \Lambda} U_\lambda.
\]
\medskip

\eqref{item1.2.1.iii} Let $x \in \bigcap_{k = 1}^N U_k$. Since every $U_k$ is open, there exist
$r_1,\ldots,r_N > 0$ with
\[
B_{r_1}(x) \subseteq U_1, \ldots, B_{r_N}(x) \subseteq U_N.
\]
Define $r = \min\{r_1,\ldots,r_N\}$. Then
\[
B_r(x) \subseteq B_{r_k}(x) \subseteq U_k, \quad k = 1,\ldots,N,
\]
and thus $B_r(x) \subseteq \bigcap_{k = 1}^N U_k$.
\end{proof}

\begin{definition}[Interior]
Let $A$ be a subset of a metric space $(X,d)$. A point $x \in A$ is called \emph{interior point}
of $A$ if there exists an $r > 0$ such that $B_r(x) \subseteq A$.
The set of all interior points of $A$ is denoted by $A^\circ$ and called the
\emph{interior} of $A$.
\end{definition}

\begin{example}
Consider the set $[0,1] \subset \R$. Then $[0,1]^\circ = (0,1)$.
\end{example}


\begin{theorem} \label{thm:interior}
Let $A$ be a subset of a metric space $(X,d)$. Then
\begin{enumerate}
\item \label{item1.2.2.i} $A^\circ$ is open, and
\item \label{item1.2.2.ii} if $U \subseteq A$ is open, then $U \subseteq A^\circ$.
\end{enumerate}
\end{theorem}

We can summarise this theorem as follows: the interior of $A$ is the largest
open set contained in $A$.

\begin{proof}
\eqref{item1.2.2.ii} Let $U \subseteq A$ be open. If $x \in U$ is an
arbitrary point, then there exists an $r > 0$ such that $B_r(x) \subseteq U \subseteq A$.
So $x$ is an interior point, and hence $x \in A^\circ$.

\eqref{item1.2.2.i} Let $x \in A^\circ$. Then there exists an $r > 0$ such that
$B_r(x) \subseteq A$. By Lemma \ref{lemma:balls}, we know that $B_r(x)$ is open.
By \eqref{item1.2.2.ii}, we have $B_r(x) \subseteq A^\circ$. Therefore $A^\circ$ is open.
\end{proof}


\begin{remark}
Using this notation, we can reformulate the definition of open sets: a set
$U \subseteq X$ is open if, and only if, each point of $U$ is an interior point.
That is,
\[
U \mbox{ is open} \Leftrightarrow U = U^\circ.
\]
\end{remark}

%%%%%%%%%
%%%%%%%%%  Fin qui


\bigskip



\begin{definition}
Let $(X,d)$ be a metric space and $A \subseteq X$. A point $x_0 \in X$ is
called a \emph{cluster point} (or an `accumulation point', or a `limit point') of $A$, if for all
$r > 0$,
\[
(B_r(x_0) \backslash \{x_0\}) \cap A \not= \emptyset.
\]
The set of all cluster points of $A$ is denoted by $A'$.

\medskip

A point $x_0 \in A$ is called an \emph{isolated point} of $A$, if $x\in A$ and there exists $r > 0$ such that
\[
(B_r(x_0) \backslash \{x_0\}) \cap A  = \emptyset.
\]

\end{definition}

In other words, the point $x_0$ is a cluster point of $A$ if it has points
of $A$ other than itself arbitrarily close. It is an isolated point if it is in $A$ but is not a cluster point of $A$.

Note that a cluster point may be contained in $A$ or not, while an isolated point is contained in $A$.


\begin{example}
In $(\R, |\cdot|)$, consider the set $A = \left\{\frac{1}{k}: k \in \N\right\}$. Then $0 \in A'$ (even if $0\notin A$), whereas $1 \not\in A'$ (even if $1\in A$).
\end{example}

\begin{definition}[Closed sets]
A subset of a metric space is called \emph{closed} if it contains
all of its cluster points.
\end{definition}

\begin{example}
The set $A$ from the previous example is not closed, as $0 \not\in A$.
On the other hand, the set $F = \left\{\frac{1}{k}: k \in \N\right\} \cup \{0\}$ is closed in $\R$.
There is only one cluster point, namely $0$. All the elements of $A$ are isolated points.
\end{example}

\begin{theorem} \label{thm:complement}
Let $(X,d)$ be a metric space. A set $F \subseteq X$ is closed if, and only if, its complement
$X \backslash F$ is open.
\end{theorem}

\begin{proof}
Suppose that $F$ is closed. Let $x \in X \backslash F$. Then $x$ is not a cluster point of $F$.
That is, there exists an $r > 0$ such that
\[
(B_r(x) \backslash \{x\}) \cap F = \emptyset.
\]
Since $x \not\in F$, this means $B_r(x) \subseteq X \backslash F$. It follows that
$X \backslash F$ is open.

Now suppose that $F$ is not closed. Then there exists a cluster point $x$ of $F$, with
$x \in X \backslash F$. Then for any $r > 0$, we have $B_r(x) \cap F \not= \emptyset$,
so $B_r(x) \not\subseteq X \backslash F$. This means that $x$ is not an interior point
of $X \backslash F$, and $X \backslash F$ is not open.
\end{proof}

\begin{example}
Let $a < b$. The interval $[a,b]$ is closed in $\R$ (because $(-\infty,a) \cup (b,\infty)$ is open),
but $(a,b]$, $[a,b)$, and $(a,b)$ are not closed.
\end{example}

\begin{example}
The closed ball $\overline{B}_r(x)$ is closed in any metric space $(X,d)$ for any $x \in X$ and any $r > 0$.
\end{example}

\begin{example}
In any metric space $(X,d)$, the sets $\emptyset$ and $X$ are closed.
\end{example}

\begin{definition}[Closure]
Let $A$ be a subset of a metric space $(X,d)$. The \emph{closure} of $A$
is the set $\overline{A} = A \cup A'$.
\end{definition}

\begin{theorem} \label{thm:closure}
Let $A$ be a subset of a metric space $(X,d)$. Then
\begin{enumerate}
\item \label{item1.3.2.i} $\overline{A}$ is closed, and
\item \label{item1.3.2.ii} if $F \supseteq A$ is closed, then $F \supseteq \overline{A}$.
\end{enumerate}
\end{theorem}

In other words, the closure $\overline{A}$ is the smallest closed set containing $A$, and
$$
A \mbox{ is closed} \Leftrightarrow A = \overline A.
$$



\begin{proof}
\eqref{item1.3.2.ii} Suppose that $F \supseteq A$ is closed. Let
$x \in \overline{A} = A \cup A'$. If $x \in A$, then it is
clear that $x \in F$ as well. If $x \in A'$, then for every
$r > 0$, we have
\[
\emptyset \not= (B_r(x) \backslash \{x\}) \cap A \subseteq (B_r(x) \backslash \{x\}) \cap F.
\]
Hence $x$ is a cluster point of $F$, too. Since $F$ is closed,
this means that $x \in F$ and $\overline{A} \subseteq F$.

\eqref{item1.3.2.i} We want to show that $X \backslash \overline{A}$ is open, and the
claim then follows from Theorem \ref{thm:complement}.

Let $x \in X \backslash \overline{A}$. As $x$ is not a cluster point of $A$,
there exists a radius $r > 0$ such that $B_r(x) \subseteq X \backslash A$.
We know that $B_r(x)$ is open (Lemma \ref{lemma:balls}); therefore  $X \backslash B_r(x)$
is closed. By \eqref{item1.3.2.ii}, we have $\overline{A} \subseteq X \backslash B_r(x)$,
i.e., $B_r(x) \subseteq X \backslash \overline{A}$. Therefore $X \backslash \overline{A}$
is open, as required.
\end{proof}



\begin{theorem}
Let $(X,d)$ be a metric space. Then
\begin{itemize}
\item $\emptyset$ and $X$ are closed;
\item if $\{F_\lambda: \lambda \in \Lambda\}$ is any collection
of closed sets, then
\[
\bigcap_{\lambda \in \Lambda} F_\lambda
\]
is closed; and
\item if $F_1, \ldots, F_N$ is a finite collection of
closed sets, then
\[
\bigcup_{k = 1}^N F_k
\]
is closed.
\end{itemize}
\end{theorem}

\begin{proof}
Again this is easily reduced to a theorem about open sets (Theorem \ref{thm:open}), using Theorem \ref{thm:complement}, since
$$
X\setminus \bigcap_{\lambda \in \Lambda} F_\lambda = \bigcup_{\lambda \in \Lambda} (X\setminus F_\lambda)
$$
and
$$
X\setminus \bigcup_{k = 1}^N F_k = \bigcap_{k = 1}^N (X\setminus F_k).
$$
\end{proof}

\begin{remark}
Open and closed are not opposites. There are sets
which are both open and closed (e.g., the empty set in any metric space), and there
are sets which are neither (e.g., the interval $[0,1)$ in $\R$). (Compare doors.)
\end{remark}



\begin{example}\label{C:set}
Consider $\R$ with the usual metric. Let $I_0 = [0,1]$ and define
\begin{align*}
I_1 & = \textstyle [0,\frac{1}{3}] \cup [\frac{2}{3},1] \\
I_2 & = \textstyle [0,\frac{1}{9}] \cup [\frac{2}{9},\frac{1}{3}] \cup [\frac{2}{3},\frac{7}{9}] \cup [\frac{8}{9},1], \\
I_2 & = \textstyle [0,\frac{1}{27}] \cup [\frac{2}{27},\frac{1}{9}] \cup \ldots \cup  [\frac{8}{9},\frac{25}{27}] \cup [\frac{26}{27},1], \\
& \hspace{6pt} \vdots
\end{align*}
(In each step, remove the middle third of every interval remaining, but keep the end points; see Figure \ref{fig:Cantor}.)
Let
\[
C = \bigcap_{k = 0}^\infty I_k.
\]
This is called the Cantor set. \index{Cantor set} Every $I_k$ is a finite union of closed sets, and therefore closed.
Hence $C$ is an (infinite) intersection of closed sets, which means that $C$ is closed.
\begin{figure}[ht]
\begin{center}
\includegraphics[width=0.6\textwidth]{Cantor.pdf}
\end{center}
\caption{The sets $I_0,\ldots,I_5$ in the construction of the Cantor set}
\label{fig:Cantor}
\end{figure}
\end{example}


\begin{definition}[Boundary of a set]
Let $(X,d)$ be a metric space. The \emph{boundary} $\partial A$ of a set $A \subseteq X$ is defined as
$$
\partial A:= \overline A\cap \overline{(X\setminus A)}.
$$
\end{definition}

Namely the boundary of $A$ is the set of points $x\in X$ (not necessarily in $A$!!) such that every ball $B_r(x)$ contains points of $A$ and of $X\setminus A$.

\begin{example}
In $\R$, $\partial [a,b] = \partial (a,b) = \{a,b\}$, $\partial \mathbb{Q} = \R$. Moreover, $\partial C= C$, where $C$ is the Cantor set in Example \ref{C:set}.
(Note that $C^\circ = \emptyset$.)

\end{example}


\begin{theorem}
Let $(X,d)$ be a metric space, $A\subset X$, and let $x\in X$. Then
$$
x\in \partial A \quad \Leftrightarrow \quad \forall \, r>0 \,\, \textrm{we have } \,\, B_r(x)\cap A\neq \emptyset \quad \textrm{and} \quad B_r(x)\cap (X\setminus A)\neq \emptyset.
$$
\end{theorem}

\begin{proof}
``$\Rightarrow$": Let $x\in \partial A$. Either $x\in A$ or $x\in X\setminus A$. We assume that $x\in A$ and a similar argument works in the other case.

Since $x\in A$, then $x\notin X\setminus A$; but
$$
x\in \partial A\subset \overline{X\setminus A} = (X\setminus A)\cup (X\setminus A)',
$$
hence $x\in (X\setminus A)'$. By definition, for every $r>0$, $(B_r(x)\setminus \{x\})\cap (X\setminus A)\neq \emptyset$; on the other hand, since
$x\in A$, $B_r(x)\cap A\neq \emptyset$.

In conclusion, we have the claim.
\medskip

``$\Leftarrow$": Let $x\in X$. Either $x\in A$ or $x\in X\setminus A$. We assume that $x\in A$ and a similar argument works in the other case.

Since $x\in A\subseteq \overline A$, clearly $x\in \overline A$.

On the other hand, $x\notin X\setminus A$, but for every $r>0$, $B_r(x) \cap (X\setminus A) \neq \emptyset$, which equivalently means that
for every $r>0$, $(B_r(x)\setminus \{x\})\cap (X\setminus A)\neq \emptyset$, since $x\notin X\setminus A$. By definition, $x\in (X\setminus A)'$,
and since $(X\setminus A)'\subseteq \overline{X\setminus A}$, we have the claim.

\end{proof}

\begin{definition}[Dense sets]
Let $(X,d)$ be a metric space. A set $A \subseteq X$ is called \emph{dense} \index{dense}
if $\overline{A} = X$.
\end{definition}

\begin{example}
The set $\Q$ is dense in $\R$.
\end{example}



\begin{remark} A further generalisation beyond metric spaces are topological spaces, where a collection of open subsets -- satisfying Theorem
\ref{thm:open} -- is defining the topology.
\end{remark}


%
%%%%%%%%%%%%%%%%%%%   FIN QUI  %%%%%%%%%%%%
%


\bigskip




\subsection{Sequences and Convergence}

We extend the concept of convergence of sequences to metric spaces.

\begin{definition}
Let $(X,d)$ be a metric space, let $(x_k)_{k \in \N}$ be a sequence in $X$ and let $x_0\in X$. The point
$x_0 \in X$ is said to be a \emph{limit} of the sequence if $d(x_k,x_0) \to 0$
in $\R$ as $k \to \infty$, namely if
$$
\forall \, \varepsilon>0 \quad \exists \,N\in \N: k\geq N \Rightarrow d(x_k,x_0)<\varepsilon.
$$
If so, we say that the sequence \emph{converges} to
$x_0$ and we write $x_0 = \lim_{k \to \infty} x_k$ or $x_k \to x_0$ (or $x_k \stackrel{d}{\to} x_0$).
\end{definition}




\begin{example}
In $(\R, |\cdot|)$, consider the sequence $(x_k)_{k \in \N}$
with $x_k = \frac{1}{k}$. Then $d(x_k,0) = \frac{1}{k}$, so $x_k$
converges to $0$.
\end{example}

\begin{example}
In $(X,|\cdot|)$, with $X = (0,1]$, let $x_k = \frac{1}{k}$. Then there is no point $x_0$ in $X$ such that
$d(x_k,x_0) \to 0$ as $k \to \infty$. The sequence $(x_k)_{k \in \N}$
is therefore not convergent in $X$.
\end{example}

\begin{example}
Let $X \not= \emptyset$ be an arbitrary set with the discrete metric
\[
d(x,y) = \begin{cases} 0 & \text{if $x = y$,} \\ 1 & \text{if $x \not= y$.} \end{cases}
\]
Then $x_k \to x_0$ if, and only if, there exists an $N \in \N$ such that $x_k = x_0$ for all
$k \ge N$.
\end{example}

\begin{example}
Consider the space $C^0([0,1])$ with the usual metric
\[
d_\infty(f,g):= \|f-g\|_{\infty} = \max_{0 \le t \le 1} |f(t) - g(t)|
\]
defined in Example \ref{continuous_functions}. (Recall that this is a normed space, see Example \ref{continuous_functions-N}.)

A sequence $(f_k)_{k \in \N}$ converges to $f_0$ with respect to the metric $d_\infty$ if, and only if,
$$
\max_{0 \le t \le 1} |f_k(t) - f_0(t)| \to 0 \quad \textrm{as } k\to \infty,
$$
namely if
\[
\forall \varepsilon > 0 \quad \exists \,N \in \N \quad\textrm{such that } \, \forall \, k \ge N \quad \textrm{and } \, \forall \, t \in [0,1] : |f_k(t) - f_0(t)| < \varepsilon.
\]
But this is exactly the definition of uniform convergence of $f_k$ to $f$ in Definition \ref{pu-conv}! For this reason the metric $d_\infty$ is often referred to as the \textit{uniform convergence metric}, since convergence in $d_\infty$ is equivalent to uniform convergence.
\end{example}

\begin{example}
Consider the space $C^0([0,1])$ with the metric
\[
d_2(f,g) = \|f-g\|_{2} = \left(\int_0^1 |f(t) - g(t)|^2 dt\right)^\frac12.
\]
A sequence $(f_k)_{k \in \N}$ converges to $f_0$ with respect to the metric $d_2$ (commonly said in the $L^2$-norm) if, and only if,
\[
\left(\int_0^1 |f_k(t) - f_0(t)|^2 dt\right)^\frac12 \to 0 \quad \textrm{as } \, k\to \infty.
\]
\smallskip

Note that this norm can be defined on all functions $f: [0,1] \to \R$ with $f^2$ Riemann-integrable (continuity is not needed).

\end{example}


\begin{theorem} \label{thm:uniqueness}
Let  $(X,d)$ be a metric space, let $x_0\in X$, and let $(x_k)_{k \in \N}$ be a sequence converging to $x_0$.
\begin{enumerate}
\item \label{item2.1.1.i} If $x_k \to y_0$ in $(X,d)$, then $x_0 = y_0$ (i.e., limits are unique).
\item \label{item2.1.1.iii} For every subsequence $(x_{k_j})_{j \in \N}$, the convergence
$x_{k_j} \to x_0$ holds as well.
\end{enumerate}
\end{theorem}

\begin{proof}
\eqref{item2.1.1.i} We have, by the triangle inequality,
\[
d(x_0,y_0) \le d(x_0,x_k) + d(y_0,x_k) \to 0
\]
as $k \to \infty$. Hence $d(x_0,y_0) = 0$, which means that $x_0 = y_0$.

\eqref{item2.1.1.iii} Recall: when we say that $(x_{k_j})_{j \in \N}$ is a subsequence of $(x_k)_{k \in \N}$,
this means that $(k_j)_{j\in \N}$ is a strictly increasing sequence of natural numbers. Let $\varepsilon > 0$.
As $x_k \to x_0$ as $k \to \infty$, there exists an $N \in \N$ such that $d(x_k,x_0) < \varepsilon$ for $k \ge N$.
Moreover, there is an $M \in \N$ such that $k_j \ge N$ for $j \ge M$. Therefore we also have $d(x_{k_j},x_0) < \varepsilon$
for $j \ge M$, which proves the convergence $x_{k_j} \to x_0$ as $j \to \infty$.
\end{proof}

\begin{theorem} \label{thm:limit_closed}
Let $(X,d)$ be a metric space and $A \subseteq X$.
\begin{enumerate}
\item \label{item2.1.2.i} A point $x_0 \in X$ belongs to $\overline{A}$ if,
and only if, there exists a sequence in $A$ converging (in $d$) to $x_0$.
\item \label{item2.1.2.ii} The set $A$ is closed if, and only if, for every sequence
in $A$ that converges in $X$, the limit belongs to $A$.
\item \label{item2.1.2.iii} The set $A$ is open if, and only if, for
every sequence $(x_k)_{k \in \N}$  converging to a limit in $A$,
there exists an $N \in \N$ such that $x_k \in A$ for all $k \ge N$.
\end{enumerate}
\end{theorem}

\begin{proof}
\eqref{item2.1.2.i} Suppose that $x_0 \in \overline{A}= A \cup A'$. If $x_0 \in A$, then consider the sequence $(x_k)_{k \in \N}$ with
$x_k = x_0$ for every $k \in \N$. This is a sequence in $A$ converging to $x_0$. If $x_0 \in A'\setminus A$,
then for every $k \in \N$, we have
\[
(B_{1/k}(x_0)\setminus\{x_0\}) \cap A \not= \emptyset \quad \Leftrightarrow \quad B_{1/k}(x_0) \cap A \not= \emptyset.
\]
Now choose $x_k \in B_{1/k}(x_0) \cap A$. Then $d(x_k,x_0) \le \frac{1}{k} \to 0$ as $k \to \infty$.
Hence the sequence $(x_k)_{k \in \N}\subseteq A$ converges to $x_0$.

Conversely, suppose that $x_0 \in X$ and that there exists a sequence $(x_k)_{k \in \N}$ in $A$
with $x_k \to x_0$ as $k \to \infty$. If $x_0 \in A$, there is nothing to prove, since $A\subseteq \overline{A}$. Otherwise,
fix $r > 0$. Then there exists an $N \in \N$ such that $d(x_k,x_0) < r$ for $k \ge N$.
In particular, we have $x_N \in B_r(x_0) \cap A$. As $x_0 \not\in A$, this implies
\[
\left(B_r(x_0) \backslash \{x_0\}\right) \cap A \not= \emptyset.
\]
So in this case, we have $x_0 \in A' (\subseteq \overline A)$.

\eqref{item2.1.2.ii} Suppose that $A$ is closed and let $(x_k)_{k \in \N}$ be a sequence
in $A$ with limit $x_0 \in X$. By \eqref{item2.1.2.i}, we have that $x_0 \in \overline{A} = A$.

Conversely, suppose that $A$ contains all limits of sequences in $A$. In order to show
that $A$ is closed, it suffices to prove $\overline{A} \subseteq A$. So let $x_0 \in \overline{A}$.
By \eqref{item2.1.2.i}, there exists a sequence $(x_k)_{k \in \N}$ in $A$ with $x_k \to x_0$
as $k \to \infty$. But then the hypothesis implies that $x_0 \in A$.

\eqref{item2.1.2.iii} Suppose that $A$ is open and let $(x_k)_{k \in \N}$ be a sequence in $X$
with limit $x_0 \in A$. Since $A$ is open, there exists an $r > 0$ such that $B_r(x_0) \subseteq A$.
Moreover, there exists a $N \in \N$ such that $d(x_k,x_0) < r$ for all $k \ge N$.
Thus $x_k \in A$ for $k \ge N$.

Conversely, suppose that $A$ is not open. Then $X \backslash A$ is not closed
by Theorem \ref{thm:complement}. According to \eqref{item2.1.2.ii}, there exists
a sequence $(x_k)_{k \in \N}$ in $X \backslash A$ that converges to an $x_0 \notin X\setminus A$, so $x_0\in A$.
But for this sequence, we have $x_k \not\in A$ for all $k \in \N$.
\end{proof}

%%%%%%%%%%%%%%%%%%%%
%\newpage
% -*- tex:uk -*-
% Course  MA20013
%
% uses concisenotes.tex
%


%%%%%%%%%%%%%%%%%%%%%%%%%%%%%%%%%%%%%%%%%%%%%%%%%%%%

\subsect{Completeness}

Why do we generally do calculus in $\R$ and not in $\Q$ (the rational numbers)?

The space $\Q$ has `holes', which means that many of the usual results would not
hold. For example, the intermediate value theorem: the function $f(x) = x^3 - 2$
satisfies $f(0) = -2$ and $f(2) = 6$, but there is no $x \in \Q$ such
that $f(x) = 0$ (because $\sqrt[3]{2}$ is not rational).

Moreover, sequences in $\Q$ might have a limit in $\R\setminus \Q$, as in the case of 
$$
\left(1+\frac1n\right)^n \to e \quad \textrm{as } n\to \infty.
$$

The following concept can be used to rule out `holes' of this sort.

\begin{definition}
A sequence $(x_k)_{k \in \N}$ in a metric space $(X,d)$ is called a \emph{Cauchy
sequence} if for every $\varepsilon > 0$ there exists a number $N \in \N$ such
that for all $m,k \ge N$,
\[
d(x_m,x_k) < \varepsilon.
\]
\end{definition}

\begin{theorem} \label{thm:bounded}
Let $(x_k)_{k \in \N}$ be a Cauchy sequence in the metric
space $(X,d)$. Then the set $\{x_k: k \in \N\}$ is bounded.
\end{theorem}

\begin{remark}
If $\{x_k: k \in \N\}$ is bounded, then we speak
of a bounded sequence.
\end{remark}

\begin{proof}
There exists an $N \in \N$ such that $d(x_m,x_k) < 1$ for $m,k \ge N$. Define
\[
M = \max\{d(x_1,x_N),\ldots,d(x_{N - 1},x_N),1\}.
\]
Then for all $k \in \N$,
\[
d(x_k,x_N) \le M.
\]
So for all $m,k \in \N$,
\[
d(x_m,x_k) \le d(x_m,x_N) + d(x_k,x_N) \le 2M.
\]
We conclude that $\textrm{diam} \left(\{x_k: k \in \N\}\right) \leq 2M$.
\end{proof}

\begin{theorem} \label{thm:Cauchy}
Any convergent sequence is a Cauchy sequence.
\end{theorem}

\begin{proof}
Let $(x_k)_{k \in \N}$ be a convergent sequence in the metric space $(X,d)$.
Denote its limit by $x_0$. Fix $\varepsilon > 0$. Then there exists an $N \in \N$
such that $d(x_k,x_0) < \frac{\varepsilon}{2}$ for $k \ge N$. For $m,k \ge N$,
we then also have
\[
d(x_m,x_k) \le d(x_m,x_0) + d(x_k,x_0) < \varepsilon,
\]
as required.
\end{proof}

\begin{corollary}
Any convergent sequence in a metric space is bounded.
\end{corollary}

\begin{proof}
This follows from Theorem \ref{thm:Cauchy} and Theorem \ref{thm:bounded}.
\end{proof}

\begin{definition}
A metric space $(X,d)$ is called \emph{complete} if every Cauchy sequence in $X$ is
convergent to a limit in $X$.

A complete normed space is called a \emph{Banach} space.
\end{definition}

\np

\begin{example}
The Euclidean spaces $(\R^n, d_2)$ are complete (here $d_2 = \|\cdot \|_2$ is the Euclidean distance - and norm - defined in Example \ref{Eucl:d2}). 

Indeed, let $(x_k)_{k \in \N} \subset \R^n$ be a Cauchy sequence, with $x_k=(x_k^{(1)},\dots,x_k^{(n)})$; namely, for every $\varepsilon>0$ 
there exists $N\in \N$ such that whenever $m,k\geq N$, we have that $\|x_k -x_m\|_2<\varepsilon$.

Since $\|x_k -x_m\|_\infty \leq \|x_k -x_m\|_2$ (see, e.g. Problem Sheet 5, Homework question 2(i)), it follows that  
$$
\|x_k -x_m\|_\infty = \max_{i=1,\dots,n} |x_k^{(i)}-x_m^{(i)}| < \varepsilon.
$$
This implies in particular that for every fixed $i$ the sequence $(x_k^{(i)})$ of the $i$-th components is a Cauchy sequence in $\R$, hence convergent. 
Let $x_0^{(i)}:= \lim_k x_k^{(i)}$, and let $x_0:=(x_0^{(1)},\dots,x_0^{(n)})\in \R^n$. Since 
$$
\|x_k - x_0\|_2 \leq \sqrt n \|x_k -x_0\|_\infty = \max_{i=1,\dots,n} |x_k^{(i)}-x_0^{(i)}|,
$$
we have that $\|x_k - x_0\|_2 \to 0$ as $k\to \infty$, and so the sequence $(x_k)_{k \in \N}$ converges to $x_0$ in $d_2$.
\end{example}

\np

\begin{example}
Consider the metric subspace $\Q$ of $\R$.
Consider the sequence $(x_k)_{k \in \N}$ in $\Q$ with
\[
x_k = \max\left\{ \frac{i}{k}: i \in \N \,\textrm{ with } i^2 < 2k^2\right\}.
\]
This is the largest fraction less than $\sqrt{2}$ with denominator $k$, and
so we always have
\[
\sqrt{2} - \frac{1}{k} < x_k < \sqrt{2}.
\]
In particular, we have
\[
|x_m - x_k| < \frac{1}{N}
\]
when $m,k \ge N$. It follows that $(x_k)_{k \in \N}$ is a Cauchy sequence
in $\Q$. But is does not have a limit in $\Q$.
Hence $\Q$ is not complete.
\end{example}

\np

\begin{example}
Consider an arbitrary set $X \not= \emptyset$ with the discrete metric
\[
d(x,y) = \begin{cases} 0 & \text{if $x = y$}, \\ 1 & \text{if $x \not= y$}. \end{cases}
\]
We have seen: a sequence $(x_k)_{k \in \N}$ is convergent in $(X,d)$ with limit $x_0$ if there
exists an $N \in \N$ such that $x_k = x_0$ for all $k \ge N$.
What does it mean for a sequence to be a Cauchy sequence in $(X,d)$?

For any $\varepsilon > 0$, we need to have an $N \in \N$ such that $d(x_m,x_k) < \varepsilon$
for $m,k \ge N$. If $\varepsilon \le 1$, then this means $x_m = x_k$ for $m,k \ge N$.
Therefore a Cauchy sequence is convergent. Thus a metric space with
the discrete metric is always complete.
\end{example}

\np

\begin{theorem} \label{thm:B}
Let $S\neq \emptyset$ and $B(S)$ the set of bounded functions $f : S \to \R$.
For $f,g \in B(S)$, let
\[
d_\infty(f,g) =\|f-g\|_\infty = \sup_{s \in S} |f(s) - g(s)|.
\]
Then $(B(S),d_\infty)$ is a complete metric space.
\end{theorem}

\begin{proof}
It is easy to check that $d_\infty$ is a metric on $B(S)$. 

To verify that the space is
complete, consider a Cauchy sequence $(f_k)_{k \in \N}$
in $(B(S), d_\infty)$: namely, for any given $\varepsilon > 0$, there exists an $N \in \N$ such that $d_\infty(f_m,f_k) < \varepsilon$
whenever $m,k \ge N$. By definition of $d_\infty$ this implies that for every $s\in S$
\[
|f_m(s) - f_k(s)| \le d_\infty(f_m,f_k) < \varepsilon.
\]
Hence $(f_k(s))_{k \in \N}$ is a Cauchy sequence in $\R$. 
\np
Since $\R$ is complete,
there exists a limit. Call the limit $f_0(s)$. Thus for every $s \in S$ we obtain
a number $f_0(s)$, and this gives rise to a function $f_0 : S \to \R$.
This is the candidate limit of $(f_k)_k$.

We claim that $f_0$ is bounded, namely $f_0\in B(S)$.

As $(f_k)_{k \in \N}$ is a Cauchy sequence, there exists $N\in \N$ such that
$d_\infty(f_m,f_k) < 1$ for $m,k \geq N$. Moreover, the function $f_N$ is bounded. So
there exists an $M \geq 0$ such that
\[
\sup_{s\in S} |f_N(s)| \le M.
\]
Then for all $k \geq N$ and all $s \in S$, we have
\[
|f_k(s)| = |f_k(s) - f_N(s) + f_N(s)| \leq |f_k(s) - f_N(s)| + |f_N(s)| < M + 1.
\]
Therefore,
\[
|f_0(s)| = \lim_{k \to \infty} |f_k(s)| \le M + 1.
\]
This proves that $f_0$ is bounded, i.e., $f_0 \in B(S)$.

Finally, we want to show that $d_\infty(f_k,f_0) = \|f_k-f_0\|_{\infty} \to 0$ as $k \to \infty$.
To this end, fix $\varepsilon > 0$. Choose $N \in \N$ such that $d_\infty(f_m,f_k) < \frac{\varepsilon}{2}$
when $m,k \geq N$. Then for any $s \in S$,
\[
|f_k(s) - f_0(s)| = \left|f_k(s) - \lim_{m \to \infty} f_m(s)\right| = \lim_{m \to \infty} |f_k(s) - f_m(s)| \le \frac{\varepsilon}{2}
\]
whenever $k \geq N$. This yields
\[
d_\infty(f_k,f_0) = \sup_{s \in S} |f_k(s) - f_0(s)| \le \frac{\varepsilon}{2} < \varepsilon
\]
for all $k \geq N$, and hence the desired convergence
\end{proof}

\np

\begin{example}
The space $C^0([0,1])$ with the metric
\[
d_\infty(f,g) = \max_{t \in [0,1]} |f(t) - g(t)| \quad (=\|f-g\|_{\infty}), \quad f,g \in C^0([0,1]),
\]
is complete. (It is actually a Banach space, since the metric can be defined by means of a norm.)

\medskip

Indeed, let $(f_k)_{k \in \N} \subset C^0([0,1])$ be a Cauchy sequence with respect to the metric $d_\infty$. Since 
$C^0([0,1]) \subset B([0,1])$ (continuous functions on a compact set are bounded), and $(B([0,1]), d_\infty)$ 
is complete by Theorem \ref{thm:B}, if follows that $(f_k)_{k \in \N}$ converges in $d_\infty$, with limit $f_0 \in B([0,1])$. 

On the other hand, since convergence in $d_\infty$ is the same as uniform convergence, and a uniform limit of 
continuous functions is continuous (by 
Theorem 1.8%%%%%%%\ref{unif:cont}
), $f_0 \in C^0([0,1])$. This proves the claim.

\end{example}


%\begin{example}[$L^2$ space]
%
%
%\end{example}

\bis

\begin{example}
The normed space $(C^1([0,1]),\|\cdot\|_{\infty})$ is not a Banach space.
\end{example}


\subsect{Continuous maps between metric spaces}

We now extend the concept of continuity to maps between metric spaces.

\begin{definition}[Continuity between metric spaces]
Let $(X,d_X)$ and $(Y,d_Y)$ be metric spaces. A map $f : X \to Y$
is \emph{continuous at a point $x_0 \in X$} if
\[
\forall\, \varepsilon > 0 \quad \exists\, \delta > 0 \,\, \textrm{such that } \, \forall \, x \in X: d_X(x,x_0) < \delta \Rightarrow d_Y(f(x),f(x_0)) < \varepsilon.
\]
If $f$ is continuous at every point of $X$, then we say that $f$ is \emph{continuous}.
\end{definition}

\bis

\begin{example}\label{continuous:primitive}
Let $(C^0([a,b]),d_\infty)$ be the metric space of continuous functions in $[a,b]$ with the uniform convergence metric, $d_\infty(f,g)=\|f-g\|_{\infty}$. Then the map 
$$
F: C^0([a,b]) \to C^0([a,b]), \quad f\mapsto F(f),\quad  F(f)(x):= \int_a^x f(t) dt
$$
is continuous.
\end{example}

\np

\begin{example}
On $\R^2$, consider three different metrics:
\begin{itemize}
\item the Euclidean metric $d_2(x,y) = \sqrt{(x_1 - y_1)^2 + (x_2 - y_2)^2}$,
\item the metric $d_1(x,y) = |x_1 - y_1| + |x_2 - y_2|$, and
\item the railway metric
\[
d_{\mathrm{R}}(x,y) = \begin{cases} d_2(x,y) & \text{if $x$, $y$, and $0$ are collinear}, \\ d_2(x,0) + d_2(y,0) & \text{else}. \end{cases}
\]
\end{itemize}
This gives rise to three metric spaces: $(\R^2,d_2)$, $(\R^2,d_1)$, and $(\R^2,d_{\mathrm{R}})$. Consider
the identity map 
$$
\id : \R^2 \to \R^2, \quad \, \id(x_1,x_2) = (x_1,x_2).
$$
The continuity of $\id$ depends on the metrics that we use.

First regard $\id$ as a map from $(\R^2,d_2)$ to $(\R^2,d_1)$. If $d_2(x,y) < \delta$, then
\[
d_1(x,y) = |x_1 - y_1| + |x_2 - y_2| < 2\delta,
\]
which we can make arbitrarily small by choosing $\delta$ small enough. Hence
$\id : (\R^2,d_2) \to (\R^2,d_1)$ is continuous.

Similarly, we see that $\id : (\R^2,d_1) \to (\R^2,d_2)$ is continuous.

Now we regard $\id$ as a map from $(\R^2,d_2)$ to $(\R^2,d_{\mathrm{R}})$.
Consider the point $(1,0) \in \R^2$. For $\eta \in \R$, also consider the
point $(1,\eta) \in \R^2$. We have
\[
d_2((1,0),(1,\eta)) = \eta
\]
and
\[
d_{\mathrm{R}}((1,0),(1,\eta)) = 1 + \sqrt{1 + \eta^2} \quad \text{if $\eta \not= 0$}.
\]
The latter does not become small when $\eta \to 0$, but the former does.
We conclude that the map $\id: (\R^2,d_2) \to (\R^2,d_{\mathrm{R}})$ is not continuous.

On the other hand, it is possible (and not too difficult) to show that
$\id: (\R^2,d_{\mathrm{R}}) \to (\R^2,d_2)$ is continuous.
\end{example}

\np



There are also stronger versions of continuity.
\begin{definition}\label{cont:ms}
Let $(X,d_X)$ and $(Y,d_Y)$ be metric spaces. A map $f : X \to Y$
is called \emph{uniformly continuous} if
\bis\nl
$\forall\, \varepsilon > 0 \,\, \exists\, \delta > 0 \,\, \textrm{such that } \,$
\[
 \forall \, x, y \in X: d_X(x, y) < \delta \Rightarrow d_Y(f(x), f(y)) < \varepsilon.
\]
\end{definition}

\begin{definition}
Let $(X,d_X)$ and $(Y,d_Y)$ be two metric spaces. A map $f : X \to Y$ is
called \emph{Lipschitz continuous} if there exists
a number $L \ge 0$ such that
\[
d_Y(f(x_1),f(x_2)) \le Ld_X(x_1,x_2)
\]
for all $x_1,x_2 \in X$.
\end{definition}

\begin{theorem} \label{thm:Lipschitz}
Every Lipschitz continuous map is uniformly continuous.
\end{theorem}

\begin{proof}
Exercise!
\end{proof}


\begin{definition} \label{def:contraction}
Let $(X,d)$ be a metric space. A map $f : X \to X$ is called a \emph{contraction}
if there exists a number $\alpha < 1$ such that
\[
d(f(x),f(y)) \le \alpha d(x,y)
\]
for all $x,y \in X$.
\end{definition}

\np

\begin{remark}
A contraction is automatically Lipschitz continuous, but the
condition is stronger than Lipschitz continuity, because we
require that $\alpha < 1$. To understand the meaning of the name, `contraction', just note that by definition, a 
ball of radius $1$ would be mapped inside a ball of radius $\alpha<1$, namely the set shrinks under the action of $f$.
\end{remark}

\np

\begin{theorem}[Continuity of compositions]
Let $(X,d_X)$, $(Y,d_Y)$, and $(Z,d_Z)$ be metric spaces.
Suppose that $f : X \to Y$ and $g : Y \to Z$ are maps. Let $x_0 \in X$. If $f$ is continuous
at $x_0$ and $g$ is continuous at $f(x_0)$, then $g \circ f$ is continuous at $x_0$.
\end{theorem}

\begin{proof}
Fix $\varepsilon > 0$. As $g$ is continuous at $f(x_0)$, there exists a $\rho > 0$ such
that
\[
d_Y(y,f(x_0)) < \rho \Rightarrow d_Z(g(y),g(f(x_0))) < \varepsilon.
\]
As $f$ is continuous at $x_0$, there exists a $\delta > 0$ such that
\[
d_X(x,x_0) < \delta \Rightarrow d_Y(f(x),f(x_0)) < \rho.
\]
So if $d_X(x,x_0) < \delta$, it follows that
$d_Z(g(f(x)),g(f(x_0))) < \varepsilon$, as required.
\end{proof}

\np

In metric spaces, continuity can be equivalently stated in terms of sequences.


\begin{theorem}\label{th:sequential}
Let $(X,d_X)$ and $(Y,d_Y)$ be metric spaces. Let $x_0 \in X$ and let $f : X \to Y$
be a map. The following are equivalent.
\begin{enumerate}
\item The map $f$ is continuous at $x_0$.
\item If $(x_k)_{k \in \N}$ is a sequence in $X$ with $x_k \to x_0$ as $k \to \infty$,
then $f(x_k) \to f(x_0)$ as $k \to \infty$.
\end{enumerate}
\end{theorem}

\np

\begin{proof}
$(1) \Rightarrow (2)$: Suppose that $f$ is continuous at $x_0$, and let $\varepsilon > 0$. Then there exists a $\delta > 0$ such that
\[
d_X(x,x_0) < \delta \Rightarrow d_Y(f(x),f(x_0)) < \varepsilon.
\]
Now let $(x_k)_{k \in \N}$ be
a sequence with $x_k \to x_0$ as $k \to \infty$. There exists an $N \in \N$ such that $d_X(x_k,x_0) < \delta$ whenever $k \ge N$.
So for $k \ge N$, we also have $d_Y(f(x_k),f(x_0)) < \varepsilon$. It then follows
that $f(x_k) \to f(x_0)$ as $k \to \infty$.

\bis

$(2) \Rightarrow (1)$: Now suppose that $f$ is not continuous at $x_0$. That is,
\[
\exists \,\varepsilon > 0 \,\,\textrm{such that } \, \forall\, \delta > 0 \, \exists\, x \in X: d_X(x,x_0) < \delta \wedge d_Y(f(x),f(x_0)) \ge \varepsilon.
\]
Choose $\varepsilon > 0$ with this property and consider $\delta = \frac{1}{k}$ for an
arbitrary $k \in \N$. We conclude that there exists a point $x_k \in X$ with
$d_X(x_k,x_0) < \frac{1}{k}$ and $d_Y(f(x_k),f(x_0)) \ge \varepsilon$.
Since we have such a point $x_k$ for every $k \in \N$, we obtain a sequence
$(x_k)_{k \in \N}$. This sequence satisfies $x_k \to x_0$, but $f(x_k) \not\to f(x_0)$
as $k \to \infty$.
\end{proof}


\np

\begin{remark}
Note that the previous theorem and Example \ref{continuous:primitive} imply 
Theorem 1.10%%%%%%%%%\ref{limit:integral}
.
\end{remark}

\bis



Continuity can also be stated in terms of open sets.

\begin{theorem} \label{thm:continuity_and_open_sets}
Let $(X,d_X)$ and $(Y,d_Y)$ be metric spaces and $f : X \to Y$ a map.
The following are equivalent.
\begin{enumerate}
\item \label{item3.1.2.i} The map $f$ is continuous.
\item \label{item3.1.2.ii} If $U \subset Y$ is open in $Y$, then
\[
f^{-1}(U) = \{x \in X: f(x) \in U\}
\]
is open in $X$.
\item \label{item3.1.2.iii} If $F \subset Y$ is closed in $Y$, then $f^{-1}(F)$ is closed in $X$.
\end{enumerate}
\end{theorem}

\begin{proof}
We use the notation $B_r^X(x_0)$ and $B_r^Y(y_0)$ for open balls in $X$ and $Y$, respectively.

Firstly we show that the statements \eqref{item3.1.2.i} and \eqref{item3.1.2.ii} are equivalent.

Suppose that $f$ is continuous, and let $U \subset Y$ be open. Let $x_0 \in f^{-1}(U)$.
Since $U$ is open, and $f(x_0) \in U$, there exists an $\varepsilon > 0$ such that $B_\varepsilon^Y(f(x_0)) \subset U$.
Choose a corresponding $\delta > 0$ such that
\[
d_X(x,x_0) < \delta \Rightarrow d_Y(f(x),f(x_0)) < \varepsilon.
\]
This condition can be rewritten in the form
\[
x \in B_\delta^X(x_0) \Rightarrow f(x) \in B_\varepsilon^Y(f(x_0)).
\]
Therefore, we have $B_\delta^X(x_0) \subset f^{-1}(U)$ and $f^{-1}(U)$ is open.

Conversely, suppose that $f^{-1}(U)$ is open for every open set $U \subset Y$.
Let $x_0 \in X$ and fix $\varepsilon > 0$. Now $B_\varepsilon^Y(f(x_0))$ is an open
set in $Y$ by Lemma \ref{lemma:balls}. So $V := f^{-1}(B_\varepsilon^Y(f(x_0)))$ is open
in $X$. We have $x_0 \in V$ (because $f(x_0) \in B_\varepsilon^Y(f(x_0))$). Hence there
exists a number $\delta > 0$ such that $B_\delta^X(x_0) \subset V$.
As a consequence, we have the following: if $d_X(x,x_0) < \delta$, then
$x \in B_\delta^X(x_0) \subset V$, so $f(x) \in B_\varepsilon^Y(f(x_0))$, so
$d_Y(f(x),f(x_0)) < \varepsilon$. Thus $f$ is continuous.

\np

Next we show that \eqref{item3.1.2.ii} and \eqref{item3.1.2.iii} are equivalent.
\nl
To this end, we use the fact that for any $A \subset Y$, we have $f^{-1}(Y \backslash A) = X \backslash f^{-1}(A)$.
Moreover, by Theorem \ref{thm:complement}, a set $A \subset Y$ is closed if, and only if,
its complement $Y \backslash A$ is open; and a similar statement holds in $X$.

If \eqref{item3.1.2.ii} is true and $F \subset Y$ is closed, then $Y \backslash F$ is open.
Therefore $f^{-1}(Y \backslash F) = X \backslash f^{-1}(F)$ is open, and $f^{-1}(F)$ is closed.

Conversely, if \eqref{item3.1.2.iii} is true and $U \subset Y$ is open, then $Y \backslash U$ is
closed. Hence $f^{-1}(Y \backslash U) = X \backslash f^{-1}(U)$ is closed, and $f^{-1}(U)$ is
open.
\end{proof}


%\subsubsection
\subsect{Continuous linear maps between normed spaces}

%\begin{definition}[Linear maps]
%Let $(X,\| \cdot \|_X)$ and $(Y,\|\cdot  \|_Y)$ be normed spaces. We denote with $L(X,Y)$ the space of linear and bounded functions between $X$ and $Y$. 
%\end{definition}

Recall the definition of linear maps between vector spaces. Given vector spaces $X$ and $Y$ on $\R$, a map $f: X\to Y$ is called linear if 
for every $\lambda,\mu \in \R$ and for every $x,y \in X$, we have that 
$$
f(\lambda x+ \mu y) = \lambda f(x) + \mu f(y).
$$
If $(X,\| \cdot \|_X)$ and $(Y,\|\cdot  \|_Y)$ are normed spaces, then they are vector spaces, so we can consider linear maps between them, and continuity is 
defined as in Definition \ref{cont:ms}, since normed spaces are metric spaces.

\begin{theorem}[Continuity of linear maps% between normed spaces
]\label{cont:0}
Let $(X,\| \cdot \|_X)$ and $(Y,\|\cdot  \|_Y)$ be normed spaces and let $f : X \to Y$ be a linear map. The following are equivalent:
\begin{enumerate}
\item $f$ is continuous in $X$;
\item $f$ is continuous at $0$;
\item there exists a constant $M > 0$  such that $\forall \, x \in X$: $\|f(x)\|_Y \leq M \|x\|_X$.
\end{enumerate}
%(Namely $f$ is continuous in $X$ if and only if $f$ is continuous at $0$).
\end{theorem}

\begin{proof}
$(1)\Rightarrow (2)$: By assumption $f$ is continuous at every $x\in X$, so in particular at $x=0\in X$.

\smallskip
$(2)\Rightarrow (1)$: We use the sequential characterisation of continuity in Theorem \ref{th:sequential}. Let $x\neq 0$ and let $(x_k)_k\subset X$ be a sequence with 
$\|x_k-x\|_X \to 0$ as $k\to \infty$. This means that the sequence $y_k:= x_k-x$ (still $\subset X$) converges to $0$ in norm. So, since $f$ is continuous at $0$ by assumption and 
$\|y_k-0\|_X\to 0$, we have that $\|f(y_k)-f(0)\|_Y\to 0$. But $f$ is linear, so $f(0)=0$ and $f(y_k) = f(x_k)-f(x)$; hence $\|f(x_k)-f(x)\|_Y\to 0$, which implies that $f$ is continuous at $x$.

\bis

$(2)\Rightarrow (3)$: By Definition \ref{cont:ms}, for every $\varepsilon>0$ there exists $\delta>0$ such that $\|x\|_X<\delta \Rightarrow \|f(x)\|_Y<\varepsilon$. 
Note that it is enough to show the claim for $x\neq 0$, since $f(0)=0$. So, let $x\neq 0$, and define $y:=\frac\delta2 \frac{x}{\|x\|_X}$. Since $\|y\|_X<\delta$, by assumption 
$\|f(y)\|_Y<\varepsilon$; equivalently,
$$
\varepsilon > \|f(y)\|_Y = \frac\delta2 \frac{1}{\|x\|_X}\|f(x)\|_Y \quad  \Leftrightarrow \quad \|f(x)\|_Y \leq \frac{2\varepsilon}{\delta} \|x\|_X,
$$
hence the claim follows for $M=\frac{2\varepsilon}{\delta}$.

%We equivalently prove that $(\textrm{NOT} (3)) \Rightarrow (\textrm{NOT}(2))$. By assumption, for every $M>0$ there exists $x_M\in X$ such that 
%\begin{equation*}
%\|f(x_M)\|_Y > M \|x_M\|_X.
%\end{equation*}
%In particular this is true for $M=k$, for every $k\in \N$, namely there exists $x_k\in X$ such that 
%\begin{equation}\label{false}
%\|f(x_k)\|_Y > k \|x_k\|_X.
%\end{equation}
%Note that we can assume that $x_k\neq 0$ for every $k\in \N$, since $f(0)=0$. Hence, from \eqref{false} we have that 
%\begin{equation}\label{false2}
%\frac{1}{\|x_k\|_X} \, \|f(x_k)\|_Y > k \quad \Leftrightarrow \quad \left\|f\left(\frac{x_k}{\|x_k\|_X}\right)\right\|_Y > k,
%\end{equation}
%by the linearity of $f$. By setting $y_k:= \frac{x_k}{\|x_k\|_X}$, we have that $\|y_k\|_X=1$ for every $k$, and by \eqref{false2} $\|f(y_k)\|_Y\to \infty$ as $k\to \infty$. Finally, define 
%$$
%w_k:= \frac{y_k}{\|f(y_k)\|_Y}.
%$$
%Then 
%$$
%\|w_k\|_X  = \left\|\frac{y_k}{\|f(y_k)\|_Y}\right\|_X = \frac{\|y_k\|_X}{\|f(y_k)\|_Y} = \frac{1}{\|f(y_k)\|_Y} \to 0,
%$$
%but on the other hand 
%$$
%f(w_k) = f\left(\frac{y_k}{\|f(y_k)\|_Y}\right) = \frac{f(y_k)}{\|f(y_k)\|_Y} 
%$$
%which implies that $\|f(w_k)\|=1$, so it does not converge to zero, contradicting continuity at zero.

%\smallskip
$(3)\Rightarrow (2)$: We use the sequential characterisation of continuity in Theorem \ref{th:sequential}. Let $(x_k)\subset X$ be such that $\|x_k-0\|_X \to 0$ as $k\to \infty$. Then by assumption 
$$
\|f(x_k)\|_Y \leq M\|x_k\|_X \to 0 \quad \textrm{as } k\to \infty,
$$
namely, since $f(0)=0$, $\|f(x_k) - f(0)\|_Y \to 0$, which implies the continuity of $f$ at zero.

(Alternatively: by linearity, the assumption implies that $f$ is Lipschitz continuous, hence continuous, which would prove $(3)\Rightarrow (1)$.)
\end{proof}

\np

\begin{definition}
Let $(X,\| \cdot \|_X)$ and $(Y,\|\cdot  \|_Y)$ be normed spaces. We denote with $L(X,Y)$ the \h{space of linear and continuous functions} (also called linear operators) between $X$ and $Y$.
\end{definition}


\begin{remark}
From Theorem \ref{cont:0} it follows that the image of a bounded set under a linear and continuous operator is also bounded. Because of this property, continuous linear operators are also known as bounded operators.
\end{remark}

%\medskip

It is easy to show that $L(X,Y)$ is a real vector space. We can then define a norm on it.

%\medskip
\np

\begin{definition}%[Operator norm]
\label{op:norm}
Let $(X,\| \cdot \|_X)$ and $(Y,\|\cdot  \|_Y)$ be normed spaces and $f\in L(X,Y)$. We define the \h{operator norm} of $f$ as 
$$
\|f\|_{{\textrm{op}}} = \inf\{M\ge 0 : \|f(x)\|_Y \le M \|x\|_X \mbox{ for all } x\in X\}.
$$
% exists as the set of all such $M$ is closed, nonempty, and bounded from below).
\end{definition}

\begin{remark}
The operator norm is a norm, namely it satisfies the axioms $(N1)-(N3)$. 

Moreover, by definition,
$$
\|f(x)\|_Y \le \|f\|_{{\textrm{op}}}\, \|x\|_X \mbox{ for all } x\in X,
$$
hence the $\inf$ in Definition \ref{op:norm} is actually a minimum.
\end{remark}

\np

\begin{remark}
 It is important to bear in mind that this operator norm depends on the choice of norms for the normed vector spaces $X$ and $Y$.
\end{remark}

\begin{proposition}[Equivalent definitions of the operator norm]
The following definitions are all equivalent:
$$
{\begin{aligned}\|f\|_{{op}}&=\inf\{M\geq 0:\|f(x)\|_Y\leq M\|x\|_X{\mbox{ for all }}x\in X\}\\
&=\sup\{\|f(x)\|_Y: x\in X{\textrm{ with }}\|x\|_X\leq 1\}\\
&=\sup\{\|f(x)\|_Y: x\in X{\mbox{ with }}\|x\|_X=1\}\\
&=\sup \left\{{\frac  {\|f(x)\|_Y}{\|x\|_X}}: x\in X{\mbox{ with }}x\neq 0\right\}.
\end{aligned}}
$$
\end{proposition}
\begin{proof} In problem class.
\end{proof}
\np

\begin{example}[$\ell^2$-space]
Consider the sequence space $\ell^2$ defined by
$$
\ell^2 = \left\{ (a_k)_{k \in \N}: \; a_k \in \mathbb{R}, \; \sum_k |a_k|^2 < \infty \right\}.
$$
Now take a bounded sequence $s = (s_k)_k$, namely with $\|s\|_\infty:=\sup _k |s_k|<\infty$. 

Define an operator $T_s: (\ell^2,\|\cdot\|_2) \to (\ell^2, \|\cdot\|_2)$ as:
$$
(a_k)_k \mapsto T((a_k)_k) = (s_k a_k)_k.
$$
The operator $T_s$ is linear and continuous (bounded) with operator norm
$$
\| T_s\|_{op} = \sup_k|s_k| =\| s \|_{\infty}.
$$
\end{example}




\subsect{Contraction Mapping Theorem}

Many problems can be solved by successive approximations. For example,
consider \emph{Newton's method} for finding a zero of a function $f : \R \to \R$.
Assuming that $f$ is differentiable, we choose a point $x_1 \in \R$ and we
recursively define
\[
x_{k + 1} = x_k - \frac{f(x_k)}{f'(x_k)}, \quad k = 1,2,3,\ldots
\]
Geometrically, this amounts to approximating the graph of $f$ by
a tangent line at $(x_k,f(x_k))$ and finding its intersection with
the $x$-axis%
% (see Figure \ref{fig:Newton})
.

\iffalse\boxdiag4\else
%\begin{figure}[ht]
%\labellist
%\pinlabel $x_{k+1}$ at 135 150
%\pinlabel $x_{k}$ at 335 150
%\endlabellist
\begin{center}
\includegraphics[width=0.4\textwidth]{Newton2}
%\caption{Graphical representation of a step in Newton's method.}
%\label{fig:Newton}
\end{center}
%\end{figure}
\fi

In many cases, the sequence $(x_k)_{k \in \N}$ will converge to a zero of $f$ as $k \to \infty$.


In this section, we will see that this is an example of a more general principle,
called the \emph{Contraction Mapping Principle} or Banach Fixed Point Theorem. 

%We will state and prove the theorem in $\R^n$, but it can be stated and 
%proved in a metric space with pretty much no change.


\bis 

First of all, we define what we mean by fixed point.


\begin{definition}
Let $X\neq \emptyset$, and let $f : X \to X$ be a map. A point $x_0 \in X$ is called a \emph{fixed point}
of $f$ if $f(x_0) = x_0$.
\end{definition}

\begin{theorem}[Contraction Mapping Principle\index{Contraction Mapping Principle}] \label{thm:contraction}
Let $(X,d)$ be a complete metric space, with $X\neq \emptyset$, and let $f : X \to X$ be a contraction. Then $f$ has a unique fixed point.
\end{theorem}

\np

\begin{proof}
Choose an arbitrary point $x_1 \in X$ and define the sequence $(x_k)_{k \in \N}$
recursively as follows:
\[
x_{k + 1} = f(x_k), \quad k = 1,2,3,\ldots
\]
We claim that this is a Cauchy sequence. In order to prove the claim,
define $\delta_k = d(x_{k + 1},x_k)$. Note first
that we have a number $\alpha \in (0,1)$ such that for all $k \in \N$,
\[
\delta_{k +1} = d(x_{k + 2},x_{k + 1}) = d(f(x_{k + 1}),f(x_k)) \le \alpha d(x_{k+ 1},x_k) = \alpha \delta_k,
\]
because $f$ is a contraction. Hence
$\delta_2 \le \alpha \delta_1$,
$\delta_3 \le \alpha \delta_2 \le \alpha^2 \delta_1$,
etc., and
\[
\delta_k \le \alpha^{k - 1} \delta_1, \quad k \in \N.
\]
If $m,k \in \N$ with $m < k$, then
\begin{align*}
d(x_m,x_k) & \le \delta_m + \cdots + \delta_{k - 1} \\
& \le \left(\alpha^{m - 1} + \cdots + \alpha^{k - 2}\right) \delta_1 \\
& \le \delta_1 \sum_{i = m - 1}^\infty \alpha^i = \frac{\delta_1 \alpha^{m - 1}}{1 - \alpha}.
\end{align*}
In the last step we have used the fact that $\alpha \in (0,1)$.
For the same reason, we have
\[
\frac{\delta_1 \alpha^{m - 1}}{1 - \alpha} \to 0 \quad \text{as $m \to \infty$}.
\]
Hence $(x_k)_{k \in \N}$ is a Cauchy sequence.

Because $(X,d)$ is a complete metric space, every Cauchy sequence converges.
In particular, there exists a point $x_0 \in X$ such that $d(x_k,x_0)\to 0$ as
$k \to \infty$.

Since $f$ is a contraction, it is Lipschitz continuous. In particular
it is continuous by Theorem \ref{thm:Lipschitz}. Hence
\[
f(x_0) = f\left(\lim_{k \to \infty} x_k\right) = \lim_{k \to \infty} f(x_k) = \lim_{k \to \infty} x_{k + 1} = x_0.
\]
So $x_0$ is a fixed point of $f$.

It is a unique fixed point, because for every other fixed point $y_0 \in X$,
we have
\[
d(x_0,y_0) = d(f(x_0),f(y_0)) \le \alpha d(x_0,y_0).
\]
Because $\alpha < 1$, it follows that $d(x_0,y_0) = 0$, and so $x_0 = y_0$.
\end{proof}

\bis

\begin{remark}
The proof also shows how to find the fixed point: choose a point $x_1 \in X$,
define recursively $x_{k + 1} = f(x_k)$ for $k = 1,2,\ldots$ and find the
limit $x_0 = \lim\limits_{k \to \infty} x_k.$
\end{remark}



%%%%%%%%%%%%
\np

\begin{example}[Solving algebraic equations]
Show that the equation 
\begin{equation}\label{alg:eq}
x^7 - x^3 -21 x+5 =0
\end{equation}
has a unique solution in the interval $[0,1]$.
%\bigskip
\bis

Note that the equation \eqref{alg:eq} can be equivalently rewritten as 
\begin{equation}
x= \frac{x^7 - x^3 +5}{21};
\end{equation}
so, by defining $f:[0,1] \to \R$ as $f(x):= \frac{x^7 - x^3 +5}{21}$, the original equation \eqref{alg:eq} can be written as a fixed point problem $x=f(x)$.

In particular, there exists a unique solution to \eqref{alg:eq} if and only if there is a unique fixed point of $f$ in $[0,1]$.
\medskip

This would be a direct consequence of the Contraction Mapping Theorem, \textbf{provided} $f$ is a contraction from a complete metric space in itself, so all is left is to verify this:

\begin{itemize}
\item We prove that $f([0,1])\subset [0,1]$: Note that, for every $x\in [0,1]$ we have
$$
0< \frac{- 1 +5}{21} f(x)= \frac{x^7 - x^3 +5}{21} \leq f(x):= \frac{1+5}{21} <1,
$$
which proves the claim.

\item Clearly $([0,1],|\cdot|)$ is a complete metric space, since $[0,1]$ is a closed subset of $\R$ and $(\R,|\cdot|)$ is a complete metric space (by Problem Sheet 7, Practice question 1).

\item Finally, we need to prove that $f$ is a contraction, namely that there exists $0<\alpha<1$ such that $|f(x) - f(y)| \leq \alpha |x-y|$ for every $x,y\in [0,1]$.

To see this, we apply the Intermediate Value Theorem to $f$ (which is differentiable with continuous derivative in particular) to obtain that 
$$
f(x) - f(y) = f'(\xi)(x-y), \quad \textrm{for some }  \xi \in [x,y] \quad (\textrm{or } \xi \in [y,x] \, \textrm{if } x>y).
$$
Hence
\begin{equation}\label{intermediate}
|f(x) - f(y)| \leq |f'(\xi)| |x-y| \quad \forall\, x,y \in [0,1].
\end{equation}
Now note that we can estimate $f'$ as we estimated $f$ previously, in $[0,1]$:
$$
 \frac{- 3}{21} \leq f'(\xi) = \frac{7x^6 - 3x^2}{21} \leq  \frac{7}{21} \quad \Rightarrow \quad \|f'\|_{\infty} = \sup_{\xi\in [0,1]} |f'(\xi)| <1.
$$
From \eqref{intermediate} it follows immediately that 
$$
|f(x) - f(y)| \leq \alpha |x-y| \quad \forall\, x,y \in [0,1], \quad \alpha = \|f'\|_{\infty} <1,
$$
hence $f$ is a contraction.
\end{itemize}
%By the Contraction Mapping Theorem there exists a unique fixed point for $f$ in $[0,1]$, as claimed.
\end{example}

\begin{remark}
Note that one could have chosen, in the previous example, a function $g:[0,1]\to \R$ as 
$$
g(x):= x^7 - x^3 - 20 x + 5,
$$
for which the equation \eqref{alg:eq} can be written as a fixed point problem as well, as $x = g(x)$.

However it is easy to see that then $g([0,1]) \nsubseteq [0,1]$, which makes $g$  not admissible.
\end{remark}

\np

\begin{example}[Unique local solutions to differential equations]
Let $f:\R^2 \to \R$ be a continuous and bounded function; namely $f: (\R^2,\|\cdot \|) \to (\R, |\cdot|)$ is continuous and there exists a constant $M>0$ such that 
$|f(x,y)|\leq M$ for every $x, y\in \R$. Assume also that there exists a constant $L>0$ such that 
\begin{equation}\label{L2}
|f(x,y)-f(x,z)| \leq L|y-z| \quad \forall \,  x, y, z\in \R.
\end{equation}

Show that for every $(x_0,y_0) \in \R^2$ the ordinary differential equation (ODE)
\begin{equation}\label{ODE}
y'(x) = f(x,y(x)), \quad y(x_0)=y_0
\end{equation}
has a unique solution in the interval $I:= \big[x_0-\frac{1}{2L}, x_0+\frac{1}{2L}\big]$.
\bigskip

We want to write the ODE as a fixed point problem; we do it by integrating the equation between $x_0$ and $x$:
$$
y(x) - y_0 = \int_{x_0}^x y'(t) dt = \int_{x_0}^x f(t,y(t)) dt  \quad \Leftrightarrow \quad y(x) = y_0 + \int_{x_0}^x f(t,y(t)) dt.
$$
We can define for $y\in C^0(I)$ a function $F(y): I\to \R$, as $(F(y))(x):= y_0 + \int_{x_0}^x f(t,y(t)) dt$; hence the ODE becomes $y = F(y)$. 
In particular, there exists a unique solution to the ODE if and only if there is a unique fixed point of $F$ in $C^0(I)$.
\medskip

Again, this would be a direct consequence of the Contraction Mapping Theorem, \textbf{provided} $F$ is a contraction from a complete metric space into itself,  so all is left is to verify this:

\begin{itemize}
\item We prove that $F(C^0(I))\subset C^0(I)$, namely that $F(y) \in C^0(I)$ for every $y \in C^0(I)$. For $x_1,x_2\in I$, $x_1<x_2$, we have 
\begin{align*}
(F(y))(x_2) - (F(y))(x_1) &= \int_{x_0}^{x_2} f(t,y(t)) dt - \int_{x_0}^{x_1} f(t,y(t)) dt \\&= \int_{x_1}^{x_2} f(t,y(t)) dt \leq M\, |x_2-x_1|,
\end{align*}
and similarly if $x_1>x_2$; so $F(y)$ is a Lipschitz function, hence continuous. 

\item Clearly $(C^0(I), \|\cdot\|_{\infty})$ is a complete metric space.

\item Finally, we need to prove that $F$ is a contraction. Let $y_1, y_2 \in C^0(I)$; then for every $x\in I$
\begin{align*}
[F(y_1) - F(y_2)](x) &= \int_{x_0}^{x} f(t,y_1(t)) dt - \int_{x_0}^{x} f(t,y_2(t)) dt \\
& \leq L \int_{x_0}^x |y_1(t) - y_2(t)| dt \\&\leq L|x_0-x| \|y_1-y_2\|_{\infty} \leq \frac12 \|y_1-y_2\|_{\infty}.
\end{align*}
So this implies that 
$$
\| F(y_1) - F(y_2)\|_{\infty} \leq \frac12 \|y_1-y_2\|_{\infty},
$$
hence $F$ is a contraction.
\end{itemize}

%Thus, by the Contraction Mapping Theorem there exists a unique fixed point for $F$ in $C^0(I)$, as claimed.


\end{example}



%%%%%%%%%%%%%




\begin{example}[Revisit the Newton method at the start of this section]
Let $f : \R \to \R$ be a function with continuous derivatives up to second order. We want to reformulate Newton's method
as an example of the Contraction Mapping Principle. Guided by the iteration
formula, we define $F : \R \to \R$ by
\[
F(x) := x - \frac{f(x)}{f'(x)}
\]
for $x \in \R$ with $f'(x) \not= 0$. Is this a contraction?

We have a criterion involving the derivative of $F$, so we compute
\[
F'(x) = 1 - \frac{f'(x)}{f'(x)} + \frac{f(x) f''(x)}{(f'(x))^2} = \frac{f(x) f''(x)}{(f'(x))^2}.
\]
This does not necessarily satisfy $|F'(x)| < 1$, and thus $F$ is not
a contraction in general. In fact, it is not even necessarily well-defined on all of
$\R$, as $f'$ may have zeros. 
\nl
On the other hand, we know that
Newton's method does not work in every situation, so it's no surprise
that we need some more restrictions.

Suppose for simplicity that we look for a simple zero $x_0 \in \R$ of
$f$, i.e., $f(x_0) = 0$ \textbf{but} $f'(x_0)\not= 0$. Then
\[
\lim_{x \to x_0}F'(x)= \lim_{x \to x_0}\frac{f(x) f''(x)}{(f'(x))^2}
% \to 0\text{\quad as\quad}
%x \to x_0
=0.
\]
Hence there exists a number $a > 0$ such that 
$$|F'(x)| \le \frac{1}{2}
\text{\quad on\quad}
[x_0 - a,x_0 + a]=:I.$$
Hence for $x, y \in I$,
we have
\[
|F(x) - F(y)| \le \frac{1}{2} |x - y|.
\]
In particular, this means that for $x \in I$,
we have
\[
|F(x) - x_0| = |F(x) - F(x_0)| \le \frac{1}{2} |x - x_0| < a.
\]
It follows that $F(I) \subset I$.
Hence we can define a map $G :I \to I$
with $G(x) = F(x)$ on $I$, and $G$ is a contraction.
\np
Since $I$ is a complete metric space, Theorem \ref{thm:contraction}
applies. We obtain a unique point $y_0 \in I$ with $G(y_0) = y_0$,
i.e.,
\[
y_0 = y_0 - \frac{f(y_0)}{f'(y_0)}.
\]
This implies $f(y_0) = 0$.

\bis

Of course we have started with the assumption $f(x_0) = 0$, so this outcome
is no surprise. In fact, the uniqueness of the fixed point implies $y_0 = x_0$.
As $x_0$ is unknown when we use Newton's method, we cannot determine $a$
in advance, and so in practice, we often have to take an educated guess
for the initial point. But these observations prove that under certain circumstances,
Newton's method will work (i.e.\ converge), as long as we choose a sufficiently
good approximation as initial point.
\end{example}


% -*- tex:uk -*-
% Course  MA20013
%
% uses concisenotes.tex
%


\sect{Topological aspects in $\R^n$}
\gdef\fpdf{./frame/slnotesfigsanno4}  % needs \gdef to act inside the section, defines file for figs, start on pag. 4

%%%%%%%%%%%%%%%%%%%%%%%%%%%%%%%%%%%%%%%%%%%%%%%%%%%%



In this last section we focus on a very special metric space (normed space in fact): $(\R^n, \|\cdot \|)$ with the Euclidean distance, which we sometimes denoted with $\|\cdot\|_2$. We will state and prove some results that generalise similar results in one dimension, and give indications about extensions to the case of more general metric spaces.


\subsect
{Compactness and convergence}

Consider the following statements for intervals of the form $[a,b]$ with $a < b$.
\begin{enumerate}
\item If $f : [a,b] \to \R$ is continuous, then it attains its minimum and its maximum (Weierstrass's extreme value theorem).
\item If $(x_k)_{k \in \N}$ is a sequence in $[a,b]$, then there exists a subsequence with a limit in $[a,b]$ (Bolzano-Weierstrass theorem).
\end{enumerate}
Both statements do not remain true if $[a,b]$ is replaced by, say, $(a,b)$ or $[a,\infty)$. The underlying property that distinguishes intervals of this form is called compactness.
In this section, we discuss compactness in $\R^n$ (this concept can be generalised to metric spaces with minor changes).

%\bigskip
\np

\begin{definition}
A subset $A\subset \R^n$ is called \emph{sequentially compact} if every sequence in $A$ has a convergent subsequence, and the limit of the subsequence belongs to $A$.
\end{definition}

\bis

\begin{example}
An interval of the form $[a, b]$ with $a < b$, equipped with the usual metric in $\R$ ($|\cdot|$), is sequentially compact.
\end{example}


The next result gives a relation between bounded sequences and convergent subsequences. 

%First we recall the definition of bounded sets in $\R^n$ {\color{red}(special case of the definition for metric spaces).}

%\begin{definition}  A set $S \subset \R^n$ is bounded if there exists a number $R \geq 0$ such that $\|x\|\leq R$ for all $x \in S$.
%A sequence $(x_k)_{k\in\N}$ is bounded if the set $\{x_k : k \in \N\}$ is bounded.
%\end{definition}

\begin{theorem}[Bolzano-Weierstrass]
Every bounded sequence in $\R^n$ has a convergent subsequence (namely, is sequentially compact).
\end{theorem}

\np

\begin{proof} Let $(x_k)_{k\in\N}$ be a bounded sequence in $\R^n$ (namely there exists 
$M>0$ such that $\|x_k \| \leq M$ for every $k\in \N$) with $x_k =  (x_k^{(1)},\dots,x_k^{(n)})$.
Note that for any $i = 1,\dots,n$, we have $|x_k^{(i)}| \leq \|x_k\|$ (indeed we have seen that $\|\cdot\|_\infty \leq \|\cdot\|_2$), so the sequence $(x_k^{(i)})_{k\in\N}$ is bounded in $\R$.

By the Bolzano-Weierstrass theorem in $\R$, there exists a convergent sub-sequence of $(x_k^{(1)})_{k\in\N}$. That is, there exists an infinite subset $\Lambda_1\subset \N$ such that $(x_k^{(1)})_{k\in\Lambda_1}$ is convergent. Let $x_0^{(1)}$ denote its limit.

Now $(x_k^{(2)})_{k\in\Lambda_1}$ is a bounded sequence in $\R$ as well. Apply the same arguments to obtain an infinite set $\Lambda_2\subset \Lambda_1$ 
such that $(x_k^{(2)})_{k\in\Lambda_2}$ is convergent with limit $x_0^{(2)}$. Note that $(x_k^{(1)})_{k\in\Lambda_2}$, being a subsequence of a convergent sequence, still converges to $x_0^{(1)}$.

Apply the same arguments to the remaining coordinates in turn, constructing infinite subsets $\N \supset \Lambda_1\supset \dots\supset \Lambda_n$ chosen such that $x_k^{(i)}\to x_0^{(i)}$ as $k\to \infty$ while $k\in \Lambda_i$ for $i=1,\dots,n$.

Then for $(x_k)_{k\in \Lambda_n}$, all the coordinates converge, hence the subsequence is convergent in $\R^n$ (since $\|\cdot\|_2\leq \sqrt n \|\cdot\|_\infty$).
\end{proof}

%\bigskip
\subsect{Compactness via coverings}

We now introduce another notion of compactness.
It is characterised not in terms of sequences, but in terms
of the following concept of `coverings'.

\begin{definition}
Let $A\subset \R^n$; an \emph{open cover} of $A$ is any collection $\{U_\lambda: \lambda \in \Lambda\}$
of open sets of $\R^n$ such that
\[
A \subseteq \bigcup_{\lambda \in \Lambda} U_\lambda.
\]
An open cover is said to be \emph{finite} if $\Lambda$ is finite. 

If $M \subset \Lambda$ and
\[
A \subseteq \bigcup_{\mu \in M} U_\mu,
\]
then $\{U_\mu: \mu \in M\}$ is called a \emph{subcover} of $\{U_\lambda:\lambda \in \Lambda\}$.
\end{definition}

\np

\begin{remark}
The index set $\Lambda$ has no other purpose than to label the
sets $U_\lambda$, and we can replace it with any other set
of the same size. In particular, for a finite open cover,
we can always write $\{U_1, \ldots, U_N\}$ for some $N \in \N \cup \{0\}$.
\end{remark}

\begin{definition}
A set $A\subset \R^n$ is said to be \emph{compact} if every open cover of $A$ has a finite subcover. 
%A subset $Y \subset X$ is called \emph{compact} if the subspace $(Y,d_Y)$ is compact.
\end{definition}


We now connect the two notions of compactness.

\begin{theorem} \label{thm:compact_implies_sequential}
If a set $A\subset \R^n$ is compact, then it is sequentially compact.
\end{theorem}

\np

\begin{proof}
Assume for contradiction that there exists a sequence $(x_k)_{k \in \N}\subset A$ that has no convergent subsequence. This means that for every $y \in A$ there exists
an $r_y > 0$ such that
\[
B_{r_y}(y) \cap \{x_k: k \in \N\} 
\]
is a finite set. 
\nl
(Indeed, if there existed $y\in A$ such that every ball $B_{r}(y)$, with $r>0$ contained infinitely many elements of the sequence, 
it would be possible to construct a subsequence converging to $y$, against the assumption.)


Now the collection $\{B_{r_y}(y): y \in A\}$ is an open cover of $A$.
Since $A$ is compact, there exist $y_1, \ldots, y_N \in A$ with
\[
A \subseteq \bigcup_{i = 1}^N B_{r_{y_i}}(y_i).
\]

Hence
\[
\{x_k: k \in \N\} = A\cap \{x_k: k \in \N\} \subset \bigcup_{i = 1}^N B_{r_{y_i}}(y_i) \cap \{x_k: k \in \N\},%\subset \{y_1, \ldots, y_N\}.
\]
where the right-hand side is a finite set (finite union of finite sets). Since every finite set is bounded and hence
sequentially compact, and since we have proved that the sequence $\{x_k: k \in \N\}$ is in a finite set, we then conclude that $(x_k)_{k \in \N}$ does have a convergent subsequence, in contradiction to the assumption.
\end{proof}



\begin{lemma}[Closed subsets of a compact set]\label{lemma:closed}
Let  $n \in \N$, let $F\subset K\subset \R^n$, and assume that $F$ is closed and $K$ is a compact set. Then $F$ is compact.
\end{lemma}

\begin{proof}
Let $\{U_\lambda: \lambda \in \Lambda\}$ be an open cover of $F$. Since $F$ is closed, then $\R^n\setminus F$ is open, and hence 
$$
\{U_\lambda: \lambda \in \Lambda\}\cup (\R^n\setminus F)=: \Phi
$$
is an open cover of $K$. As $K$ is compact, we can extract a finite subcover $\Psi$ of $K$ which will also be a subcover of $F$. If the set $\R^n\setminus F$ is a member of $\Psi$, we may remove it from $\Phi$, and still have a finite cover of $F$, which is also a subcover of $\{U_\lambda: \lambda \in \Lambda\}$.
\end{proof}


\begin{theorem}[Tychonov's Theorem]\label{Tycho}
Let $n,m\in \N$. If $A\subset \R^n$ and $B\subset \R^m$ are compact, then the product 
$$
A\times B=\{(a,b): a\in A, b\in B\}
$$ 
is compact.
\end{theorem}

\begin{proof}
Let $\{U_{\lambda}: \lambda\in \Lambda\}$ be an open cover of $A\times B$. The idea is to kind of `project' the cover onto $A$ and $B$, which are compact, to extract a finite sub-cover for the `projected' sets, and then combine them to get a finite sub-cover of $A\times B$.

Let $(a,b)\in A\times B$; then there exists $\lambda_{(a,b)}\in \Lambda$ such that 
$(a,b) \in U_{\lambda_{(a,b)}}$. Since $U_{\lambda_{(a,b)}}$ is open, there exists a radius $r_{(a,b)}>0$ such that 
$$
(a,b) \in B_{r_{(a,b)}}(a,b)\subset U_{\lambda_{(a,b)}}.
$$
On the other hand, we can always put a `rectangle' in the ball $B_{r_{(a,b)}}(a,b)$; more precisely, there exist open sets $V_{(a,b)}\subset A$ and $W_{(a,b)}\subset B$ such that 
$$
(a,b) \in V_{(a,b)}\times W_{(a,b)} \subset   B_{r_{(a,b)}}(a,b)\subset U_{\lambda_{(a,b)}}.
$$
Clearly, for fixed $a\in A$, $\{W_{(a,b)}: b\in B\}$ is an open sub-cover of $B$. Since $B$ is compact, there exists a finite sub-cover
$$
\{ W_{(a,b_1(a))}, W_{(a,b_2(a))}, \dots, W_{(a,b_N(a))}\}
$$
of $B$. Now we do a similar thing for $A$. Note that for every $a\in A$, the set 
$$
V_a:= \bigcap_{j=1}^N V_{(a,b_j(a))}
$$
is open (finite intersection of open sets) and $a\in V_a$. So $\{V_a: a\in A\}$ is an open cover of $A$. Since $A$ is compact, there exists a finite sub-cover 
$$
\{V_{a_1}, V_{a_2}, \dots, V_{a_M}\}
$$
of $A$. 

\np

In conclusion, we have that 
$$
\{  V_{a_i} \times W_{(a_i,b_j(a_i))}: i=1,\dots, M, \quad j=1,\dots, N\}
$$
is an open and finite cover of $A\times B$. From this we can obtain a finite sub-cover of the original cover $\{U_{\lambda}: \lambda\in \Lambda\}$:
$$
\{U_{\lambda_{(a_i,b_j(a_i))}}:  i=1,\dots, M, \quad j=1,\dots, N\},
$$
which is a cover of $A\times B$ since 
$$
V_{a_i} \times W_{(a_i,b_j(a_i))} \subset V_{(a_i,b_j(a_i))} \times W_{(a_i,b_j(a_i))} \subset U_{\lambda_{(a_i,b_j(a_i))}}.
$$
Hence $A\times B$ is compact.
\end{proof}

%\bigskip
\subsect{Equivalent notions of compactness for $\R^n$}

%We note the  equivalence of the following statements.

\begin{theorem}[Heine-Borel] \label{thm:Heine-Borel}
Let $n \in \N$. Suppose that $A$ is a subset of the Euclidean
space $\R^n$. Then the following are equivalent.
\begin{enumerate}
\item $A$ is sequentially compact;
\item $A$ is closed and bounded;
\item $A$ is compact.
\end{enumerate}
\end{theorem}

\begin{proof}
We are going to prove %that 
$(1) \Rightarrow (2)$, %that 
$(2) \Rightarrow (3)$ and $(3) \Rightarrow (1)$.
\bis

$(1) \Rightarrow (2)$ \, Clearly a sequentially compact set is closed. Now, assume that $A$ is not bounded and let $x_1\in A$. Let now $r>0$; since $A$ is not bounded, there exists 
$x_2\in A\setminus B_r(x_1)$. Similarly there is $x_3\in A\setminus (B_r(x_1)\cup B_r(x_2))$ and so on. In this way be can build a sequence $(x_k)_{k\in \N}\subset A$ such that 
$|x_k-x_j|\geq r$ whenever $k\neq j$. This sequence clearly does not have any convergent subsequence, contradicting the assumption. Hence $A$ is bounded.

\bis

$(2) \Rightarrow (3)$\, Let $A$ be closed and bounded. Being $A$ bounded, there exists $R>0$ such that 
$A\subset [-R,R]^n$, where $[-R,R]^n$ denoted the $n$-dimensional cube. Being $A$ a closed subset of $[-R,R]^n$, by Lemma \ref{lemma:closed} we only need to prove that $[-R,R]^n$ is compact. This follows by $[-R,R]^n$ being the product of compact sets, by Theorem \ref{Tycho}.
%
%
%, and let $\{U_\lambda: \lambda \in \Lambda\}$ be an open covering of $A$ with no finite subcover. 
%Being $A$ bounded, there exists $r>0$ such that 
%$A\subset B_r(0)$. Since {\color{red} TO BE COMPLETED}

\bis

$(3) \Rightarrow (1)$\, This is exactly Theorem \ref{thm:compact_implies_sequential}. 
\end{proof}




\begin{remark}
It may seem silly to have different names for concepts that are equivalent. However, while they are equivalent for $\R^n$ 
\nl
(and actually (1) and (3) remain equivalent for metric spaces),
\nl
these notions of compactness can also be defined for more general topological spaces, and then they are no longer equivalent.
\end{remark}

\np

\begin{lemma}\label{lemma:infinite}
Let $K\subset \R^n$ be a compact set, and let $E\subset K$ be an infinite set. Then $E$ has a cluster point in $K$.
\end{lemma}

\begin{proof}
Assume that no points in $K$ are cluster points of $E$. Then for every $y\in K$ there exists $r_y$ such that 
$$
(B_{r_y}(y)\setminus \{y\})\cap E = \emptyset.$$

Since $\{B_{r_y}(y): y\in K\}$ is an open cover of $K$, there exists 
%$N\in\N$ and
$$y_1, \dots, y_N \in K
\quad\text{such that}\quad
K\subset \bigcup_{k=1}^N B_{r_{y_k}}(y_k).
$$
But $E\subset K$  implies that 
\slinc{}{\vspace*{-2ex}}
$$
E = E\cap K\subset E\cap \bigcup_{k=1}^{N
} B_{r_{y_k}}(y_k) \subset \{y_1,\dots,y_N\},
$$
contradicting the fact that $E$ is an infinite set.
\end{proof}


\begin{theorem}[Weierstrass]
Every infinite and bounded set $A\subset R^n$ has a cluster point.
\end{theorem}

\bis

\begin{proof}
$A$ being bounded, there exists $R>0$ such that $A\subset [-R,R]^n$, where $[-R,R]^n$ denoted 
the $n$-dimensional cube, which is compact by Theorem \ref{Tycho}. By Lemma \ref{lemma:infinite}, the compact set $[-R,R]^n$ 
contains an infinite set, which has then a cluster point in $[-R,R]^n$, hence in $\R^n$.
\end{proof}

%\bis
\subsect{Families of compact sets}

\begin{theorem}
Let $(K_\lambda)_{\lambda\in \Lambda}$ be a family of compact sets in $\R^n$ such that the intersection of every finite sub-family is nonempty. Then the intersection 
$$
\bigcap_{\lambda\in \Lambda} K_\lambda
$$
is nonempty.
\end{theorem}

\bis

\begin{proof}
Fix an element $K_{\lambda^*}$ of the family and assume, for contradiction, that no point of $K_{\lambda^*}$ belongs to every $K_{\lambda}$. 

Then the family $\{U_\lambda: \lambda\in \Lambda\}$, for $U_\lambda:= \R^n\setminus K_\lambda$, is an open cover of $K_{\lambda^*}$. Since $K_{\lambda^*}$ 
is compact, it has a finite sub-cover $\{U_{\lambda_1}, \dots, U_{\lambda_N}\}$.

This however implies that 
$$
K_{\lambda_1}\cap \dots \cap K_{\lambda_N}\cap K_{\lambda^*} = \emptyset,
$$
contradicting the assumption.
\end{proof}

A useful corollary is the following:


\begin{corollary}
Let $(K_i)_{i\in \N}$ be a sequence of nonempty compact sets in $\R^n$ such that $K_i\supseteq K_{i+1}$ for every $i\in \N$. Then the intersection 
$$
\bigcap_{i=1}^\infty K_i
$$
is nonempty.
\end{corollary}
\bis

The compactness of the sets is crucial for this result to hold as these non-compact \h{counter}-examples will show:
\bis

\begin{example}
For unbounded $U_i:=[i,\infty)$ follows $\bigcap_{i=1}^\infty U_i=\emptyset.$
\end{example}

\begin{example}
For open but bounded $U_i:=(0,\frac1i))$ follows $\bigcap_{i=1}^\infty U_i=\emptyset.$
\end{example}






\end{document}
