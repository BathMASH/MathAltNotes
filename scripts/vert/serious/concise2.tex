% -*- tex:uk -*-
% Course  MA20013
%
% uses concisenotes.tex
%


\sect{Metric spaces}
\gdef\fpdf{./frame/slnotesfigsanno2}  % needs \gdef to act inside the section, defines file for figs, start on pag. 4

%%%%%%%%%%%%%%%%%%%%%%%%%%%%%%%%%%%%%%%%%%%%%%%%%%%%


Many mathematical problems involve a notion of distance. In this chapter we
are going to develop a theory of `distance' from an abstract point of view. This relies
on the notion of a metric, which is an abstraction of distance between two
points of a given set.

\begin{definition}
Let $X$ be a set. A function $d : X \times X \to \R$ is called
a \emph{metric} if it satisfies the following properties:
\begin{enumerate}
\item \label{def:non-negative} $d(x,y) \ge 0$ for all $x,y \in X$,
\item \label{def:positive} $d(x,y) = 0$ if, and only if, $x = y$,
\item \label{def:symmetry} $d(x,y) = d(y,x)$ for all $x,y \in X$ (symmetry),
\item \label{def:triangle} $d(x,z) \le d(x,y) + d(y,z)$ for all $x,y,z \in X$ (triangle inequality).
\end{enumerate}
The pair $(X,d)$ is then called a \emph{metric space}.
\end{definition}

Sometimes $d$ is called `distance function' rather than `metric'. Often you can
find statements such as `$X$ is a metric space'; but this makes only sense if
it is clear what the corresponding metric is. 
%Some authors require that $X$ is non-empty in the definition of a metric space.

\begin{example}[Euclidean space] \label{Eucl:d2}
Let $n \in \N$; we define $d_2 : \R^n \times \R^n \to \R$ as
\[
d_2(x,y):= \|x - y\|_2 = \left(\sum_{i = 1}^n (x_i - y_i)^2\right)^{1/2}, \quad x,y \in \R^n.
\]
If $n=1$ we use the notation $d_2(x,y) = |x-y|$. We often omit the subscript $2$ from the Euclidean distance $\|x-y\|_2$.

We prove that $d_2$ is a metric and thus $(\R^n,d_2)$ is a metric space. Axioms \eqref{def:non-negative}--\eqref{def:symmetry}
are easy to verify. For the triangle inequality, we first need the \textbf{Cauchy-Schwarz inequality} (see Algebra 2A MA20216):
\begin{equation}\label{CS}
 \left|\sum_{i=1}^n u_i v_i \right| \leq \left(\sum_{i=1}^n u_i^2\right)^{\frac12} \left(\sum_{i=1}^n v_i^2\right)^{\frac12} \quad (=\|u\|\, \|v\|).
\end{equation}
To prove \eqref{CS} it is sufficient to prove the inequality for $u_i, v_i\geq 0$ and for $u\neq 0$ and $v\neq 0$ (otherwise it is trivial).
For every $\alpha,\beta>0$, we have
$$
0\leq \sum_{i=1}^n (\alpha u_i - \beta v_i)^2 = \alpha^2 \sum_{i=1}^n u_i^2 + \beta^2 \sum_{i=1}^n v_i^2 - 2\alpha\beta \sum_{i=1}^n u_i v_i 
$$
or equivalently, rearranging the terms,
$$
2\alpha\beta \sum_{i=1}^n u_i v_i  \leq \alpha^2 \sum_{i=1}^n u_i^2 + \beta^2 \sum_{i=1}^n v_i^2.
$$
The inequality \eqref{CS} follows by choosing $\alpha = \left(\sum_{i=1}^n v_i^2\right)^{\frac12}$ and $\beta = \left(\sum_{i=1}^n u_i^2\right)^{\frac12}$, 
and by dividing the resulting estimate by $2\alpha\beta$ (note that $2\alpha\beta> 0$).

\medskip

Now we are ready to prove the triangle inequality. By the definition of the Euclidean norm, proving \eqref{def:triangle} means to show that 
$$
\|x-z\| \leq \|x-y\| + \|y-z\| \quad \textrm{for all } \, x,y,z\in \R^n.
$$
Let $u:=x-y$ and $v:=y-z$; then $u+v=x-z$. We are going to prove (equivalently) that $\|u+v\| \leq \|u\| + \|v\|$. 

To this end, note that by expanding the square and by using the Cauchy-Schwarz inequality, we have 
\begin{align*}
\|u+v\|^2 = \sum_{i=1}^n  (u_i + v_i)^2 &= \sum_{i=1}^n  u_i^2 + \sum_{i=1}^n  v_i^2 + 2\sum_{i=1}^n u_i v_i 
\\&\leq 
\|u\|^2 + \|v\|^2 + 2 \|u\|\, \|v\| = (\|u\|+\|v\|)^2,
\end{align*}
and the claim follows by taking the square root.
\end{example}

\begin{example}
The following are also metrics on $\R^n$:
\begin{align*}
d_1(x,y):= \|x-y\|_1 = \sum_{i = 1}^n |x_i - y_i|,
\end{align*}
also called the $\ell^1$ metric (also referred to informally as the ``taxi-cab" metric, since it's the distance one would travel by taxi on a rectangular grid of streets);
\begin{align*}
d_\infty(x,y):= \|x-y\|_\infty = \max_{i \in \{1,\ldots,n\}} |x_i - y_i|,
\end{align*}
also called the $\ell^\infty$ metric, or the maximum metric.
More generally, for $1 \le p < \infty$, we have a metric
\[
d_p(x,y):= \|x-y\|_p = \left(\sum_{i = 1}^n |x_i - y_i|^p\right)^{1/p}
\]
on $\R^n$, but this is more difficult to verify (the triangle inequality does not follow from the Cauchy-Schwarz inequality, but from the Minkowski inequality). 
For $p < 1$, $d_p$ does not define a metric since the triangle inequality fails! E.g., in $\R^2$, for $p=\frac12$, and $x=(1,0)$, $z=(0,-1)$, $y=(0,0)$, we have 
$$
\|x-z\|_{\frac12} = \|(1,1)\|_{\frac12} = (1+1)^2 = 4; \quad \|x\|_{\frac12} + \|z\|_{\frac12} = 2.
$$
\end{example}

\np

\begin{example}[Discrete metric] \label{discrete-metric}
Let $X \not= \emptyset$ be a set. Define 
$$
d(x,y) = \begin{cases} 
0 & \text{if } x=y, \\ 
1 & \text{if } x\neq y. 
\end{cases}
$$
Then \eqref{def:non-negative}--\eqref{def:symmetry} are clearly satisfied. To
verify \eqref{def:triangle}, we just need to check four different cases, and it turns
out that $d$ is a metric.
\end{example}

\begin{example}[Railway metric]\label{railway}
Define $d : \R^n \times \R^n \to \R$ as follows: for $x,y \in \R^n$,
\[
d(x,y) = \begin{cases} \|x - y\| & \text{if $x,y,0$ are collinear}, \\ \|x\| + \|y\| & \text{otherwise}. \end{cases}
\]
E.g., for $n = 2$,
\begin{align*}
d((1,1),(2,2)) & = \sqrt{2}, \\
d((1,1),(-2,2)) & = 3\sqrt{2}.
\end{align*}
(\textit{Curiosity:} It is called the Railway metric in Britain (and in France), because all the train lines radiate from London (and Paris), located at the origin. To take a train from town $x$ to town $y$, one has to take a train from $x$ to $0$ and then take a train from $0$ to $y$, unless $x$ and $y$ are on the same line, when one can take a direct train.)

\medskip


%We want to verify that $d$ is a metric. Axioms \eqref{def:non-negative} and \eqref{def:symmetry} are obvious.
%
%\eqref{def:positive} If $x = y$, then $x,y,0$ are collinear and $d(x,y) = \|x - y\| = 0$. If $d(x,y) = 0$, then
%either
%\begin{itemize}
%\item $x$, $y$, and $0$ are collinear and $\|x - y\| = 0$, or
%\item $\|x\| = 0$ and $\|y\| = 0$.
%\end{itemize}
%In the first case, we have $x = y$. In the second case, we have $x = 0 = y$.
%
%\eqref{def:triangle} Let $x,y,z \in \R^n$. We have to distinguish a number of cases in order to prove the triangle
%inequality. If $x,y,z,0$ are all collinear, then
%\[
%d(x,z) = \|x - z\| \le \|x - y\| + \|y - z\| = d(x,y) + d(y,z).
%\]
%If $x,y,0$ are collinear and $y,z,0$ are not, then $x,z,0$ are not collinear. Hence
%\[
%d(x,z) = \|x\| + \|z\| \le \|x - y\| + \|y\| + \|z\| = d(x,y) + d(y,z).
%\]
%When the roles of $x$ and $z$ are reversed, the proof remains the same. Finally, if neither $x,y,0$ nor $y,z,0$ nor $x,z,0$ are
%collinear, then
%\[
%d(x,z) \le \|x\| + \|z\| \le \|x\| + 2\|y\| + \|z\| = d(x,y) + d(y,z).
%\]
\end{example}



\begin{example}\label{continuous_functions}
Consider the space $C^0([0,1])$ of all continuous functions from $[0,1]$ to $\R$ (sometimes it is denoted with $C([0,1])$).
Define
\[
d_\infty(f,g) = \max_{t \in [0,1]} |f(t) - g(t)|, \quad f,g \in C^0([0,1]).
\]
Then $(C^0([0,1]),d_\infty)$ is a metric space. 

\quad 

Another possible metric on $C^0([0,1])$ can be defined as 
$$
d^\ast(f,g) = \int_0^1|f(t)-g(t)|dt.
$$

\end{example}


In many of the examples above the distance can be defined in terms of a norm. This is the case for the Euclidean space, but not only. 

\subsect{Normed vector spaces}

We start by recalling the definition of vector spaces.
\begin{definition}
A \h{real vector space} is a nonempty set $X$ on which two operations are defined, addition in $X$ 
and multiplication by an element in $\R$ 
\slinc{}{\vspace*{-1ex}}
$$
+ : X\times X \to X
%$$
%and multiplication by an element in $\R$ 
%$$
,\qquad
\cdot : \R\times X \to X
$$
satisfying the following axioms:
%
\begin{itemize}
\item The addition is associative and commutative, namely
\slinc{}{\vspace*{-1ex}}
$$
x+y = y+x, \quad (x+y)+z = x+y+z \quad \forall \, x,y,z \in X;
$$
\item There exists an element $0\in X$ such that $x+0=x$ for every $x\in X$;
\item $0\cdot x = 0$ and $1\cdot x= x$ for every $x\in X$;
\item Addition and multiplication are distributive, namely
\slinc{}{\vspace*{-1ex}}
$$
(\lambda+\mu)x = \lambda x+ \mu x, \quad \lambda(x+y) = \lambda x+ \lambda y \quad \forall \, x,y \in X, \lambda,\mu\in \R;
$$
\item For every $\lambda, \mu \in \R$ and every $x\in X$:
$
\lambda(\mu x) = (\lambda \mu) x.
$
\end{itemize}
\end{definition}

The vector space is called complex if the multiplication is defined with elements in $\C$.
%\end{definition}

On vector spaces one can define a \emph{norm}, which is essentially a way to measure the magnitude of the elements in the vector space.


\begin{definition}[Normed space]
Let $X$ be a (real) vector space. An application $\|\cdot\|: X\to \R$ is called a norm on $X$ if it satisfies the following axioms:
\begin{itemize}
\item[(N1)] $\|x\|\geq0$ and $\|x\|=0$ if and only $x=0$;
\item[(N2)] $\|\lambda x\| = |\lambda| \|x\|$ for every $x\in X$ and every $\lambda\in \R$;
\item[(N3)] $\|x+y\|\leq \|x\| + \|y\|$.
\end{itemize}
The pair $(X,\|\cdot\|)$ is called a normed space. 
\end{definition}

\np

\begin{remark}
The properties in the definition of the norm are the natural ones to require of a length: The length of $x$ is $0$ if and only if $x$ is the $0$-vector; multiplying a vector by $\lambda$ multiplies its length by $|\lambda|$; and the length of the side $x + y$ (in a triangle) is less than or equal to the sum of the lengths of the sides $x$, $y$. Because of this last interpretation, property (N3) is referred to as the triangle inequality (as for the distance).
\end{remark}

\np

\begin{example}[Norms in $\R^n$]
$(\R,|\cdot|)$ is a normed space, where $|\cdot|$ denotes the absolute value. Also $(\R^n,\|\cdot\|)$ is a normed space, where $\|\cdot\|$ denotes
$$
\|x\|:= \sqrt{\langle x,x\rangle} = \sqrt{\sum_{i=1}^n x_i^2}.
$$
(We will often write $\|\cdot\|_2$ for the Euclidean norm.)

\medskip

The set $\R^n$ with any of the norms defined for $x = (x_1,\dots, x_n)$ by 
$$
\|x\|_1 := \sum_{i=1}^n|x_i|, \quad \|x\|_{\infty} := \max_{i=1,\dots,n}\{ |x_i|\}
$$
is an $n$-dimensional normed vector space. The corresponding metrics are the ``taxi-cab'' metric and the maximum metric, respectively.

\end{example}

\np

\begin{example}[Norms on spaces of functions]\label{continuous_functions-N}
The distance in Example \ref{continuous_functions} can be defined via the norm given, for $f \in C^0([0,1])$, by 
$$
\|f\|:= \max_{t \in [0,1]} |f(t)|.
$$
This norm is usually denoted with $\|f\|_{\infty}$ and referred to as the infinity-norm or sup-norm. (We will see a connection between this norm and uniform convergence.)
%\medskip

\noindent
In $C^0([0,1])$ one can also define $L^p$-norms, for $p\geq 1$, as
$$
\|f\|_p:= \left(\int_0^1|f(t)|^p dt\right)^{\frac1p},
$$
for $f\in C^0([0,1])$. Then $(C^0([0,1]), \|\cdot\|_p)$ is a normed space.

%\medskip

Another example of a normed space is the space $C^1([0,1])$ of continuously differentiable functions on $[0,1]$%
\slinc{\footnote{Usually differentiability is defined in an open interval. Here differentiable in the closed interval $[0,1]$ means that it is differentiable in $(0,1)$ and the derivative can be extended up to $[0,1]$ continuously.}}{}%
, with the norm defined as 
$$
\|f\|_{C^1}:= \max_{t \in [0,1]} |f(t)| + \max_{t \in [0,1]} |f'(t)| = \|f\|_{\infty}+\|f'\|_{\infty}.
$$

\end{example}


\begin{remark}
A \h{normed space} $(X,\|\cdot\|)$ is a metric space, with the distance
$$
d(x,y):= \|x-y\|, \quad x,y\in X.
$$
(The triangle inequality comes from the axiom (N3) of the norm.)

\medskip

The viceversa is not true! A metric associated with a norm is special: It has the additional properties that for all $x, y, z \in X$ and $\lambda\in \R$
$$d(x + z,y + z) = d(x,y), \quad d(\lambda x,\lambda y) = |\lambda| \, d(x,y),$$
which are called translation invariance and homogeneity, respectively. These properties do not even make sense in a general metric space since we cannot add points or multiply them by scalars. If $X$ is a normed vector space, we always use the metric associated with its norm, unless stated specifically otherwise.
\end{remark}

\np

\begin{example}[Metrics not associated to a norm]
Consider the discrete metric in Example \ref{discrete-metric} and take $X=\R$ (or any vector space on $\R$ containing at least two distinct elements). Then, for any $\lambda\in \R$, with $\lambda\neq 0$ and $|\lambda|\neq 1$, and any $x,y \in \R$ with $x\neq y$, we have that 
$$
d(\lambda x, \lambda y) =1,
$$
while 
$$
|\lambda| d(x,y) = |\lambda| \neq 1.
$$
\medskip

Also the railway metric in Example \ref{railway} is not derived from a norm (prove it!).

\end{example}

\subsect{Open and Closed Sets}

We now extend to metric spaces the definitions of open and closed sets that we learnt in $\R$.



\begin{definition} \label{def:ball}
Let $(X,d)$ be a metric space. Let $x_0 \in X$ and $r > 0$. Then
\[
B_r(x_0) = \{x \in X: d(x_0,x) < r\}
\]
is called the \emph{open ball} of centre $x_0$ and radius $r$, whereas
\[
\overline{B}_r(x_0) = \{x \in X: d(x_0,x) \leq r\}
\]
is called the \emph{closed ball} of centre $x_0$ and radius $r$.
\end{definition}

\begin{example}
In $(\R^n, d_2)$, namely in the Euclidean space, balls are the usual shapes. (Note that, for $n=1$, an open (resp. closed) ball is an open (resp. closed) interval centred at $x_0$ and with length $2r$.)
\end{example}

\np

\begin{example}
Consider $\R^2$ with the metric
\[
d_1(x,y) = |x_1 - y_1| + |x_2 - y_2|.
\]
Then $B_1(0)$ has the shape shown in Figure \ref{fig:ball1}.
\begin{figure}[ht]
\begin{center}
\includegraphics[width=0.3\textwidth]{ball1.pdf}
\caption{The unit ball $B_1(0)$ in the metric $d_1$ (or in the norm $\|\cdot\|_{1}$)}
\label{fig:ball1}
\end{center}
\end{figure}
\end{example}

\begin{example}
Consider $\R^2$ with the metric
\[
d_\infty(x,y) = \max\{|x_1 - y_1|,|x_2 - y_2|\}.
\]
Then $B_1(0)$ has the shape shown in Figure \ref{fig:ballinfty}.
\begin{figure}[ht]
\begin{center}
\includegraphics[width=0.3\textwidth]{ballinfty.pdf}
\caption{The unit ball $B_1(0)$ in the metric $d_\infty$ (or in the norm $\|\cdot\|_{\infty}$)}
\label{fig:ballinfty}
\end{center}
\end{figure}
\end{example}

\begin{example}
Consider $C^0([0,1])$ with the usual metric
\[
d_\infty(f,g) = \max_{t \in [0,1]} |f(t) - g(t)| \quad (=\|f-g\|_\infty)
\]
defined in Example \ref{continuous_functions}.

Let $f_0 \in C^0([0,1])$. A graphical representation of the ball $B_r(f_0)$ is given in Figure \ref{fig:ballC0}.
\begin{figure}[ht]
\begin{center}
\includegraphics[width=0.45\textwidth]{ballC0.pdf}
\caption{The thick solid curve represents the graph of a function $f_0 \in C^0([0,1])$. The open ball $B_r(f_0)$ is given by all functions with graphs contained in the grey region (not including the upper and lower boundary).}
\label{fig:ballC0}
\end{center}
\end{figure}
\end{example}


We can then define what it means for a set to be bounded.



\begin{definition}[Bounded sets]
Let $(X,d)$ be a metric space. A set $A\subseteq X$ is said to be \emph{bounded} if there exist $x_0\in X$ and $r>0$ such that 
$$
A\subseteq B_r(x_0).
$$ 

We say that $(X,d)$ is a \emph{bounded metric space} if $X$ is bounded.
\end{definition}


\begin{remark}
Alternatively, one can say that $A\subseteq X$ is bounded if
\[
\textrm{diam} (A) := \sup_{x ,y \in A} d(x,y) <\infty,
\]
and $\textrm{diam} (A)$ is called the diameter of $A$.

\end{remark}

\np

\begin{example}
The Euclidean space $\R^n$ is unbounded, but the open unit ball
$\{x \in \R^n: \|x\| < 1\}$
in $\R^n$ is bounded (by definition) with diameter $2$.
\end{example}

\bis

\begin{example}
On $\R$, define
\[
d(x,y) = |\arctan x - \arctan y|.
\]
Then $(\R,d)$ is a metric space. As $|\arctan x| \le \frac{\pi}{2}$ for every
$x \in \R$, it is bounded. In fact, we have $\textrm{diam} (\R) = \pi$ in this metric.
\end{example}

\np

%\begin{definition}
%Let $(X,d)$ be a metric space and $x_0 \in X$. A set $N \subseteq X$ is a
%\emph{neighbourhood} of $x_0$ if there exists a radius $r > 0$ such that $B_r(x_0) \subseteq N$.
%\end{definition}

%\bis

\begin{definition}[Open sets]
Let $(X,d)$ be a metric space. A set $U \subseteq X$ is \emph{open} if
\[
\forall \, x \in U \quad \exists \, r > 0 : B_r(x) \subseteq U.
\]
\end{definition}

\bis

\begin{lemma} \label{lemma:balls}
Let $(X,d)$ be a metric space. Let $x_0 \in X$ and $r > 0$. Then $B_r(x_0)$ is open.
\end{lemma}

\begin{proof}
Let $x \in B_r(x_0)$. Set $s = r - d(x,x_0)$. Then $s > 0$. Moreover, we have $B_s(x) \subseteq B_r(x_0)$. It follows that $B_r(x_0)$ is open.
\end{proof}

\np

\begin{theorem}[Properties of open sets] \label{thm:open}
Let $(X,d)$ be a metric space. Then
\begin{enumerate}
\item \label{item1.2.1.i} $\emptyset$ and $X$ are open;
\item \label{item1.2.1.ii} if $\{U_\lambda: \lambda \in \Lambda\}$ is any collection of open sets, then the union
\[
\bigcup_{\lambda \in \Lambda} U_\lambda
\]
is open;
\item \label{item1.2.1.iii} if $\{U_1, \ldots, U_N\}$ is a finite collection of open sets, then the intersection
\[
\bigcap_{k = 1}^N U_k
\]
is open.
\end{enumerate}
\end{theorem}

\begin{proof}
\eqref{item1.2.1.i} Clearly $\emptyset$ is open. Moreover, for every $x \in X$, we have
$B_1(x)=\{y \in X: d(x,y) < 1\} \subseteq X$, so $X$ is open.
\medskip

\eqref{item1.2.1.ii} Let $x \in \bigcup_{\lambda \in \Lambda} U_\lambda$. Then there exists an $\lambda^\ast \in \Lambda$ such
that $x \in U_{\lambda^\ast}$. As $U_{\lambda^\ast}$ is open, there exists an $r > 0$ such that $B_r(x) \subseteq U_{\lambda^\ast}$, and
then
\[
B_r(x) \subseteq \bigcup_{\lambda \in \Lambda} U_\lambda.
\]
\medskip

\eqref{item1.2.1.iii} Let $x \in \bigcap_{k = 1}^N U_k$. Since every $U_k$ is open, there exist
$r_1,\ldots,r_N > 0$ with
\[
B_{r_1}(x) \subseteq U_1, \ldots, B_{r_N}(x) \subseteq U_N.
\]
Define $r = \min\{r_1,\ldots,r_N\}$. Then
\[
B_r(x) \subseteq B_{r_k}(x) \subseteq U_k, \quad k = 1,\ldots,N,
\]
and thus $B_r(x) \subseteq \bigcap_{k = 1}^N U_k$.
\end{proof}

\np

\begin{definition}[Interior]
Let $A$ be a subset of a metric space $(X,d)$. A point $x \in A$ is called \emph{interior point} 
of $A$ if there exists an $r > 0$ such that $B_r(x) \subseteq A$.
The set of all interior points of $A$ is denoted by $A^\circ$ and called the
\emph{interior} of $A$.
\end{definition}

\begin{example}
Consider the set $[0,1] \subset \R$. Then $[0,1]^\circ = (0,1)$.
\end{example}


\begin{theorem} \label{thm:interior}
Let $A$ be a subset of a metric space $(X,d)$. Then
\begin{enumerate}
\item \label{item1.2.2.i} $A^\circ$ is open, and
\item \label{item1.2.2.ii} if $U \subseteq A$ is open, then $U \subseteq A^\circ$.
\end{enumerate}
\end{theorem}

We can summarise this theorem as follows: the interior of $A$ is the largest
open set contained in $A$.

\np

\begin{proof}
\eqref{item1.2.2.ii} Let $U \subseteq A$ be open. If $x \in U$ is an
arbitrary point, then there exists an $r > 0$ such that $B_r(x) \subseteq U \subseteq A$.
So $x$ is an interior point, and hence $x \in A^\circ$.

\eqref{item1.2.2.i} Let $x \in A^\circ$. Then there exists an $r > 0$ such that
$B_r(x) \subseteq A$. By Lemma \ref{lemma:balls}, we know that $B_r(x)$ is open.
By \eqref{item1.2.2.ii}, we have $B_r(x) \subseteq A^\circ$. Therefore $A^\circ$ is open.
\end{proof}


\begin{remark}
Using this notation, we can reformulate the definition of open sets: a set
$U \subseteq X$ is open if, and only if, each point of $U$ is an interior point.
That is,
\[
U \mbox{ is open} \Leftrightarrow U = U^\circ.
\]
\end{remark}

%%%%%%%%%
%%%%%%%%%  Fin qui


%\bigskip


\np

\begin{definition}
Let $(X,d)$ be a metric space and $A \subseteq X$. A point $x_0 \in X$ is
called a \emph{cluster point} (or an `accumulation point', or a `limit point') of $A$, if for all
$r > 0$,
\[
(B_r(x_0) \backslash \{x_0\}) \cap A \not= \emptyset.
\]
The set of all cluster points of $A$ is denoted by $A'$.

%\medskip
%\bis

A point $x_0 \in A$ is called an \emph{isolated point} of $A$, if $x\in A$ and there exists $r > 0$ such that
\[
(B_r(x_0) \backslash \{x_0\}) \cap A  = \emptyset.
\]

\end{definition}

In other words, the point $x_0$ is a cluster point of $A$ if it has points
of $A$ other than itself arbitrarily close. It is an isolated point if it is in $A$ but is not a cluster point of $A$.

Note that a cluster point may be contained in $A$ or not, while an isolated point is contained in $A$.




\begin{example}
In $(\R, |\cdot|)$, consider the set $A = \left\{\frac{1}{k}: k \in \N\right\}$. Then $0 \in A'$ (even if $0\notin A$), whereas $1 \not\in A'$ (even if $1\in A$).
\end{example}

\begin{definition}[Closed sets]
A subset of a metric space is called \emph{closed} if it contains
all of its cluster points.
\end{definition}

\begin{example}
The set $A$ from the previous example is not closed, as $0 \not\in A$.
On the other hand, the set $F = \left\{\frac{1}{k}: k \in \N\right\} \cup \{0\}$ is closed in $\R$.
There is only one cluster point, namely $0$. All the elements of $A$ are isolated points.
\end{example}


\begin{theorem} \label{thm:complement}
Let $(X,d)$ be a metric space. A set $F \subseteq X$ is closed if, and only if, its complement
$X \backslash F$ is open.
\end{theorem}

\np

\begin{proof}
Suppose that $F$ is closed. Let $x \in X \backslash F$. Then $x$ is not a cluster point of $F$.
That is, there exists an $r > 0$ such that
\[
(B_r(x) \backslash \{x\}) \cap F = \emptyset.
\]
Since $x \not\in F$, this means $B_r(x) \subseteq X \backslash F$. It follows that
$X \backslash F$ is open.

Now suppose that $F$ is not closed. Then there exists a cluster point $x$ of $F$, with 
$x \in X \backslash F$. Then for any $r > 0$, we have $B_r(x) \cap F \not= \emptyset$,
so $B_r(x) \not\subseteq X \backslash F$. This means that $x$ is not an interior point
of $X \backslash F$, and $X \backslash F$ is not open.
\end{proof}

%\bis

\begin{example}
Let $a < b$. The interval $[a,b]$ is closed in $\R$ (because $(-\infty,a) \cup (b,\infty)$ is open),
but $(a,b]$, $[a,b)$, and $(a,b)$ are not closed.
\end{example}


\begin{example}
The closed ball $\overline{B}_r(x)$ is closed in any metric space $(X,d)$ for any $x \in X$ and any $r > 0$.
\end{example}

\begin{example}
In any metric space $(X,d)$, the sets $\emptyset$ and $X$ are closed.
\end{example}

\begin{definition}[Closure]
Let $A$ be a subset of a metric space $(X,d)$. The \emph{closure} of $A$
is the set $\overline{A} = A \cup A'$.
\end{definition}

\begin{theorem} \label{thm:closure}
Let $A$ be a subset of a metric space $(X,d)$. Then
\begin{enumerate}
\item \label{item1.3.2.i} $\overline{A}$ is closed, and
\item \label{item1.3.2.ii} if $F \supseteq A$ is closed, then $F \supseteq \overline{A}$.
\end{enumerate}
\end{theorem}

In other words, the closure $\overline{A}$ is the smallest closed set containing $A$, and 
$$
A \mbox{ is closed} \Leftrightarrow A = \overline A.
$$

\np

\begin{proof}
\eqref{item1.3.2.ii} Suppose that $F \supseteq A$ is closed. Let
$x \in \overline{A} = A \cup A'$. If $x \in A$, then it is
clear that $x \in F$ as well. If $x \in A'$, then for every
$r > 0$, we have
\[
\emptyset \not= (B_r(x) \backslash \{x\}) \cap A \subseteq (B_r(x) \backslash \{x\}) \cap F.
\]
Hence $x$ is a cluster point of $F$, too. Since $F$ is closed,
this means that $x \in F$ and $\overline{A} \subseteq F$.

\bis

\eqref{item1.3.2.i} We want to show that $X \backslash \overline{A}$ is open, and the
claim then follows from Theorem \ref{thm:complement}.

Let $x \in X \backslash \overline{A}$. As $x$ is not a cluster point of $A$,
there exists a radius $r > 0$ such that $B_r(x) \subseteq X \backslash A$.
We know that $B_r(x)$ is open (Lemma \ref{lemma:balls}); therefore  $X \backslash B_r(x)$
is closed. By \eqref{item1.3.2.ii}, we have $\overline{A} \subseteq X \backslash B_r(x)$,
i.e., $B_r(x) \subseteq X \backslash \overline{A}$. Therefore $X \backslash \overline{A}$
is open, as required.
\end{proof}

\np

\begin{theorem}
Let $(X,d)$ be a metric space. Then
\begin{itemize}
\item $\emptyset$ and $X$ are closed;
\item if $\{F_\lambda: \lambda \in \Lambda\}$ is any collection
of closed sets, then
\[
\bigcap_{\lambda \in \Lambda} F_\lambda
\]
is closed; and
\item if $F_1, \ldots, F_N$ is a finite collection of
closed sets, then
\[
\bigcup_{k = 1}^N F_k
\]
is closed.
\end{itemize}
\end{theorem}

\begin{proof}
Again this is easily reduced to a theorem about open sets (Theorem \ref{thm:open}), using Theorem \ref{thm:complement}, since 
$$
X\setminus \bigcap_{\lambda \in \Lambda} F_\lambda = \bigcup_{\lambda \in \Lambda} (X\setminus F_\lambda)
$$
and 
$$
X\setminus \bigcup_{k = 1}^N F_k = \bigcap_{k = 1}^N (X\setminus F_k).
$$
\end{proof}

\bis

\begin{remark}
Sets are note like doors! There are sets
which are both open and closed (e.g., the empty set in any metric space), and there
are sets which are neither (e.g., the interval $[0,1)$ in $\R$).
\end{remark}

\np

\begin{example}[Cantor]\label{C:set}
Consider $\R$ with the usual metric. Let $I_0 = [0,1]$ and define
\begin{align*}
I_1 & = \textstyle [0,\frac{1}{3}] \cup [\frac{2}{3},1] \\
I_2 & = \textstyle [0,\frac{1}{9}] \cup [\frac{2}{9},\frac{1}{3}] \cup [\frac{2}{3},\frac{7}{9}] \cup [\frac{8}{9},1], \\
I_2 & = \textstyle [0,\frac{1}{27}] \cup [\frac{2}{27},\frac{1}{9}] \cup \ldots \cup  [\frac{8}{9},\frac{25}{27}] \cup [\frac{26}{27},1], \\
& \hspace{6pt} \vdots
\end{align*}
(In each step, remove the middle third of every interval remaining, but keep the end points; see Figure \ref{fig:Cantor}.)
Let
\[
C = \bigcap_{k = 0}^\infty I_k.
\]
This is called the Cantor set. \index{Cantor set} Every $I_k$ is a finite union of closed sets, and therefore closed.
Hence $C$ is an (infinite) intersection of closed sets, which means that $C$ is closed.
\begin{figure}[ht]
\begin{center}
\includegraphics[width=0.6\textwidth]{Cantor.pdf}
\end{center}
\caption{The sets $I_0,\ldots,I_5$ in the construction of the Cantor set}
\label{fig:Cantor}
\end{figure}
\end{example}

\np

\begin{definition}[Boundary of a set]
Let $(X,d)$ be a metric space. The \emph{boundary} $\partial A$ of a set $A \subseteq X$ is defined as
$$
\partial A:= \overline A\cap \overline{(X\setminus A)}.
$$
\end{definition}

Namely the boundary of $A$ is the set of points $x\in X$ (not necessarily in $A$!!) such that every ball $B_r(x)$ contains points of $A$ and of $X\setminus A$.

\bis

\begin{example}
In $\R$, $\partial [a,b] = \partial (a,b) = \{a,b\}$, $\partial \mathbb{Q} = \R$. Moreover, $\partial C= C$, where $C$ is the Cantor set in Example \ref{C:set}. 
(Note that $C^\circ = \emptyset$.)

\end{example}

\bis

\begin{theorem}
Let $(X,d)$ be a metric space, $A\subset X$, and let $x\in X$. Then
$$
x\in \partial A \quad \Leftrightarrow \quad \forall \, r>0 \,\, \textrm{we have } \,\, B_r(x)\cap A\neq \emptyset \quad \textrm{and} \quad B_r(x)\cap (X\setminus A)\neq \emptyset.
$$
\end{theorem}

\np

\begin{proof}
``$\Rightarrow$": Let $x\in \partial A$. Either $x\in A$ or $x\in X\setminus A$. We assume that $x\in A$ and a similar argument works in the other case.

Since $x\in A$, then $x\notin X\setminus A$; but 
$$
x\in \partial A\subset \overline{X\setminus A} = (X\setminus A)\cup (X\setminus A)',
$$
hence $x\in (X\setminus A)'$. By definition, for every $r>0$, $(B_r(x)\setminus \{x\})\cap (X\setminus A)\neq \emptyset$; on the other hand, since 
$x\in A$, $B_r(x)\cap A\neq \emptyset$.

In conclusion, we have the claim.
%\medskip
\bis

``$\Leftarrow$": Let $x\in X$. Either $x\in A$ or $x\in X\setminus A$. We assume that $x\in A$ and a similar argument works in the other case.

Since $x\in A\subseteq \overline A$, clearly $x\in \overline A$.

On the other hand, $x\notin X\setminus A$, but for every $r>0$, $B_r(x) \cap (X\setminus A) \neq \emptyset$, which equivalently means that 
for every $r>0$, $(B_r(x)\setminus \{x\})\cap (X\setminus A)\neq \emptyset$, since $x\notin X\setminus A$. By definition, $x\in (X\setminus A)'$, 
and since $(X\setminus A)'\subseteq \overline{X\setminus A}$, we have the claim.
\end{proof}

\np

\begin{definition}[Dense sets]
Let $(X,d)$ be a metric space. A set $A \subseteq X$ is called \emph{dense} \index{dense}
if $\overline{A} = X$.
\end{definition}

\begin{example}
The set $\Q$ is dense in $\R$.
\end{example}

\bis

\begin{remark} A further generalisation beyond metric spaces are topological spaces, where a collection of open subsets -- satisfying Theorem 
\ref{thm:open} -- is defining the topology.  
\end{remark}


%
%%%%%%%%%%%%%%%%%%%   FIN QUI  %%%%%%%%%%%%
%


%\bigskip




\subsect{Sequences and Convergence}

We extend the concept of convergence of sequences to metric spaces.

\begin{definition}
Let $(X,d)$ be a metric space, let $(x_k)_{k \in \N}$ be a sequence in $X$ and let $x_0\in X$. The point
$x_0 \in X$ is said to be a \emph{limit} of the sequence if $d(x_k,x_0) \to 0$
in $\R$ as $k \to \infty$, namely if  
$$
\forall \, \varepsilon>0 \quad \exists \,N\in \N: k\geq N \Rightarrow d(x_k,x_0)<\varepsilon.
$$
If so, we say that the sequence \emph{converges} to
$x_0$ and we write $x_0 = \lim\limits_{k \to \infty} x_k$ or $x_k \to x_0$ (or $x_k \stackrel{d}{\to} x_0$ to denote the metric).
\end{definition}

\np

\begin{example}
In $(\R, |\cdot|)$, consider the sequence $(x_k)_{k \in \N}$
with $x_k = \frac{1}{k}$. Then $d(x_k,0) = \frac{1}{k}$, so $x_k$
converges to $0$.
\end{example}

\begin{example}
In $(X,|\cdot|)$, with $X = (0,1]$, let $x_k = \frac{1}{k}$. Then there is no point $x_0$ in $X$ such that
$d(x_k,x_0) \to 0$ as $k \to \infty$. The sequence $(x_k)_{k \in \N}$
is therefore not convergent in $X$.
\end{example}

\begin{example}
Let $X \not= \emptyset$ be an arbitrary set with the discrete metric
\[
d(x,y) = \begin{cases} 0 & \text{if $x = y$,} \\ 1 & \text{if $x \not= y$.} \end{cases}
\]
Then $x_k \to x_0$ if, and only if, there exists an $N \in \N$ such that $x_k = x_0$ for all
$k \ge N$.
\end{example}

\np

\begin{example}
Consider the space $C^0([0,1])$ with the usual metric
\[
d_\infty(f,g):= \|f-g\|_{\infty} = \max_{0 \le t \le 1} |f(t) - g(t)|
\]
defined in Example \ref{continuous_functions}. (Recall that this is a normed space, see Example \ref{continuous_functions-N}.)

A sequence $(f_k)_{k \in \N}$ converges to $f_0$ with respect to the metric $d_\infty$ if, and only if,
$$
\max_{0 \le t \le 1} |f_k(t) - f_0(t)| \to 0 \quad \textrm{as } k\to \infty,
$$
namely if
\[
\forall \varepsilon > 0 \quad \exists \,N \in \N \quad\textrm{such that } \, \forall \, k \ge N \quad \textrm{and } \, \forall \, t \in [0,1] : |f_k(t) - f_0(t)| < \varepsilon.
\]
But this is exactly the definition of uniform convergence of $f_k$ to $f$ in 
Definition 1.1%%%%%\ref{pu-conv}
! For this reason the metric $d_\infty$ is often referred to as the \textit{uniform convergence metric}, since convergence in $d_\infty$ is equivalent to uniform convergence.
\end{example}

\begin{example}
Consider the space $C^0([0,1])$ with the metric
\[
d_2(f,g) = \|f-g\|_{2} = \left(\int_0^1 |f(t) - g(t)|^2 dt\right)^\frac12.
\]
A sequence $(f_k)_{k \in \N}$ converges to $f_0$ with respect to the metric $d_2$ (commonly said in the $L^2$-norm) if, and only if,
\[
\left(\int_0^1 |f_k(t) - f_0(t)|^2 dt\right)^\frac12 \to 0 \quad \textrm{as } \, k\to \infty.
\]
\smallskip

Note that this norm can be defined on all functions $f: [0,1] \to \R$ with $f^2$ Riemann-integrable (continuity is not needed).

\end{example}

\np

\begin{theorem} \label{thm:uniqueness}
Let  $(X,d)$ be a metric space, let $x_0\in X$, and let $(x_k)_{k \in \N}$ be a sequence converging to $x_0$.
\begin{enumerate}
\item \label{item2.1.1.i} If $x_k \to y_0$ in $(X,d)$, then $x_0 = y_0$ (i.e., limits are unique).
\item \label{item2.1.1.iii} For every subsequence $(x_{k_j})_{j \in \N}$, there holds
%the convergence
$x_{k_j} \to x_0$.
\end{enumerate}
\end{theorem}

\begin{proof}
\eqref{item2.1.1.i} We have, by the triangle inequality,
\[
d(x_0,y_0) \le d(x_0,x_k) + d(y_0,x_k) \to 0
\]
as $k \to \infty$. Hence $d(x_0,y_0) = 0$, which means that $x_0 = y_0$.

\eqref{item2.1.1.iii} Recall: when we say that $(x_{k_j})_{j \in \N}$ is a subsequence of $(x_k)_{k \in \N}$,
this means that $(k_j)_{j\in \N}$ is a strictly increasing sequence of natural numbers. Let $\varepsilon > 0$.
As $x_k \to x_0$ as $k \to \infty$, there exists an $N \in \N$ such that $d(x_k,x_0) < \varepsilon$ for $k \ge N$.
Moreover, there is an $M \in \N$ such that $k_j \ge N$ for $j \ge M$. Therefore we also have $d(x_{k_j},x_0) < \varepsilon$
for $j \ge M$, which proves the convergence $x_{k_j} \to x_0$ as $j \to \infty$.
\end{proof}

\begin{theorem} \label{thm:limit_closed}
Let $(X,d)$ be a metric space and $A \subseteq X$.
\begin{enumerate}
\item \label{item2.1.2.i} A point $x_0 \in X$ belongs to $\overline{A}$ if,
and only if, there exists a sequence in $A$ converging (in $d$) to $x_0$.
\item \label{item2.1.2.ii} The set $A$ is closed if, and only if, for every sequence
in $A$ that converges in $X$, the limit belongs to $A$.
\item \label{item2.1.2.iii} The set $A$ is open if, and only if, for
every sequence $(x_k)_{k \in \N}$  converging to a limit in $A$,
there exists an $N \in \N$ such that $x_k \in A$ for all $k \ge N$.
\end{enumerate}
\end{theorem}

\np

\begin{proof}
\eqref{item2.1.2.i} Suppose that $x_0 \in \overline{A}= A \cup A'$. If $x_0 \in A$, then consider the sequence $(x_k)_{k \in \N}$ with
$x_k = x_0$ for every $k \in \N$. This is a sequence in $A$ converging to $x_0$. If $x_0 \in A'\setminus A$,
then for every $k \in \N$, we have
\[
(B_{1/k}(x_0)\setminus\{x_0\}) \cap A \not= \emptyset \quad \Leftrightarrow \quad B_{1/k}(x_0) \cap A \not= \emptyset.
\]
Now choose $x_k \in B_{1/k}(x_0) \cap A$. Then $d(x_k,x_0) \le \frac{1}{k} \to 0$ as $k \to \infty$.
Hence the sequence $(x_k)_{k \in \N}\subseteq A$ converges to $x_0$.

Conversely, suppose that $x_0 \in X$ and that there exists a sequence $(x_k)_{k \in \N}$ in $A$
with $x_k \to x_0$ as $k \to \infty$. If $x_0 \in A$, there is nothing to prove, since $A\subseteq \overline{A}$. Otherwise,
fix $r > 0$. Then there exists an $N \in \N$ such that $d(x_k,x_0) < r$ for $k \ge N$.
In particular, we have $x_N \in B_r(x_0) \cap A$. As $x_0 \not\in A$, this implies
\[
\left(B_r(x_0) \backslash \{x_0\}\right) \cap A \not= \emptyset.
\]
So in this case, we have $x_0 \in A' (\subseteq \overline A)$.

\np

\eqref{item2.1.2.ii} Suppose that $A$ is closed and let $(x_k)_{k \in \N}$ be a sequence
in $A$ with limit $x_0 \in X$. By \eqref{item2.1.2.i}, we have that $x_0 \in \overline{A} = A$.

Conversely, suppose that $A$ contains all limits of sequences in $A$. In order to show
that $A$ is closed, it suffices to prove $\overline{A} \subseteq A$. So let $x_0 \in \overline{A}$.
By \eqref{item2.1.2.i}, there exists a sequence $(x_k)_{k \in \N}$ in $A$ with $x_k \to x_0$
as $k \to \infty$. But then the hypothesis implies that $x_0 \in A$.

\bis

\eqref{item2.1.2.iii} Suppose that $A$ is open and let $(x_k)_{k \in \N}$ be a sequence in $X$
with limit $x_0 \in A$. Since $A$ is open, there exists an $r > 0$ such that $B_r(x_0) \subseteq A$.
Moreover, there exists a $N \in \N$ such that $d(x_k,x_0) < r$ for all $k \ge N$.
Thus $x_k \in A$ for $k \ge N$.

Conversely, suppose that $A$ is not open. Then $X \backslash A$ is not closed
by Theorem \ref{thm:complement}. According to \eqref{item2.1.2.ii}, there exists
a sequence $(x_k)_{k \in \N}$ in $X \backslash A$ that converges to an $x_0 \notin X\setminus A$, so $x_0\in A$.
But for this sequence, we have $x_k \not\in A$ for all $k \in \N$.
\end{proof}



