\documentclass[12pt,a4paper,onecolumn]{article}
\usepackage[T1]{fontenc} 
\usepackage[a4paper]{geometry} 
\usepackage[english]{babel} 
\usepackage{amsmath}
\usepackage{amssymb} 
\usepackage{amsfonts} 
\usepackage[mathscr]{eucal}
\usepackage{eufrak} 
\usepackage{amsthm} 
\usepackage{longtable} 
\usepackage{graphicx} 
\usepackage{verbatim} 
\usepackage{url} 
\newtheorem{theorem}{Theorem}[section]
\title{Starting with AMS}
\author{Emma Cliffe}
\date{}
\pagestyle{headings}
\begin{document}
\maketitle
\tableofcontents
\newpage

\section{Textmode symbols and structures}

\subsection{Additional symbols}
\noindent
Mathmode and textmode: \ddag  \P  \dots  \S  \dag  \pounds

\subsection{AMS theorems}

\noindent
A baseline of text which is a single line long in 12pt font with no indent applied.

\begin{theorem}[Title of the theorem]
This is a theorem that has been produced WITH the AMS theorem environment and package.
\end{theorem}

\noindent
A baseline of text which is a single line long in 12pt font with no indent applied.
\end{document}
