%\documentclass[12pt,a4paper,onecolumn]{article}
\ifboolexpr{togl {web} or togl{clearprint}}{}{\usepackage[T1]{fontenc}} 
\ifboolexpr{togl {web} or togl{clearprint}}{}{\usepackage[a4paper]{geometry}} 
\ifboolexpr{togl {web} or togl{clearprint}}{}{\usepackage[english]{babel}} 
\ifboolexpr{togl {web} or togl{clearprint}}{}{\usepackage{amsmath}}
\ifboolexpr{togl {web} or togl{clearprint}}{}{\usepackage{amssymb}} 
\ifboolexpr{togl {web} or togl{clearprint}}{}{\usepackage{amsfonts}} 
\usepackage[mathscr]{eucal}
\usepackage{eufrak} 
\ifboolexpr{togl {web} or togl{clearprint}}{}{\usepackage{amsthm}} 
\usepackage{longtable} 
\ifboolexpr{togl {web} or togl{clearprint}}{}{\usepackage{graphicx}} 
\usepackage{spverbatim} 
\usepackage{url} 
\newtheorem{theorem}{Theorem}[section]
\newtheorem{corollary}[theorem]{Corollary}
\theoremstyle{definition}
\newtheorem{definition}{Definition}[section]
\theoremstyle{remark}
\newtheorem*{note}{Note}
\newenvironment{ProofOri}{\noindent{\bf Proof.}\hspace*{1em}}{\qed\par}

\title{AMS theorems and newenvironment}
\author{Emma Cliffe}
\date{}
\pagestyle{headings}
\usepackage{calc}
\usepackage{longtable}
\usepackage{tabu}
\usepackage{breqn}
\setlength{\arraycolsep}{0.800000em}
\renewcommand{\arraystretch}{1.400000}
\begin{document}
\renewcommand{\baselinestretch}{1.250000}
\selectfont
\setlength{\parskip}{1.0\baselineskip}
\ifdefstring{\fleqn}{fleqn}{
\belowdisplayskip=-10pt plus2pt
\abovedisplayskip=-10pt plus2pt
}{}


\maketitle

\section{Some theorem and a new environment}

\begin{theorem}
This is a numbered theorem (within the section) in the standard style and hence this text is in italics. 
\end{theorem}

\begin{corollary}
This is a numbered corollary (numbered with theorems) also in the standard style.
\end{corollary}

\subsection{Built-in and self-defined proof environments}

\begin{proof}
This is the built-in proof environment.
\end{proof}

\begin{ProofOri}
This a self-made proof environment.
\end{ProofOri}

\begin{definition}
This is a numbered definition (numbered within section but not with theorems) and in the ``definition'' style. The title and number should be bold but the rest of the text should be normal font.
\end{definition}

\begin{note}
This is an unnumbered note and is in the ``remark'' style.
\end{note}

\section{Second section for numbering demo}

\begin{theorem}
And so we see that the numbering works out as expected.
\end{theorem}

\end{document}
