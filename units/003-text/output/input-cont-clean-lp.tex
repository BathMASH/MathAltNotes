%Taken from HESTEM project vanilla tests
%\documentclass[12pt,a4paper,onecolumn]{article}
\usepackage{calc}
\usepackage{longtable}
\usepackage{tabu}
\usepackage{breqn}
\setlength{\arraycolsep}{0.800000em}
\renewcommand{\arraystretch}{1.400000}
\begin{document}
\renewcommand{\baselinestretch}{1.250000}
\selectfont
\setlength{\parskip}{1.0\baselineskip}


\noindent
A baseline of text which is one line long in 12pt font with no indent applied.

\begin{center}
Centered text.
\end{center}

\begin{flushleft}
Flush left text.
\end{flushleft}

\begin{flushright}
Flush right text.
\end{flushright}

Standard text. {\ifboolexpr{togl {web} or togl {clearprint}}{\normalsize}{\tiny} Tiny text.} {\ifboolexpr{togl {web} or togl {clearprint}}{\normalsize}{\scriptsize} Scriptsize text.} {\ifboolexpr{togl {web} or togl {clearprint}}{\normalsize}{\footnotesize} Footnotesize text.} {\ifboolexpr{togl {web} or togl {clearprint}}{\normalsize}{\small} Small text.} {\normalsize Normalsize text.} {\large large text.} {\Large Large text.} {\LARGE LARGE text.} {\huge huge text.} {\Huge Huge text.}

Standard text. \emph{Emphasized text}. \textrm{Roman text.} {\rm Roman inline.} \textsc{Small caps text.} {\sc Small caps inline} \texttt{Typewriter text.} {\tt Typewriter inline.} \textit{Italics text.} {\it Italics inline.} \textsf{Sans serif text.} {\sf San serif inline.} \textsl{Slant text.} {\sl Slant inline.}  \textbf{Bold text.} {\bf Bold inline.} \textbf{A combination of bold and \textit{italic text.}} {\bf A combination inline of bold {\it and inline italics}.}
\end{document}

