%Taken from HESTEM project vanilla tests
%\documentclass[12pt,a4paper,onecolumn]{article}
\ifboolexpr{togl {web} or togl{clearprint}}{}{\usepackage[T1]{fontenc}}
\newtheorem{theorem}{Theorem}[section]
\ifboolexpr{togl {web} or togl{clearprint}}{}{\usepackage[a4paper]{geometry}}
\usepackage{longtable} 
\title{Basic structures}
\author{Emma Cliffe}
\date{}
\begin{document}
\maketitle
\noindent
A baseline of text which is a single line long in 12pt font with no indent applied.

\begin{quote}
In the quote environment [paragraphs] are indicated with more vertical spacing between them. 

Additional vertical spacing is inserted above and below the displayed text to separate it visually from the the normal text.
\end{quote}

A baseline of text to show the height change in the above and below environments. This line was indented though to show off the next environment. The quotations are from ``A Guide to \LaTeX'' \cite{KopkaDaly}

\begin{quotation}
In the quotation environment, paragraphs are marked by extra indentation of the first line. 

The quotation environment is only really meaningful when the regular text makes use of first-line indentation to show off new paragraphs.
\end{quotation}

%ignored verse enivironment

\noindent
A baseline of text which is a single line long in 12pt font with no indent applied.

\begin{itemize}
\item An itemized list
\item Using standard itemize
\begin{itemize}
\item With a level 2 sub-point
\begin{itemize}
\item With a level 3 sub-point
\begin{itemize}
\item With a level 4 sub-point
\end{itemize}
\end{itemize}
\end{itemize}
\item[\&] Or I can control the marker manually
\end{itemize}

\noindent
A baseline of text which is a single line long in 12pt font with no indent applied.

Same list with redefinition using renewcommand of the labels labelitem(i-iv)
\begin{itemize}
\renewcommand{\labelitemi}{*}
\renewcommand{\labelitemii}{**}
\renewcommand{\labelitemiii}{***}
\renewcommand{\labelitemiv}{****}
\item An itemized list
\item Using standard itemize
\begin{itemize}
\item With a level 2 sub-point
\begin{itemize}
\item With a level 3 sub-point
\begin{itemize}
\item With a level 4 sub-point
\end{itemize}
\end{itemize}
\end{itemize}
\item[\&] Or I can control the marker manually
\end{itemize}

\begin{itemize}
\item Because the renewcommands were contained in the environment they are not global
\end{itemize}

\noindent
A baseline of text which is a single line long in 12pt font with no indent applied.

\begin{enumerate}
\item An enumerated list
\item Using standard enumerate
\begin{enumerate}
\item With a level 2 sub-point
\begin{enumerate}
\item With a level 3 sub-point
\begin{enumerate}
\item With a level 4 sub-point
\end{enumerate}
\end{enumerate}
\end{enumerate}
\item[\&] Or I can control the marker
\end{enumerate}

\noindent
A baseline of text which is a single line long in 12pt font with no indent applied.

Same list with redefinition using renewcommand of the labels labelenum(i-iv) by application of arabic, roman, Roman, alph or Alph
\begin{enumerate}
\renewcommand{\labelenumi}{\Roman{enumi}.}
\renewcommand{\labelenumii}{\roman{enumii}.}
\renewcommand{\labelenumiii}{\Alph{enumiii}.}
\renewcommand{\labelenumiv}{\alph{enumiv}.}
\item An enumerated list
\item Using standard enumerate
\begin{enumerate}
\item With a level 2 sub-point
\begin{enumerate}
\item With a level 3 sub-point
\begin{enumerate}
\item With a level 4 sub-point
\end{enumerate}
\end{enumerate}
\end{enumerate}
\item[\&] Or I can control the marker
\end{enumerate}

\begin{enumerate}
\item Because the renewcommands were contained in the environment they are not global
\end{enumerate}

\noindent
A baseline of text which is a single line long in 12pt font with no indent applied.

\begin{description}
\item[first] The marker is a description
\item[second] in the description environment
\item But it is optional
\end{description}

%ignored the bibliography (non bibtex) for now)
%ignored generalised lists for now

\noindent
A baseline of text which is a single line long in 12pt font with no indent applied.

\begin{theorem}[Title of the theorem]
This is a theorem that has been produced without the AMS theorem environment or package
\end{theorem}

\noindent
A baseline of text which is a single line long in 12pt font with no indent applied.

\begin{tabbing}
There is the \=tabbing environment which lines\\
\>this with tabbing above \= and\+\\
\>this with and\\
and this with tabbing again\-\\
until I backwards tab
\end{tabbing}

\noindent
A baseline of text which is a single line long in 12pt font with no indent applied.

\bigskip

\noindent
\fbox{This text is framed in a box. The width is determined by the text.}

\bigskip

\noindent
\framebox[0.5\textwidth]{This box is 0.5 textwidth wide}

\bigskip

\noindent
A baseline of text which is a single line long in 12pt font with no indent applied.

\bigskip

\noindent
\parbox{0.5\textwidth}{This is a parbox half the textwidth of the page. \par This is the second paragraph in the box.}

\bigskip

\noindent
\fbox{\parbox{0.5\textwidth}{This is a parbox half the textwidth of the page. \par This is the second paragraph in the box.}}

\bigskip

\noindent
\begin{minipage}{0.5\textwidth}This is a minipage half the textwidth of the page. \par This is the second paragraph in the minipage.\end{minipage}\hfill
\begin{minipage}{0.3\textwidth}A second minipage is over here...\end{minipage}

\bigskip

\noindent
\rule{\textwidth}{\baselineskip}

\bigskip

%We want to allow tabular, longtable (maybe later tabular* and maybe tabularx)
\begin{table}[t]\label{table}
\begin{tabular}{|l*{2}{|c|}r|@{insert}|p{2cm}}
%\begin{tabu} to 0.5\textwidth {|X*{2}{|X|}X|@{insert}|X}
\hline
a & b & c & d & abcde\\
\cline{1-4}
\multicolumn{4}{c}{abcd}\vline & abcde\\
\hline
\(\frac{a}{e}\) & \(\frac{b}{e}\) & \(\frac{c}{e}\) & \(\frac{d}{e}\) & \(\alpha\beta\gamma\delta\epsilon\) \\
\hline
\end{tabular}
\caption{This is a table}
\end{table}

\noindent
This is just below where the floating table \ref{table} was defined. It should appear at the top of either this page or the page after this. 

\bigskip

\begin{center}
\renewcommand{\arraystretch}{2}
%\begin{longtabu} to \textwidth {*{3}{|X|}}
\begin{longtable}{*{3}{|l|}}
\hline
\textbf{First} & \textbf{Second} & \textbf{Third} \\
\hline
\endfirsthead
\multicolumn{3}{c}%
{\tablename\ \thetable\ -- \textit{Continued from previous page}} \\
\hline
\textbf{First} & \textbf{Second} & \textbf{Third} \\
\hline
\endhead
\hline \multicolumn{3}{r}{\textit{Continued on next page}} \\
\endfoot
\hline
\endlastfoot
\hline
This is the first line & & \\
\hline
This is the second line & \(1 \times 2\) & \\
\hline
This is the third line & \(1 \times 2 \times 3\) & \(6\)\\
\hline
This is the fourth line & \(1 \times 2 \times 3 \times 4\) & \(24\)\\
\hline
This is the fifth line & \(1 \times 2 \times 3 \times 4 \times 5\) & \(120\)\\
\hline
This is the sixth line & \(1 \times 2 \times 3 \times 4 \times 5 \times 6\) & \(720\)\\
\hline
This is the seventh line & \(1 \times 2 \times 3 \times 4 \times 5 \times 6 \times 7\) & \(5040\)\\
\hline
This is the eighth line & \(1 \times 2 \times 3 \times 4 \times 5 \times 6 \times 7 \times 8\) & \(40320\)\\
\hline
& The & End\\
\hline
\end{longtable}
\end{center}

\bigskip

\begin{verbatim}
This text should be printed verbatim with a linebreak here
  then two spaces at the start of this line which breaks here
> this line has a prompt at the start and now some braces {}
\end{verbatim}

\bigskip

This next \verb=verbatim= but with spaces shown\footnote{The word verbatim used inline verbatim.}. 

\bigskip
A piece \verb=of verbatim text that we are using to test line breaking=.

%spverbatim does not support a starred form 
%\begin{verbatim*}
%This text should be printed verbatim with a linebreak here
%  then two spaces at the start of this line which breaks here
%> this line has a prompt at the start and now some braces {}
%\end{verbatim*}

\bigskip

\noindent
A baseline of text which is a single line long in 12pt font with no indent applied.\marginpar{Note\\ in the\\ margin.}

\vspace{4\baselineskip}

\noindent
A baseline of text which is a single line long in 12pt font with no indent applied.\marginpar{\rule[-1ex]{0.3em}{4ex}}

% \thebibliography
\begin{thebibliography}{99}
\bibitem{KopkaDaly} Kopka, H. and Daly, P., \textit{A Guide to \LaTeX}. Pearson Education Ltd., 1999
\end{thebibliography}

\end{document}
